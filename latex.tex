\def\version{0.0.0}
\def\numberofcommands{55}
\def\numberoftraits{11}
\def\numberoffilehandlers{21}
\def\ebicommandlist{\begin{itemize}
\item\texttt{Ebi analyse all-traces} or \texttt{Ebi ana all} (Section~\ref{command:Ebi analyse all-traces})
\item\texttt{Ebi analyse completeness} or \texttt{Ebi ana comp} (Section~\ref{command:Ebi analyse completeness})
\item\texttt{Ebi analyse coverage} or \texttt{Ebi ana cov} (Section~\ref{command:Ebi analyse coverage})
\item\texttt{Ebi analyse directly-follows-edge-difference} or \texttt{Ebi ana dfgedi} (Section~\ref{command:Ebi analyse directly-follows-edge-difference})
\item\texttt{Ebi analyse medoid} or \texttt{Ebi ana med} (Section~\ref{command:Ebi analyse medoid})
\item\texttt{Ebi analyse minimum-probability-traces} or \texttt{Ebi ana minprob} (Section~\ref{command:Ebi analyse minimum-probability-traces})
\item\texttt{Ebi analyse mode} or \texttt{Ebi ana mode} (Section~\ref{command:Ebi analyse mode})
\item\texttt{Ebi analyse most-likely-traces} or \texttt{Ebi ana mostlikely} (Section~\ref{command:Ebi analyse most-likely-traces})
\item\texttt{Ebi analyse variety} or \texttt{Ebi ana var} (Section~\ref{command:Ebi analyse variety})
\item\texttt{Ebi analyse-non-stochastic any-traces} or \texttt{Ebi anans at} (Section~\ref{command:Ebi analyse-non-stochastic any-traces})
\item\texttt{Ebi analyse-non-stochastic bounded} or \texttt{Ebi anans bnd} (Section~\ref{command:Ebi analyse-non-stochastic bounded})
\item\texttt{Ebi analyse-non-stochastic cluster} or \texttt{Ebi anans clus} (Section~\ref{command:Ebi analyse-non-stochastic cluster})
\item\texttt{Ebi analyse-non-stochastic executions} or \texttt{Ebi anans exe} (Section~\ref{command:Ebi analyse-non-stochastic executions})
\item\texttt{Ebi analyse-non-stochastic infinitely-many-traces} or \texttt{Ebi anans inft} (Section~\ref{command:Ebi analyse-non-stochastic infinitely-many-traces})
\item\texttt{Ebi analyse-non-stochastic medoid} or \texttt{Ebi anans med} (Section~\ref{command:Ebi analyse-non-stochastic medoid})
\item\texttt{Ebi association all-trace-attributes} or \texttt{Ebi asso atts} (Section~\ref{command:Ebi association all-trace-attributes})
\item\texttt{Ebi association trace-attribute} or \texttt{Ebi asso att} (Section~\ref{command:Ebi association trace-attribute})
\item\texttt{Ebi conformance earth-movers-stochastic-conformance} or \texttt{Ebi conf emsc} (Section~\ref{command:Ebi conformance earth-movers-stochastic-conformance})
\item\texttt{Ebi conformance earth-movers-stochastic-conformance-sample} or \texttt{Ebi conf emsc-sample} (Section~\ref{command:Ebi conformance earth-movers-stochastic-conformance-sample})
\item\texttt{Ebi conformance entropic-relevance} or \texttt{Ebi conf er} (Section~\ref{command:Ebi conformance entropic-relevance})
\item\texttt{Ebi conformance jensen-shannon} or \texttt{Ebi conf jssc} (Section~\ref{command:Ebi conformance jensen-shannon})
\item\texttt{Ebi conformance jensen-shannon-sample} or \texttt{Ebi conf jssc-sample} (Section~\ref{command:Ebi conformance jensen-shannon-sample})
\item\texttt{Ebi conformance unit-earth-movers-stochastic-conformance} or \texttt{Ebi conf uemsc} (Section~\ref{command:Ebi conformance unit-earth-movers-stochastic-conformance})
\item\texttt{Ebi conformance-non-stochastic alignments} or \texttt{Ebi confns ali} (Section~\ref{command:Ebi conformance-non-stochastic alignments})
\item\texttt{Ebi conformance-non-stochastic escaping-edges-precision} or \texttt{Ebi confns eep} (Section~\ref{command:Ebi conformance-non-stochastic escaping-edges-precision})
\item\texttt{Ebi conformance-non-stochastic set-alignments} or \texttt{Ebi confns setali} (Section~\ref{command:Ebi conformance-non-stochastic set-alignments})
\item\texttt{Ebi conformance-non-stochastic trace-fitness} or \texttt{Ebi confns tfit} (Section~\ref{command:Ebi conformance-non-stochastic trace-fitness})
\item\texttt{Ebi convert finite-stochastic-language} or \texttt{Ebi conv slang} (Section~\ref{command:Ebi convert finite-stochastic-language})
\item\texttt{Ebi convert labelled-petri-net} or \texttt{Ebi conv lpn} (Section~\ref{command:Ebi convert labelled-petri-net})
\item\texttt{Ebi convert stochastic-finite-deterministic-automaton} or \texttt{Ebi conv sdfa} (Section~\ref{command:Ebi convert stochastic-finite-deterministic-automaton})
\item\texttt{Ebi discover alignments} or \texttt{Ebi disc ali} (Section~\ref{command:Ebi discover alignments})
\item\texttt{Ebi discover directly-follows-graph} or \texttt{Ebi disc dfg} (Section~\ref{command:Ebi discover directly-follows-graph})
\item\texttt{Ebi discover occurrence labelled-petri-net} or \texttt{Ebi disc occ lpn} (Section~\ref{command:Ebi discover occurrence labelled-petri-net})
\item\texttt{Ebi discover occurrence process-tree} or \texttt{Ebi disc occ ptree} (Section~\ref{command:Ebi discover occurrence process-tree})
\item\texttt{Ebi discover uniform labelled-petri-net} or \texttt{Ebi disc uni lpn} (Section~\ref{command:Ebi discover uniform labelled-petri-net})
\item\texttt{Ebi discover uniform process-tree} or \texttt{Ebi disc uni ptree} (Section~\ref{command:Ebi discover uniform process-tree})
\item\texttt{Ebi discover-non-stochastic flower deterministic-finite-automaton} or \texttt{Ebi dins flw dfa} (Section~\ref{command:Ebi discover-non-stochastic flower deterministic-finite-automaton})
\item\texttt{Ebi discover-non-stochastic flower process-tree} or \texttt{Ebi dins flw ptree} (Section~\ref{command:Ebi discover-non-stochastic flower process-tree})
\item\texttt{Ebi discover-non-stochastic prefix-tree deterministic-finite-automaton} or \texttt{Ebi dins pfxt dfa} (Section~\ref{command:Ebi discover-non-stochastic prefix-tree deterministic-finite-automaton})
\item\texttt{Ebi discover-non-stochastic prefix-tree process-tree} or \texttt{Ebi dins pfxt tree} (Section~\ref{command:Ebi discover-non-stochastic prefix-tree process-tree})
\item\texttt{Ebi information} or \texttt{Ebi info} (Section~\ref{command:Ebi information})
\item\texttt{Ebi itself graph} or \texttt{Ebi it graph} (Section~\ref{command:Ebi itself graph})
\item\texttt{Ebi itself html} or \texttt{Ebi it html} (Section~\ref{command:Ebi itself html})
\item\texttt{Ebi itself java} or \texttt{Ebi it java} (Section~\ref{command:Ebi itself java})
\item\texttt{Ebi itself logo} or \texttt{Ebi it log} (Section~\ref{command:Ebi itself logo})
\item\texttt{Ebi itself manual} or \texttt{Ebi it man} (Section~\ref{command:Ebi itself manual})
\item\texttt{Ebi probability explain-trace} or \texttt{Ebi prob exptra} (Section~\ref{command:Ebi probability explain-trace})
\item\texttt{Ebi probability log} or \texttt{Ebi prob log} (Section~\ref{command:Ebi probability log})
\item\texttt{Ebi probability trace} or \texttt{Ebi prob trac} (Section~\ref{command:Ebi probability trace})
\item\texttt{Ebi sample} or \texttt{Ebi sam} (Section~\ref{command:Ebi sample})
\item\texttt{Ebi test bootstrap-test} or \texttt{Ebi tst btst} (Section~\ref{command:Ebi test bootstrap-test})
\item\texttt{Ebi test log-categorical-attribute} or \texttt{Ebi tst lcat} (Section~\ref{command:Ebi test log-categorical-attribute})
\item\texttt{Ebi validate} or \texttt{Ebi vali} (Section~\ref{command:Ebi validate})
\item\texttt{Ebi visualise graph} or \texttt{Ebi vis graph} (Section~\ref{command:Ebi visualise graph})
\item\texttt{Ebi visualise text} or \texttt{Ebi vis txt} (Section~\ref{command:Ebi visualise text})
\end{itemize}}
\def\ebicommands{
\subsection{\texttt{Ebi analyse all-traces}}
\label{command:Ebi analyse all-traces}
Alias: \texttt{Ebi ana all}.\\
List all traces of a stohastic model.
Models containing loops and unbounded models are not supported and the computation will run forever.\\
\begin{tabularx}{\linewidth}{lX}
\toprule
Parameter \\\midrule
<\texttt{FILE}>&Any object with deterministic stochastic semantics.\\
&\textit{Mandatory:} \quad yes, though it can be given on STDIN by giving a `-' on the command line.\\
&\textit{Accepted values:}\quad \hyperref[filehandler:event log]{event log (.xes)}, \hyperref[filehandler:stochastic deterministic finite automaton]{stochastic deterministic finite automaton (.sdfa)}, \hyperref[filehandler:stochastic directly follows model]{stochastic directly follows model (.sdfm)}, \hyperref[filehandler:compressed event log]{compressed event log (.xes.gz)}, \hyperref[filehandler:stochastic labelled Petri net]{stochastic labelled Petri net (.slpn)}, \hyperref[filehandler:directly follows graph]{directly follows graph (.dfg)}, \hyperref[filehandler:stochastic process tree]{stochastic process tree (.sptree)} and \hyperref[filehandler:finite stochastic language]{finite stochastic language (.slang)}.\\
\texttt{-o} or \texttt{--output} <\texttt{FILE}> &
The file to which the results must be written. Based on the file extension, Ebi will output either a deterministic finite automaton (.dfa -- Section~\ref{filehandler:deterministic finite automaton}), a finite language (.lang -- Section~\ref{filehandler:finite language}), a finite stochastic language (.slang -- Section~\ref{filehandler:finite stochastic language}) or a stochastic deterministic finite automaton (.sdfa -- Section~\ref{filehandler:stochastic deterministic finite automaton}).
If the parameter is not given, the results will be written to STDOUT as a finite stochastic language (.slang).\\
&\textit{Mandatory:} \quad no\\
\texttt{-a} or \texttt{--approximate} & Use approximate arithmetic instead of exact arithmetic.\\
&\textit{Mandatory:}\quad no\\
\bottomrule
\end{tabularx}
\noindent Output: finite stochastic language, which can be written as a deterministic finite automaton (.dfa -- Section~\ref{filehandler:deterministic finite automaton}), a finite language (.lang -- Section~\ref{filehandler:finite language}), a finite stochastic language (.slang -- Section~\ref{filehandler:finite stochastic language}) or a stochastic deterministic finite automaton (.sdfa -- Section~\ref{filehandler:stochastic deterministic finite automaton}).
\\This command is not available in Java and ProM.
\subsection{\texttt{Ebi analyse completeness}}
\label{command:Ebi analyse completeness}
Alias: \texttt{Ebi ana comp}.\\
Estimate the completeness of an event log using species discovery.\\
More information: ~\cite{DBLP:conf/icpm/KabierskiRW23}.\\
\begin{tabularx}{\linewidth}{lX}
\toprule
Parameter \\\midrule
<\texttt{FILE}>&An event log.\\
&\textit{Mandatory:} \quad yes, though it can be given on STDIN by giving a `-' on the command line.\\
&\textit{Accepted values:}\quad \hyperref[filehandler:compressed event log]{compressed event log (.xes.gz)} and \hyperref[filehandler:event log]{event log (.xes)}.\\
\texttt{-o} or \texttt{--output} <\texttt{FILE}> &
The fraction file to which the result must be written. If the parameter is not given, the results will be written to STDOUT.\\
&\textit{Mandatory:} \quad no\\
\texttt{-a} or \texttt{--approximate} & Use approximate arithmetic instead of exact arithmetic.\\
&\textit{Mandatory:}\quad no\\
\bottomrule
\end{tabularx}
\noindent Output: fraction, which can be written as a fraction.
\\This command is available in Java and ProM.
\subsection{\texttt{Ebi analyse coverage}}
\label{command:Ebi analyse coverage}
Alias: \texttt{Ebi ana cov}.\\
Find the most-likely traces that together cover the given minimum probability.
Will return a finite stochastic language with the extracted traces.
The computation may not terminate if the model has non-decreasing livelocks, or if the model is unbounded and this unboundedness can be triggered using silent transitions, or if the model has livelocks.\\
\begin{tabularx}{\linewidth}{lX}
\toprule
Parameter \\\midrule
<\texttt{FILE}>&Any object with deterministic stochastic semantics.\\
&\textit{Mandatory:} \quad yes, though it can be given on STDIN by giving a `-' on the command line.\\
&\textit{Accepted values:}\quad \hyperref[filehandler:stochastic labelled Petri net]{stochastic labelled Petri net (.slpn)}, \hyperref[filehandler:event log]{event log (.xes)}, \hyperref[filehandler:stochastic directly follows model]{stochastic directly follows model (.sdfm)}, \hyperref[filehandler:directly follows graph]{directly follows graph (.dfg)}, \hyperref[filehandler:finite stochastic language]{finite stochastic language (.slang)}, \hyperref[filehandler:stochastic deterministic finite automaton]{stochastic deterministic finite automaton (.sdfa)}, \hyperref[filehandler:compressed event log]{compressed event log (.xes.gz)} and \hyperref[filehandler:stochastic process tree]{stochastic process tree (.sptree)}.\\
<\texttt{MINIMUM\_COVERAGE}>&The minimum probability that a trace should have to be included.\\
&\textit{Mandatory:} \quad yes, though it can be given on STDIN by giving a `-' on the command line.\\
&\textit{Accepted values:}\quad fraction between 0 and 1.\\
\texttt{-o} or \texttt{--output} <\texttt{FILE}> &
The file to which the results must be written. Based on the file extension, Ebi will output either a deterministic finite automaton (.dfa -- Section~\ref{filehandler:deterministic finite automaton}), a finite language (.lang -- Section~\ref{filehandler:finite language}), a finite stochastic language (.slang -- Section~\ref{filehandler:finite stochastic language}) or a stochastic deterministic finite automaton (.sdfa -- Section~\ref{filehandler:stochastic deterministic finite automaton}).
If the parameter is not given, the results will be written to STDOUT as a finite stochastic language (.slang).\\
&\textit{Mandatory:} \quad no\\
\texttt{-a} or \texttt{--approximate} & Use approximate arithmetic instead of exact arithmetic.\\
&\textit{Mandatory:}\quad no\\
\bottomrule
\end{tabularx}
\noindent Output: finite stochastic language, which can be written as a deterministic finite automaton (.dfa -- Section~\ref{filehandler:deterministic finite automaton}), a finite language (.lang -- Section~\ref{filehandler:finite language}), a finite stochastic language (.slang -- Section~\ref{filehandler:finite stochastic language}) or a stochastic deterministic finite automaton (.sdfa -- Section~\ref{filehandler:stochastic deterministic finite automaton}).
\\This command is not available in Java and ProM.
\subsection{\texttt{Ebi analyse directly-follows-edge-difference}}
\label{command:Ebi analyse directly-follows-edge-difference}
Alias: \texttt{Ebi ana dfgedi}.\\
The number of edges that differ between two directly follows graphs.\\
\begin{tabularx}{\linewidth}{lX}
\toprule
Parameter \\\midrule
<\texttt{DFG\_1}>&A directly follows graph.\\
&\textit{Mandatory:} \quad yes, though it can be given on STDIN by giving a `-' on the command line.\\
&\textit{Accepted values:}\quad \hyperref[filehandler:directly follows graph]{directly follows graph (.dfg)}.\\
<\texttt{DFG\_2}>&A directly follows graph.\\
&\textit{Mandatory:} \quad yes, though it can be given on STDIN by giving a `-' on the command line.\\
&\textit{Accepted values:}\quad \hyperref[filehandler:directly follows graph]{directly follows graph (.dfg)}.\\
\texttt{-o} or \texttt{--output} <\texttt{FILE}> &
The fraction file to which the result must be written. If the parameter is not given, the results will be written to STDOUT.\\
&\textit{Mandatory:} \quad no\\
\texttt{-a} or \texttt{--approximate} & Use approximate arithmetic instead of exact arithmetic.\\
&\textit{Mandatory:}\quad no\\
\bottomrule
\end{tabularx}
\noindent Output: fraction, which can be written as a fraction.
\\This command is not available in Java and ProM.
\subsection{\texttt{Ebi analyse medoid}}
\label{command:Ebi analyse medoid}
Alias: \texttt{Ebi ana med}.\\
Find the traces with the lowest average normalised Levenshtein distance to the other traces.
If there are more than one such trace, an arbitrary one is returned.\\
\begin{tabularx}{\linewidth}{lX}
\toprule
Parameter \\\midrule
<\texttt{FILE}>&Any object with a finite stochastic language.\\
&\textit{Mandatory:} \quad yes, though it can be given on STDIN by giving a `-' on the command line.\\
&\textit{Accepted values:}\quad \hyperref[filehandler:event log]{event log (.xes)}, \hyperref[filehandler:finite stochastic language]{finite stochastic language (.slang)} and \hyperref[filehandler:compressed event log]{compressed event log (.xes.gz)}.\\
<\texttt{NUMBER\_OF\_TRACES}>&The number of traces that should be extracted.\\
&\textit{Mandatory:} \quad no: if no value is provided, a default of 1 will be used. It can also be provided on STDIN by giving a `-' on the command line.\\
&\textit{Accepted values:}\quad integer.\\
\texttt{-o} or \texttt{--output} <\texttt{FILE}> &
The file to which the results must be written. Based on the file extension, Ebi will output either a deterministic finite automaton (.dfa -- Section~\ref{filehandler:deterministic finite automaton}) or a finite language (.lang -- Section~\ref{filehandler:finite language}).
If the parameter is not given, the results will be written to STDOUT as a finite language (.lang).\\
&\textit{Mandatory:} \quad no\\
\texttt{-a} or \texttt{--approximate} & Use approximate arithmetic instead of exact arithmetic.\\
&\textit{Mandatory:}\quad no\\
\bottomrule
\end{tabularx}
\noindent Output: finite language, which can be written as a deterministic finite automaton (.dfa -- Section~\ref{filehandler:deterministic finite automaton}) or a finite language (.lang -- Section~\ref{filehandler:finite language}).
\\This command is not available in Java and ProM.
\subsection{\texttt{Ebi analyse minimum-probability-traces}}
\label{command:Ebi analyse minimum-probability-traces}
Alias: \texttt{Ebi ana minprob}.\\
Find all traces that have a given minimum probability.
Will return a finate stochastic language with the extracted traces.
Will return an error if there are no such traces.
The computation may not terminate if the model is unbounded and this unboundedness can be triggered using silent transitions.\\
\begin{tabularx}{\linewidth}{lX}
\toprule
Parameter \\\midrule
<\texttt{FILE}>&Any object with deterministic stochastic semantics.\\
&\textit{Mandatory:} \quad yes, though it can be given on STDIN by giving a `-' on the command line.\\
&\textit{Accepted values:}\quad \hyperref[filehandler:finite stochastic language]{finite stochastic language (.slang)}, \hyperref[filehandler:stochastic process tree]{stochastic process tree (.sptree)}, \hyperref[filehandler:event log]{event log (.xes)}, \hyperref[filehandler:stochastic labelled Petri net]{stochastic labelled Petri net (.slpn)}, \hyperref[filehandler:directly follows graph]{directly follows graph (.dfg)}, \hyperref[filehandler:stochastic deterministic finite automaton]{stochastic deterministic finite automaton (.sdfa)}, \hyperref[filehandler:compressed event log]{compressed event log (.xes.gz)} and \hyperref[filehandler:stochastic directly follows model]{stochastic directly follows model (.sdfm)}.\\
<\texttt{MINIMUM\_PROBABILITY}>&The minimum probability that a trace should have to be included.\\
&\textit{Mandatory:} \quad yes, though it can be given on STDIN by giving a `-' on the command line.\\
&\textit{Accepted values:}\quad fraction between 0 and 1.\\
\texttt{-o} or \texttt{--output} <\texttt{FILE}> &
The file to which the results must be written. Based on the file extension, Ebi will output either a deterministic finite automaton (.dfa -- Section~\ref{filehandler:deterministic finite automaton}), a finite language (.lang -- Section~\ref{filehandler:finite language}), a finite stochastic language (.slang -- Section~\ref{filehandler:finite stochastic language}) or a stochastic deterministic finite automaton (.sdfa -- Section~\ref{filehandler:stochastic deterministic finite automaton}).
If the parameter is not given, the results will be written to STDOUT as a finite stochastic language (.slang).\\
&\textit{Mandatory:} \quad no\\
\texttt{-a} or \texttt{--approximate} & Use approximate arithmetic instead of exact arithmetic.\\
&\textit{Mandatory:}\quad no\\
\bottomrule
\end{tabularx}
\noindent Output: finite stochastic language, which can be written as a deterministic finite automaton (.dfa -- Section~\ref{filehandler:deterministic finite automaton}), a finite language (.lang -- Section~\ref{filehandler:finite language}), a finite stochastic language (.slang -- Section~\ref{filehandler:finite stochastic language}) or a stochastic deterministic finite automaton (.sdfa -- Section~\ref{filehandler:stochastic deterministic finite automaton}).
\\This command is not available in Java and ProM.
\subsection{\texttt{Ebi analyse mode}}
\label{command:Ebi analyse mode}
Find the trace with the highest probability.
If there is more than one trace with the highest probability, an arbitrary choice is made which one to return.
The computation may run forever if the model is unbounded.
Equivalent to `Ebi evaluate mostlikely 1`.
Computation is more efficient for a model with a finite stochastic language.\\
\begin{tabularx}{\linewidth}{lX}
\toprule
Parameter \\\midrule
<\texttt{FILE}>&Any object with deterministic stochastic semantics.\\
&\textit{Mandatory:} \quad yes, though it can be given on STDIN by giving a `-' on the command line.\\
&\textit{Accepted values:}\quad \hyperref[filehandler:stochastic process tree]{stochastic process tree (.sptree)}, \hyperref[filehandler:directly follows graph]{directly follows graph (.dfg)}, \hyperref[filehandler:event log]{event log (.xes)}, \hyperref[filehandler:stochastic deterministic finite automaton]{stochastic deterministic finite automaton (.sdfa)}, \hyperref[filehandler:stochastic directly follows model]{stochastic directly follows model (.sdfm)}, \hyperref[filehandler:stochastic labelled Petri net]{stochastic labelled Petri net (.slpn)}, \hyperref[filehandler:finite stochastic language]{finite stochastic language (.slang)} and \hyperref[filehandler:compressed event log]{compressed event log (.xes.gz)}.\\
\texttt{-o} or \texttt{--output} <\texttt{FILE}> &
The file to which the results must be written. Based on the file extension, Ebi will output either a deterministic finite automaton (.dfa -- Section~\ref{filehandler:deterministic finite automaton}), a finite language (.lang -- Section~\ref{filehandler:finite language}), a finite stochastic language (.slang -- Section~\ref{filehandler:finite stochastic language}) or a stochastic deterministic finite automaton (.sdfa -- Section~\ref{filehandler:stochastic deterministic finite automaton}).
If the parameter is not given, the results will be written to STDOUT as a finite stochastic language (.slang).\\
&\textit{Mandatory:} \quad no\\
\texttt{-a} or \texttt{--approximate} & Use approximate arithmetic instead of exact arithmetic.\\
&\textit{Mandatory:}\quad no\\
\bottomrule
\end{tabularx}
\noindent Output: finite stochastic language, which can be written as a deterministic finite automaton (.dfa -- Section~\ref{filehandler:deterministic finite automaton}), a finite language (.lang -- Section~\ref{filehandler:finite language}), a finite stochastic language (.slang -- Section~\ref{filehandler:finite stochastic language}) or a stochastic deterministic finite automaton (.sdfa -- Section~\ref{filehandler:stochastic deterministic finite automaton}).
\\This command is not available in Java and ProM.
\subsection{\texttt{Ebi analyse most-likely-traces}}
\label{command:Ebi analyse most-likely-traces}
Alias: \texttt{Ebi ana mostlikely}.\\
Find the given number of traces with the highest probabilities.
If there are more than one trace with the same probability, an arbitrary choice is made which one to return.
The computation may run forever if the model is unbounded.
Computation is more efficient for an object with a finite stochastic language.\\
\begin{tabularx}{\linewidth}{lX}
\toprule
Parameter \\\midrule
<\texttt{FILE}>&Any object with deterministic stochastic semantics.\\
&\textit{Mandatory:} \quad yes, though it can be given on STDIN by giving a `-' on the command line.\\
&\textit{Accepted values:}\quad \hyperref[filehandler:stochastic process tree]{stochastic process tree (.sptree)}, \hyperref[filehandler:directly follows graph]{directly follows graph (.dfg)}, \hyperref[filehandler:stochastic labelled Petri net]{stochastic labelled Petri net (.slpn)}, \hyperref[filehandler:event log]{event log (.xes)}, \hyperref[filehandler:compressed event log]{compressed event log (.xes.gz)}, \hyperref[filehandler:finite stochastic language]{finite stochastic language (.slang)}, \hyperref[filehandler:stochastic deterministic finite automaton]{stochastic deterministic finite automaton (.sdfa)} and \hyperref[filehandler:stochastic directly follows model]{stochastic directly follows model (.sdfm)}.\\
<\texttt{NUMBER\_OF\_TRACES}>&The number of traces that should be extracted.\\
&\textit{Mandatory:} \quad yes, though it can be given on STDIN by giving a `-' on the command line.\\
&\textit{Accepted values:}\quad integer above 1.\\
\texttt{-o} or \texttt{--output} <\texttt{FILE}> &
The file to which the results must be written. Based on the file extension, Ebi will output either a deterministic finite automaton (.dfa -- Section~\ref{filehandler:deterministic finite automaton}), a finite language (.lang -- Section~\ref{filehandler:finite language}), a finite stochastic language (.slang -- Section~\ref{filehandler:finite stochastic language}) or a stochastic deterministic finite automaton (.sdfa -- Section~\ref{filehandler:stochastic deterministic finite automaton}).
If the parameter is not given, the results will be written to STDOUT as a finite stochastic language (.slang).\\
&\textit{Mandatory:} \quad no\\
\texttt{-a} or \texttt{--approximate} & Use approximate arithmetic instead of exact arithmetic.\\
&\textit{Mandatory:}\quad no\\
\bottomrule
\end{tabularx}
\noindent Output: finite stochastic language, which can be written as a deterministic finite automaton (.dfa -- Section~\ref{filehandler:deterministic finite automaton}), a finite language (.lang -- Section~\ref{filehandler:finite language}), a finite stochastic language (.slang -- Section~\ref{filehandler:finite stochastic language}) or a stochastic deterministic finite automaton (.sdfa -- Section~\ref{filehandler:stochastic deterministic finite automaton}).
\\This command is not available in Java and ProM.
\subsection{\texttt{Ebi analyse variety}}
\label{command:Ebi analyse variety}
Alias: \texttt{Ebi ana var}.\\
Compute the variety of a stochastic language. That is, the average distance between two arbitrary traces in the language.\\
\begin{tabularx}{\linewidth}{lX}
\toprule
Parameter \\\midrule
<\texttt{FILE}>&An event log.\\
&\textit{Mandatory:} \quad yes, though it can be given on STDIN by giving a `-' on the command line.\\
&\textit{Accepted values:}\quad \hyperref[filehandler:finite stochastic language]{finite stochastic language (.slang)}, \hyperref[filehandler:compressed event log]{compressed event log (.xes.gz)} and \hyperref[filehandler:event log]{event log (.xes)}.\\
\texttt{-o} or \texttt{--output} <\texttt{FILE}> &
The fraction file to which the result must be written. If the parameter is not given, the results will be written to STDOUT.\\
&\textit{Mandatory:} \quad no\\
\texttt{-a} or \texttt{--approximate} & Use approximate arithmetic instead of exact arithmetic.\\
&\textit{Mandatory:}\quad no\\
\bottomrule
\end{tabularx}
\noindent Output: fraction, which can be written as a fraction.
\\This command is available in Java and ProM.
\subsection{\texttt{Ebi analyse-non-stochastic any-traces}}
\label{command:Ebi analyse-non-stochastic any-traces}
Alias: \texttt{Ebi anans at}.\\
Compute whether the model has any traces.
        Reasons for a model not to have any traces could be if the initial state is part of a livelock, or if there is no initial state.
        'true' means that the model has traces, 'false' means that the model has no traces.
        The computation may not terminate if the model is unbounded.\\
\begin{tabularx}{\linewidth}{lX}
\toprule
Parameter \\\midrule
<\texttt{MODEL}>&The model.\\
&\textit{Mandatory:} \quad yes, though it can be given on STDIN by giving a `-' on the command line.\\
&\textit{Accepted values:}\quad \hyperref[filehandler:Petri net markup language]{Petri net markup language (.pnml)}, \hyperref[filehandler:finite stochastic language]{finite stochastic language (.slang)}, \hyperref[filehandler:stochastic directly follows model]{stochastic directly follows model (.sdfm)}, \hyperref[filehandler:labelled Petri net]{labelled Petri net (.lpn)}, \hyperref[filehandler:compressed event log]{compressed event log (.xes.gz)}, \hyperref[filehandler:directly follows graph]{directly follows graph (.dfg)}, \hyperref[filehandler:deterministic finite automaton]{deterministic finite automaton (.dfa)}, \hyperref[filehandler:stochastic process tree]{stochastic process tree (.sptree)}, \hyperref[filehandler:stochastic deterministic finite automaton]{stochastic deterministic finite automaton (.sdfa)}, \hyperref[filehandler:process tree markup language]{process tree markup language (.ptml)}, \hyperref[filehandler:finite language]{finite language (.lang)}, \hyperref[filehandler:stochastic labelled Petri net]{stochastic labelled Petri net (.slpn)}, \hyperref[filehandler:directly follows model]{directly follows model (.dfm)}, \hyperref[filehandler:event log]{event log (.xes)} and \hyperref[filehandler:process tree]{process tree (.ptree)}.\\
\texttt{-o} or \texttt{--output} <\texttt{FILE}> &
The boolean file to which the result must be written. If the parameter is not given, the results will be written to STDOUT.\\
&\textit{Mandatory:} \quad no\\
\texttt{-a} or \texttt{--approximate} & Use approximate arithmetic instead of exact arithmetic.\\
&\textit{Mandatory:}\quad no\\
\bottomrule
\end{tabularx}
\noindent Output: bool, which can be written as a boolean.
\\This command is available in Java and ProM.
\subsection{\texttt{Ebi analyse-non-stochastic bounded}}
\label{command:Ebi analyse-non-stochastic bounded}
Alias: \texttt{Ebi anans bnd}.\\
Compute whether the model has a bounded state space. 
        For Petri nets, a coverability graph is computed~\cite{esparza2019petri}. 
        For other types of models, `true' is returned.\\
\begin{tabularx}{\linewidth}{lX}
\toprule
Parameter \\\midrule
<\texttt{MODEL}>&The model.\\
&\textit{Mandatory:} \quad yes, though it can be given on STDIN by giving a `-' on the command line.\\
&\textit{Accepted values:}\quad \hyperref[filehandler:Petri net markup language]{Petri net markup language (.pnml)}, \hyperref[filehandler:stochastic directly follows model]{stochastic directly follows model (.sdfm)}, \hyperref[filehandler:labelled Petri net]{labelled Petri net (.lpn)}, \hyperref[filehandler:directly follows graph]{directly follows graph (.dfg)}, \hyperref[filehandler:compressed event log]{compressed event log (.xes.gz)}, \hyperref[filehandler:stochastic process tree]{stochastic process tree (.sptree)}, \hyperref[filehandler:stochastic labelled Petri net]{stochastic labelled Petri net (.slpn)}, \hyperref[filehandler:deterministic finite automaton]{deterministic finite automaton (.dfa)}, \hyperref[filehandler:directly follows model]{directly follows model (.dfm)}, \hyperref[filehandler:event log]{event log (.xes)}, \hyperref[filehandler:finite stochastic language]{finite stochastic language (.slang)}, \hyperref[filehandler:stochastic deterministic finite automaton]{stochastic deterministic finite automaton (.sdfa)}, \hyperref[filehandler:process tree]{process tree (.ptree)}, \hyperref[filehandler:finite language]{finite language (.lang)} and \hyperref[filehandler:process tree markup language]{process tree markup language (.ptml)}.\\
\texttt{-o} or \texttt{--output} <\texttt{FILE}> &
The boolean file to which the result must be written. If the parameter is not given, the results will be written to STDOUT.\\
&\textit{Mandatory:} \quad no\\
\texttt{-a} or \texttt{--approximate} & Use approximate arithmetic instead of exact arithmetic.\\
&\textit{Mandatory:}\quad no\\
\bottomrule
\end{tabularx}
\noindent Output: bool, which can be written as a boolean.
\\This command is available in Java and ProM.
\subsection{\texttt{Ebi analyse-non-stochastic cluster}}
\label{command:Ebi analyse-non-stochastic cluster}
Alias: \texttt{Ebi anans clus}.\\
Apply k-medoid clustering: group the traces into a given number of clusters, such that the average distance of each trace to its closest medoid is minimal. The computation is random and does not take into account how often each trace occurs.\\
More information: ~\cite{DBLP:journals/is/SchubertR21}.\\
\begin{tabularx}{\linewidth}{lX}
\toprule
Parameter \\\midrule
<\texttt{LANGUAGE}>&The finite stochastic language.\\
&\textit{Mandatory:} \quad yes, though it can be given on STDIN by giving a `-' on the command line.\\
&\textit{Accepted values:}\quad \hyperref[filehandler:event log]{event log (.xes)}, \hyperref[filehandler:finite stochastic language]{finite stochastic language (.slang)}, \hyperref[filehandler:finite language]{finite language (.lang)} and \hyperref[filehandler:compressed event log]{compressed event log (.xes.gz)}.\\
<\texttt{NUMBER\_OF\_CLUSTERS}>&The number of clusters.\\
&\textit{Mandatory:} \quad yes, though it can be given on STDIN by giving a `-' on the command line.\\
&\textit{Accepted values:}\quad integer above 1.\\
\texttt{-o} or \texttt{--output} <\texttt{FILE}> &
The file to which the results must be written. Based on the file extension, Ebi will output either a deterministic finite automaton (.dfa -- Section~\ref{filehandler:deterministic finite automaton}) or a finite language (.lang -- Section~\ref{filehandler:finite language}).
If the parameter is not given, the results will be written to STDOUT as a finite language (.lang).\\
&\textit{Mandatory:} \quad no\\
\texttt{-a} or \texttt{--approximate} & Use approximate arithmetic instead of exact arithmetic.\\
&\textit{Mandatory:}\quad no\\
\bottomrule
\end{tabularx}
\noindent Output: finite language, which can be written as a deterministic finite automaton (.dfa -- Section~\ref{filehandler:deterministic finite automaton}) or a finite language (.lang -- Section~\ref{filehandler:finite language}).
\\This command is not available in Java and ProM.
\subsection{\texttt{Ebi analyse-non-stochastic executions}}
\label{command:Ebi analyse-non-stochastic executions}
Alias: \texttt{Ebi anans exe}.\\
Compute executions.
NB 1: the model must be able to terminate and its states must be bounded.
NB 2: the search performed is not optimised. For Petri nets, the ProM implementation may be more efficient.\\
\begin{tabularx}{\linewidth}{lX}
\toprule
Parameter \\\midrule
<\texttt{LOG}>&The event log.\\
&\textit{Mandatory:} \quad yes, though it can be given on STDIN by giving a `-' on the command line.\\
&\textit{Accepted values:}\quad \hyperref[filehandler:compressed event log]{compressed event log (.xes.gz)} and \hyperref[filehandler:event log]{event log (.xes)}.\\
<\texttt{MODEL}>&The model.\\
&\textit{Mandatory:} \quad yes, though it can be given on STDIN by giving a `-' on the command line.\\
&\textit{Accepted values:}\quad \hyperref[filehandler:compressed event log]{compressed event log (.xes.gz)}, \hyperref[filehandler:stochastic directly follows model]{stochastic directly follows model (.sdfm)}, \hyperref[filehandler:event log]{event log (.xes)}, \hyperref[filehandler:deterministic finite automaton]{deterministic finite automaton (.dfa)}, \hyperref[filehandler:process tree markup language]{process tree markup language (.ptml)}, \hyperref[filehandler:stochastic process tree]{stochastic process tree (.sptree)}, \hyperref[filehandler:finite stochastic language]{finite stochastic language (.slang)}, \hyperref[filehandler:Petri net markup language]{Petri net markup language (.pnml)}, \hyperref[filehandler:finite language]{finite language (.lang)}, \hyperref[filehandler:stochastic labelled Petri net]{stochastic labelled Petri net (.slpn)}, \hyperref[filehandler:stochastic deterministic finite automaton]{stochastic deterministic finite automaton (.sdfa)}, \hyperref[filehandler:directly follows model]{directly follows model (.dfm)}, \hyperref[filehandler:labelled Petri net]{labelled Petri net (.lpn)}, \hyperref[filehandler:directly follows graph]{directly follows graph (.dfg)}, \hyperref[filehandler:LoLa Petri net]{LoLa Petri net (.lola)} and \hyperref[filehandler:process tree]{process tree (.ptree)}.\\
\texttt{-o} or \texttt{--output} <\texttt{FILE}> &
The executions (.exs) file to which the result must be written. If the parameter is not given, the results will be written to STDOUT.\\
&\textit{Mandatory:} \quad no\\
\texttt{-a} or \texttt{--approximate} & Use approximate arithmetic instead of exact arithmetic.\\
&\textit{Mandatory:}\quad no\\
\bottomrule
\end{tabularx}
\noindent Output: executions, which can be written as  executions (.exs -- Section~\ref{filehandler:executions}).
\\This command is not available in Java and ProM.
\subsection{\texttt{Ebi analyse-non-stochastic infinitely-many-traces}}
\label{command:Ebi analyse-non-stochastic infinitely-many-traces}
Alias: \texttt{Ebi anans inft}.\\
Compute whether the model has infinitely many traces. The computation may not terminate if the model is unbounded.\\
\begin{tabularx}{\linewidth}{lX}
\toprule
Parameter \\\midrule
<\texttt{MODEL}>&The model.\\
&\textit{Mandatory:} \quad yes, though it can be given on STDIN by giving a `-' on the command line.\\
&\textit{Accepted values:}\quad \hyperref[filehandler:Petri net markup language]{Petri net markup language (.pnml)}, \hyperref[filehandler:event log]{event log (.xes)}, \hyperref[filehandler:labelled Petri net]{labelled Petri net (.lpn)}, \hyperref[filehandler:finite language]{finite language (.lang)}, \hyperref[filehandler:stochastic labelled Petri net]{stochastic labelled Petri net (.slpn)}, \hyperref[filehandler:directly follows graph]{directly follows graph (.dfg)}, \hyperref[filehandler:process tree markup language]{process tree markup language (.ptml)}, \hyperref[filehandler:deterministic finite automaton]{deterministic finite automaton (.dfa)}, \hyperref[filehandler:stochastic process tree]{stochastic process tree (.sptree)}, \hyperref[filehandler:finite stochastic language]{finite stochastic language (.slang)}, \hyperref[filehandler:stochastic directly follows model]{stochastic directly follows model (.sdfm)}, \hyperref[filehandler:directly follows model]{directly follows model (.dfm)}, \hyperref[filehandler:stochastic deterministic finite automaton]{stochastic deterministic finite automaton (.sdfa)}, \hyperref[filehandler:compressed event log]{compressed event log (.xes.gz)} and \hyperref[filehandler:process tree]{process tree (.ptree)}.\\
\texttt{-o} or \texttt{--output} <\texttt{FILE}> &
The boolean file to which the result must be written. If the parameter is not given, the results will be written to STDOUT.\\
&\textit{Mandatory:} \quad no\\
\texttt{-a} or \texttt{--approximate} & Use approximate arithmetic instead of exact arithmetic.\\
&\textit{Mandatory:}\quad no\\
\bottomrule
\end{tabularx}
\noindent Output: bool, which can be written as a boolean.
\\This command is available in Java and ProM.
\subsection{\texttt{Ebi analyse-non-stochastic medoid}}
\label{command:Ebi analyse-non-stochastic medoid}
Alias: \texttt{Ebi anans med}.\\
Find the traces with the lowest average normalised Levenshtein distance to the other traces; ties are resolved arbritrarily. The computation is random and does not take into account how often each trace occurs.\\
\begin{tabularx}{\linewidth}{lX}
\toprule
Parameter \\\midrule
<\texttt{FILE}>&The finite stochastic language.\\
&\textit{Mandatory:} \quad yes, though it can be given on STDIN by giving a `-' on the command line.\\
&\textit{Accepted values:}\quad \hyperref[filehandler:finite language]{finite language (.lang)}, \hyperref[filehandler:compressed event log]{compressed event log (.xes.gz)}, \hyperref[filehandler:event log]{event log (.xes)} and \hyperref[filehandler:finite stochastic language]{finite stochastic language (.slang)}.\\
<\texttt{NUMBER\_OF\_TRACES}>&The number of traces that should be extracted.\\
&\textit{Mandatory:} \quad no: if no value is provided, a default of 1 will be used. It can also be provided on STDIN by giving a `-' on the command line.\\
&\textit{Accepted values:}\quad integer.\\
\texttt{-o} or \texttt{--output} <\texttt{FILE}> &
The file to which the results must be written. Based on the file extension, Ebi will output either a deterministic finite automaton (.dfa -- Section~\ref{filehandler:deterministic finite automaton}) or a finite language (.lang -- Section~\ref{filehandler:finite language}).
If the parameter is not given, the results will be written to STDOUT as a finite language (.lang).\\
&\textit{Mandatory:} \quad no\\
\texttt{-a} or \texttt{--approximate} & Use approximate arithmetic instead of exact arithmetic.\\
&\textit{Mandatory:}\quad no\\
\bottomrule
\end{tabularx}
\noindent Output: finite language, which can be written as a deterministic finite automaton (.dfa -- Section~\ref{filehandler:deterministic finite automaton}) or a finite language (.lang -- Section~\ref{filehandler:finite language}).
\\This command is not available in Java and ProM.
\subsection{\texttt{Ebi association all-trace-attributes}}
\label{command:Ebi association all-trace-attributes}
Alias: \texttt{Ebi asso atts}.\\
Compute the association between the process and trace attributes.\\
More information: \cite{DBLP:journals/tkde/LeemansMPH23}.\\
\begin{tabularx}{\linewidth}{lX}
\toprule
Parameter \\\midrule
<\texttt{FILE}>&The event log for which association is to be computed.\\
&\textit{Mandatory:} \quad yes, though it can be given on STDIN by giving a `-' on the command line.\\
&\textit{Accepted values:}\quad \hyperref[filehandler:compressed event log]{compressed event log (.xes.gz)} and \hyperref[filehandler:event log]{event log (.xes)}.\\
<\texttt{SAMPLES}>&The number of samples taken.\\
&\textit{Mandatory:} \quad no: if no value is provided, a default of 500 will be used. It can also be provided on STDIN by giving a `-' on the command line.\\
&\textit{Accepted values:}\quad integer above 1.\\
\texttt{-o} or \texttt{--output} <\texttt{FILE}> &
The text file to which the result must be written. If the parameter is not given, the results will be written to STDOUT.\\
&\textit{Mandatory:} \quad no\\
\texttt{-a} or \texttt{--approximate} & Use approximate arithmetic instead of exact arithmetic.\\
&\textit{Mandatory:}\quad no\\
\bottomrule
\end{tabularx}
\noindent Output: text, which can be written as  text.
\\This command is available in Java and ProM.
\subsection{\texttt{Ebi association trace-attribute}}
\label{command:Ebi association trace-attribute}
Alias: \texttt{Ebi asso att}.\\
Compute the association between the process and a trace attribute.\\
More information: \cite{DBLP:journals/tkde/LeemansMPH23}.\\
\begin{tabularx}{\linewidth}{lX}
\toprule
Parameter \\\midrule
<\texttt{FILE}>&The event log for which association is to be computed.\\
&\textit{Mandatory:} \quad yes, though it can be given on STDIN by giving a `-' on the command line.\\
&\textit{Accepted values:}\quad \hyperref[filehandler:compressed event log]{compressed event log (.xes.gz)} and \hyperref[filehandler:event log]{event log (.xes)}.\\
<\texttt{ATTRIBUTE}>&The trace attribute for which association is to be computed. The trace attributes of a log can be found using `Ebi info`.\\
&\textit{Mandatory:} \quad yes, though it can be given on STDIN by giving a `-' on the command line.\\
&\textit{Accepted values:}\quad text.\\
<\texttt{SAMPLES}>&The number of samples.\\
&\textit{Mandatory:} \quad no: if no value is provided, a default of 500 will be used. It can also be provided on STDIN by giving a `-' on the command line.\\
&\textit{Accepted values:}\quad integer above 1.\\
\texttt{-o} or \texttt{--output} <\texttt{FILE}> &
The root file to which the result must be written. If the parameter is not given, the results will be written to STDOUT.\\
&\textit{Mandatory:} \quad no\\
\texttt{-a} or \texttt{--approximate} & Use approximate arithmetic instead of exact arithmetic.\\
&\textit{Mandatory:}\quad no\\
\bottomrule
\end{tabularx}
\noindent Output: root, which can be written as a root.
\\This command is available in Java and ProM.
\subsection{\texttt{Ebi conformance earth-movers-stochastic-conformance}}
\label{command:Ebi conformance earth-movers-stochastic-conformance}
Alias: \texttt{Ebi conf emsc}.\\
Compute Earth mover's stochastic conformance, also known as the Wasserstein distance.\\
More information: \cite{DBLP:journals/is/LeemansABP21}.\\
\begin{tabularx}{\linewidth}{lX}
\toprule
Parameter \\\midrule
<\texttt{FILE\_1}>&A finite stochastic language to compare.\\
&\textit{Mandatory:} \quad yes, though it can be given on STDIN by giving a `-' on the command line.\\
&\textit{Accepted values:}\quad \hyperref[filehandler:event log]{event log (.xes)}, \hyperref[filehandler:compressed event log]{compressed event log (.xes.gz)} and \hyperref[filehandler:finite stochastic language]{finite stochastic language (.slang)}.\\
<\texttt{FILE\_2}>&A finite stochastic language to compare.\\
&\textit{Mandatory:} \quad yes, though it can be given on STDIN by giving a `-' on the command line.\\
&\textit{Accepted values:}\quad \hyperref[filehandler:compressed event log]{compressed event log (.xes.gz)}, \hyperref[filehandler:event log]{event log (.xes)} and \hyperref[filehandler:finite stochastic language]{finite stochastic language (.slang)}.\\
\texttt{-o} or \texttt{--output} <\texttt{FILE}> &
The fraction file to which the result must be written. If the parameter is not given, the results will be written to STDOUT.\\
&\textit{Mandatory:} \quad no\\
\texttt{-a} or \texttt{--approximate} & Use approximate arithmetic instead of exact arithmetic.\\
&\textit{Mandatory:}\quad no\\
\bottomrule
\end{tabularx}
\noindent Output: fraction, which can be written as a fraction.
\\This command is available in Java and ProM.
\subsection{\texttt{Ebi conformance earth-movers-stochastic-conformance-sample}}
\label{command:Ebi conformance earth-movers-stochastic-conformance-sample}
Alias: \texttt{Ebi conf emsc-sample}.\\
Compute Earth mover's stochastic conformance with sampling, also known as the Wasserstein distance.\\
More information: \cite{DBLP:journals/is/LeemansABP21}.\\
\begin{tabularx}{\linewidth}{lX}
\toprule
Parameter \\\midrule
<\texttt{FILE\_1}>&A finite stochastic language to compare.\\
&\textit{Mandatory:} \quad yes, though it can be given on STDIN by giving a `-' on the command line.\\
&\textit{Accepted values:}\quad \hyperref[filehandler:finite stochastic language]{finite stochastic language (.slang)}, \hyperref[filehandler:event log]{event log (.xes)} and \hyperref[filehandler:compressed event log]{compressed event log (.xes.gz)}.\\
<\texttt{FILE\_2}>&A queriable stochastic language to compare.\\
&\textit{Mandatory:} \quad yes, though it can be given on STDIN by giving a `-' on the command line.\\
&\textit{Accepted values:}\quad \hyperref[filehandler:finite stochastic language]{finite stochastic language (.slang)}, \hyperref[filehandler:event log]{event log (.xes)}, \hyperref[filehandler:stochastic directly follows model]{stochastic directly follows model (.sdfm)}, \hyperref[filehandler:stochastic process tree]{stochastic process tree (.sptree)}, \hyperref[filehandler:stochastic deterministic finite automaton]{stochastic deterministic finite automaton (.sdfa)}, \hyperref[filehandler:compressed event log]{compressed event log (.xes.gz)}, \hyperref[filehandler:directly follows graph]{directly follows graph (.dfg)} and \hyperref[filehandler:stochastic labelled Petri net]{stochastic labelled Petri net (.slpn)}.\\
<\texttt{NUMBER\_OF\_TRACES}>&Number of traces to sample.\\
&\textit{Mandatory:} \quad yes, though it can be given on STDIN by giving a `-' on the command line.\\
&\textit{Accepted values:}\quad integer above 1.\\
\texttt{-o} or \texttt{--output} <\texttt{FILE}> &
The fraction file to which the result must be written. If the parameter is not given, the results will be written to STDOUT.\\
&\textit{Mandatory:} \quad no\\
\texttt{-a} or \texttt{--approximate} & Use approximate arithmetic instead of exact arithmetic.\\
&\textit{Mandatory:}\quad no\\
\bottomrule
\end{tabularx}
\noindent Output: fraction, which can be written as a fraction.
\\This command is available in Java and ProM.
\subsection{\texttt{Ebi conformance entropic-relevance}}
\label{command:Ebi conformance entropic-relevance}
Alias: \texttt{Ebi conf er}.\\
Compute entropic relevance (uniform).\\
More information: Entropic relevance is computed as follows:
        
        \begin{definition}[Entropic Relevance~\cite{DBLP:journals/is/AlkhammashPMG22}]
            \label{def:ER}
                Let $L$ be a finite stochastic language and let $M$ be a queriable stochastic langauge.
                Let $\Lambda$ be the set of all activities appearing in the traces of $L$.
                Then, the \emph{entropic relevance ($\entrel$) of $M$ to $L$} is defined as follows: 
                \begin{align*}
                    \entrel(L, M) ={}& H_0\left(\sum_{\sigma \in \bar{L},\, M(\sigma)>0}{L(\sigma)}\right) + 
                    \sum_{\sigma \in \bar{L}}L(\sigma) J(\sigma, M)\\
                    J(\sigma, M) ={}& \begin{cases}
                    -\log_2 M(\sigma) & M(\sigma) > 0\\
                    (1+|\sigma|) \log_2 (1 + |\Lambda|)) & \text{otherwise}
                    \end{cases}\\
                    H_0(x) ={}& -x \log_2{x} - (1-x) \log_2{(1-x)} \text{ with } H_0(0) = H_0(1) = 0 &\\
                \end{align*}       
            \end{definition}.\\
\begin{tabularx}{\linewidth}{lX}
\toprule
Parameter \\\midrule
<\texttt{FILE\_1}>&A finite stochastic language (log) to compare.\\
&\textit{Mandatory:} \quad yes, though it can be given on STDIN by giving a `-' on the command line.\\
&\textit{Accepted values:}\quad \hyperref[filehandler:event log]{event log (.xes)}, \hyperref[filehandler:finite stochastic language]{finite stochastic language (.slang)} and \hyperref[filehandler:compressed event log]{compressed event log (.xes.gz)}.\\
<\texttt{FILE\_2}>&A queriable stochastic language (model) to compare.\\
&\textit{Mandatory:} \quad yes, though it can be given on STDIN by giving a `-' on the command line.\\
&\textit{Accepted values:}\quad \hyperref[filehandler:stochastic process tree]{stochastic process tree (.sptree)}, \hyperref[filehandler:finite stochastic language]{finite stochastic language (.slang)}, \hyperref[filehandler:event log]{event log (.xes)}, \hyperref[filehandler:stochastic deterministic finite automaton]{stochastic deterministic finite automaton (.sdfa)}, \hyperref[filehandler:compressed event log]{compressed event log (.xes.gz)} and \hyperref[filehandler:stochastic labelled Petri net]{stochastic labelled Petri net (.slpn)}.\\
\texttt{-o} or \texttt{--output} <\texttt{FILE}> &
The logarithm file to which the result must be written. If the parameter is not given, the results will be written to STDOUT.\\
&\textit{Mandatory:} \quad no\\
\texttt{-a} or \texttt{--approximate} & Use approximate arithmetic instead of exact arithmetic.\\
&\textit{Mandatory:}\quad no\\
\bottomrule
\end{tabularx}
\noindent Output: logarithm, which can be written as a logarithm.
\\This command is available in Java and ProM.
\subsection{\texttt{Ebi conformance jensen-shannon}}
\label{command:Ebi conformance jensen-shannon}
Alias: \texttt{Ebi conf jssc}.\\
Compute Jensen-Shannon stochastic conformance.\\
\begin{tabularx}{\linewidth}{lX}
\toprule
Parameter \\\midrule
<\texttt{FILE\_1}>&A finite stochastic language to compare.\\
&\textit{Mandatory:} \quad yes, though it can be given on STDIN by giving a `-' on the command line.\\
&\textit{Accepted values:}\quad \hyperref[filehandler:compressed event log]{compressed event log (.xes.gz)}, \hyperref[filehandler:finite stochastic language]{finite stochastic language (.slang)} and \hyperref[filehandler:event log]{event log (.xes)}.\\
<\texttt{FILE\_2}>&A queriable stochastic language to compare.\\
&\textit{Mandatory:} \quad yes, though it can be given on STDIN by giving a `-' on the command line.\\
&\textit{Accepted values:}\quad \hyperref[filehandler:stochastic process tree]{stochastic process tree (.sptree)}, \hyperref[filehandler:event log]{event log (.xes)}, \hyperref[filehandler:finite stochastic language]{finite stochastic language (.slang)}, \hyperref[filehandler:compressed event log]{compressed event log (.xes.gz)}, \hyperref[filehandler:stochastic deterministic finite automaton]{stochastic deterministic finite automaton (.sdfa)} and \hyperref[filehandler:stochastic labelled Petri net]{stochastic labelled Petri net (.slpn)}.\\
\texttt{-o} or \texttt{--output} <\texttt{FILE}> &
The rootlog file to which the result must be written. If the parameter is not given, the results will be written to STDOUT.\\
&\textit{Mandatory:} \quad no\\
\bottomrule
\end{tabularx}
\noindent Output: rootlog, which can be written as a rootlog.
\\This command is available in Java and ProM.
\subsection{\texttt{Ebi conformance jensen-shannon-sample}}
\label{command:Ebi conformance jensen-shannon-sample}
Alias: \texttt{Ebi conf jssc-sample}.\\
Compute Jensen-Shannon stochastic conformance with sampling.\\
\begin{tabularx}{\linewidth}{lX}
\toprule
Parameter \\\midrule
<\texttt{FILE\_1}>&A queriable stochastic language to compare.\\
&\textit{Mandatory:} \quad yes, though it can be given on STDIN by giving a `-' on the command line.\\
&\textit{Accepted values:}\quad \hyperref[filehandler:finite stochastic language]{finite stochastic language (.slang)}, \hyperref[filehandler:stochastic directly follows model]{stochastic directly follows model (.sdfm)}, \hyperref[filehandler:directly follows graph]{directly follows graph (.dfg)}, \hyperref[filehandler:event log]{event log (.xes)}, \hyperref[filehandler:stochastic labelled Petri net]{stochastic labelled Petri net (.slpn)}, \hyperref[filehandler:stochastic deterministic finite automaton]{stochastic deterministic finite automaton (.sdfa)}, \hyperref[filehandler:compressed event log]{compressed event log (.xes.gz)} and \hyperref[filehandler:stochastic process tree]{stochastic process tree (.sptree)}.\\
<\texttt{FILE\_2}>&A queriable stochastic language to compare.\\
&\textit{Mandatory:} \quad yes, though it can be given on STDIN by giving a `-' on the command line.\\
&\textit{Accepted values:}\quad \hyperref[filehandler:event log]{event log (.xes)}, \hyperref[filehandler:compressed event log]{compressed event log (.xes.gz)}, \hyperref[filehandler:stochastic deterministic finite automaton]{stochastic deterministic finite automaton (.sdfa)}, \hyperref[filehandler:finite stochastic language]{finite stochastic language (.slang)}, \hyperref[filehandler:stochastic directly follows model]{stochastic directly follows model (.sdfm)}, \hyperref[filehandler:stochastic process tree]{stochastic process tree (.sptree)}, \hyperref[filehandler:directly follows graph]{directly follows graph (.dfg)} and \hyperref[filehandler:stochastic labelled Petri net]{stochastic labelled Petri net (.slpn)}.\\
<\texttt{NUMBER\_OF\_TRACES}>&Number of traces to sample.\\
&\textit{Mandatory:} \quad yes, though it can be given on STDIN by giving a `-' on the command line.\\
&\textit{Accepted values:}\quad integer above 1.\\
\texttt{-o} or \texttt{--output} <\texttt{FILE}> &
The rootlog file to which the result must be written. If the parameter is not given, the results will be written to STDOUT.\\
&\textit{Mandatory:} \quad no\\
\bottomrule
\end{tabularx}
\noindent Output: rootlog, which can be written as a rootlog.
\\This command is available in Java and ProM.
\subsection{\texttt{Ebi conformance unit-earth-movers-stochastic-conformance}}
\label{command:Ebi conformance unit-earth-movers-stochastic-conformance}
Alias: \texttt{Ebi conf uemsc}.\\
Compute unit-earth movers' stochastic conformance, also known as total variation distance.\\
More information: \cite{DBLP:conf/bpm/LeemansSA19}.\\
\begin{tabularx}{\linewidth}{lX}
\toprule
Parameter \\\midrule
<\texttt{FILE\_1}>&A finite stochastic language (log) to compare.\\
&\textit{Mandatory:} \quad yes, though it can be given on STDIN by giving a `-' on the command line.\\
&\textit{Accepted values:}\quad \hyperref[filehandler:event log]{event log (.xes)}, \hyperref[filehandler:finite stochastic language]{finite stochastic language (.slang)} and \hyperref[filehandler:compressed event log]{compressed event log (.xes.gz)}.\\
<\texttt{FILE\_2}>&A queriable stochastic language (model) to compare.\\
&\textit{Mandatory:} \quad yes, though it can be given on STDIN by giving a `-' on the command line.\\
&\textit{Accepted values:}\quad \hyperref[filehandler:finite stochastic language]{finite stochastic language (.slang)}, \hyperref[filehandler:event log]{event log (.xes)}, \hyperref[filehandler:stochastic deterministic finite automaton]{stochastic deterministic finite automaton (.sdfa)}, \hyperref[filehandler:stochastic process tree]{stochastic process tree (.sptree)}, \hyperref[filehandler:stochastic labelled Petri net]{stochastic labelled Petri net (.slpn)} and \hyperref[filehandler:compressed event log]{compressed event log (.xes.gz)}.\\
\texttt{-o} or \texttt{--output} <\texttt{FILE}> &
The fraction file to which the result must be written. If the parameter is not given, the results will be written to STDOUT.\\
&\textit{Mandatory:} \quad no\\
\texttt{-a} or \texttt{--approximate} & Use approximate arithmetic instead of exact arithmetic.\\
&\textit{Mandatory:}\quad no\\
\bottomrule
\end{tabularx}
\noindent Output: fraction, which can be written as a fraction.
\\This command is available in Java and ProM.
\subsection{\texttt{Ebi conformance-non-stochastic alignments}}
\label{command:Ebi conformance-non-stochastic alignments}
Alias: \texttt{Ebi confns ali}.\\
Compute alignments.
The model must be able to terminate and its states must be bounded. The search performed is not optimised. For Petri nets, the ProM implementation may be more efficient.\\
More information: Compute alignments according to the method described by Adriansyah~\cite{DBLP:conf/edoc/AdriansyahDA11}. By default, all traces are computed concurrently on all CPU cores. If this requires too much RAM, please see speed trick~\ref{speedtrick:multithreaded} in Section~\ref{sec:speedtricks} for how to reduce the number of CPU cores utilised..\\
\begin{tabularx}{\linewidth}{lX}
\toprule
Parameter \\\midrule
<\texttt{FILE\_1}>&The finite stochastic language.\\
&\textit{Mandatory:} \quad yes, though it can be given on STDIN by giving a `-' on the command line.\\
&\textit{Accepted values:}\quad \hyperref[filehandler:compressed event log]{compressed event log (.xes.gz)}, \hyperref[filehandler:finite stochastic language]{finite stochastic language (.slang)} and \hyperref[filehandler:event log]{event log (.xes)}.\\
<\texttt{FILE\_2}>&The model.\\
&\textit{Mandatory:} \quad yes, though it can be given on STDIN by giving a `-' on the command line.\\
&\textit{Accepted values:}\quad \hyperref[filehandler:process tree]{process tree (.ptree)}, \hyperref[filehandler:LoLa Petri net]{LoLa Petri net (.lola)}, \hyperref[filehandler:finite language]{finite language (.lang)}, \hyperref[filehandler:finite stochastic language]{finite stochastic language (.slang)}, \hyperref[filehandler:deterministic finite automaton]{deterministic finite automaton (.dfa)}, \hyperref[filehandler:stochastic directly follows model]{stochastic directly follows model (.sdfm)}, \hyperref[filehandler:stochastic deterministic finite automaton]{stochastic deterministic finite automaton (.sdfa)}, \hyperref[filehandler:stochastic labelled Petri net]{stochastic labelled Petri net (.slpn)}, \hyperref[filehandler:directly follows model]{directly follows model (.dfm)}, \hyperref[filehandler:process tree markup language]{process tree markup language (.ptml)}, \hyperref[filehandler:stochastic process tree]{stochastic process tree (.sptree)}, \hyperref[filehandler:compressed event log]{compressed event log (.xes.gz)}, \hyperref[filehandler:directly follows graph]{directly follows graph (.dfg)}, \hyperref[filehandler:Petri net markup language]{Petri net markup language (.pnml)}, \hyperref[filehandler:event log]{event log (.xes)} and \hyperref[filehandler:labelled Petri net]{labelled Petri net (.lpn)}.\\
\texttt{-o} or \texttt{--output} <\texttt{FILE}> &
The file to which the results must be written. Based on the file extension, Ebi will output either a language of alignments (.ali -- Section~\ref{filehandler:language of alignments}) or a stochastic language of alignments (.sali -- Section~\ref{filehandler:stochastic language of alignments}).
If the parameter is not given, the results will be written to STDOUT as a stochastic language of alignments (.sali).\\
&\textit{Mandatory:} \quad no\\
\texttt{-a} or \texttt{--approximate} & Use approximate arithmetic instead of exact arithmetic.\\
&\textit{Mandatory:}\quad no\\
\bottomrule
\end{tabularx}
\noindent Output: stochastic language of alignments, which can be written as a language of alignments (.ali -- Section~\ref{filehandler:language of alignments}) or a stochastic language of alignments (.sali -- Section~\ref{filehandler:stochastic language of alignments}).
\\This command is not available in Java and ProM.
\subsection{\texttt{Ebi conformance-non-stochastic escaping-edges-precision}}
\label{command:Ebi conformance-non-stochastic escaping-edges-precision}
Alias: \texttt{Ebi confns eep}.\\
Computes a prefix automaton of the alignments, where the states represent model states that have been visited by the alignments. The states are weighed by the probability of traces visiting them. The precision is then the sum of the number of taken edges out of each state multiplied by the weight of the state, divided by the sum of the number of outgoing edges multiplied by the weight of the states.\\
More information: Compute escaping-edges precision using the method of Adriansyah and Munoz-Gama~\cite{DBLP:conf/bpm/AdriansyahMCDA12}.\\
\begin{tabularx}{\linewidth}{lX}
\toprule
Parameter \\\midrule
<\texttt{ALIGNMENTS}>&The alignments.\\
&\textit{Mandatory:} \quad yes, though it can be given on STDIN by giving a `-' on the command line.\\
&\textit{Accepted values:}\quad \hyperref[filehandler:stochastic language of alignments]{stochastic language of alignments (.sali)}.\\
<\texttt{MODEL}>&The model from which the alignments were computed.\\
&\textit{Mandatory:} \quad yes, though it can be given on STDIN by giving a `-' on the command line.\\
&\textit{Accepted values:}\quad \hyperref[filehandler:finite language]{finite language (.lang)}, \hyperref[filehandler:deterministic finite automaton]{deterministic finite automaton (.dfa)}, \hyperref[filehandler:directly follows model]{directly follows model (.dfm)}, \hyperref[filehandler:event log]{event log (.xes)}, \hyperref[filehandler:finite stochastic language]{finite stochastic language (.slang)}, \hyperref[filehandler:compressed event log]{compressed event log (.xes.gz)}, \hyperref[filehandler:process tree]{process tree (.ptree)}, \hyperref[filehandler:LoLa Petri net]{LoLa Petri net (.lola)}, \hyperref[filehandler:stochastic directly follows model]{stochastic directly follows model (.sdfm)}, \hyperref[filehandler:stochastic process tree]{stochastic process tree (.sptree)}, \hyperref[filehandler:process tree markup language]{process tree markup language (.ptml)}, \hyperref[filehandler:labelled Petri net]{labelled Petri net (.lpn)}, \hyperref[filehandler:stochastic labelled Petri net]{stochastic labelled Petri net (.slpn)}, \hyperref[filehandler:directly follows graph]{directly follows graph (.dfg)}, \hyperref[filehandler:stochastic deterministic finite automaton]{stochastic deterministic finite automaton (.sdfa)} and \hyperref[filehandler:Petri net markup language]{Petri net markup language (.pnml)}.\\
\texttt{-o} or \texttt{--output} <\texttt{FILE}> &
The fraction file to which the result must be written. If the parameter is not given, the results will be written to STDOUT.\\
&\textit{Mandatory:} \quad no\\
\texttt{-a} or \texttt{--approximate} & Use approximate arithmetic instead of exact arithmetic.\\
&\textit{Mandatory:}\quad no\\
\bottomrule
\end{tabularx}
\noindent Output: fraction, which can be written as a fraction.
\\This command is not available in Java and ProM.
\subsection{\texttt{Ebi conformance-non-stochastic set-alignments}}
\label{command:Ebi conformance-non-stochastic set-alignments}
Alias: \texttt{Ebi confns setali}.\\
Compute a non-weighted set of alignments.
The model must be able to terminate and its states must be bounded. The search performed is not optimised. For Petri nets, the ProM implementation may be more efficient.\\
More information: Alignments according to the method described by Adriansyah~\cite{DBLP:conf/edoc/AdriansyahDA11}. By default, all traces are computed concurrently on all CPU cores. If this requires too much RAM, please see speed trick~\ref{speedtrick:multithreaded} in Section~\ref{sec:speedtricks} for how to reduce the number of CPU cores utilised..\\
\begin{tabularx}{\linewidth}{lX}
\toprule
Parameter \\\midrule
<\texttt{FILE\_1}>&The finite language.\\
&\textit{Mandatory:} \quad yes, though it can be given on STDIN by giving a `-' on the command line.\\
&\textit{Accepted values:}\quad \hyperref[filehandler:finite stochastic language]{finite stochastic language (.slang)}, \hyperref[filehandler:compressed event log]{compressed event log (.xes.gz)}, \hyperref[filehandler:finite language]{finite language (.lang)} and \hyperref[filehandler:event log]{event log (.xes)}.\\
<\texttt{FILE\_2}>&The model.\\
&\textit{Mandatory:} \quad yes, though it can be given on STDIN by giving a `-' on the command line.\\
&\textit{Accepted values:}\quad \hyperref[filehandler:finite language]{finite language (.lang)}, \hyperref[filehandler:stochastic directly follows model]{stochastic directly follows model (.sdfm)}, \hyperref[filehandler:event log]{event log (.xes)}, \hyperref[filehandler:deterministic finite automaton]{deterministic finite automaton (.dfa)}, \hyperref[filehandler:process tree markup language]{process tree markup language (.ptml)}, \hyperref[filehandler:stochastic labelled Petri net]{stochastic labelled Petri net (.slpn)}, \hyperref[filehandler:directly follows graph]{directly follows graph (.dfg)}, \hyperref[filehandler:process tree]{process tree (.ptree)}, \hyperref[filehandler:LoLa Petri net]{LoLa Petri net (.lola)}, \hyperref[filehandler:compressed event log]{compressed event log (.xes.gz)}, \hyperref[filehandler:stochastic deterministic finite automaton]{stochastic deterministic finite automaton (.sdfa)}, \hyperref[filehandler:Petri net markup language]{Petri net markup language (.pnml)}, \hyperref[filehandler:finite stochastic language]{finite stochastic language (.slang)}, \hyperref[filehandler:directly follows model]{directly follows model (.dfm)}, \hyperref[filehandler:labelled Petri net]{labelled Petri net (.lpn)} and \hyperref[filehandler:stochastic process tree]{stochastic process tree (.sptree)}.\\
\texttt{-o} or \texttt{--output} <\texttt{FILE}> &
The language of alignments (.ali) file to which the result must be written. If the parameter is not given, the results will be written to STDOUT.\\
&\textit{Mandatory:} \quad no\\
\texttt{-a} or \texttt{--approximate} & Use approximate arithmetic instead of exact arithmetic.\\
&\textit{Mandatory:}\quad no\\
\bottomrule
\end{tabularx}
\noindent Output: alignments, which can be written as a language of alignments (.ali -- Section~\ref{filehandler:language of alignments}).
\\This command is not available in Java and ProM.
\subsection{\texttt{Ebi conformance-non-stochastic trace-fitness}}
\label{command:Ebi conformance-non-stochastic trace-fitness}
Alias: \texttt{Ebi confns tfit}.\\
Compute the trace-fitness of a stochastic language of alignments: the number of synchronous moves divided by the total number of moves, both without silent moves.\\
\begin{tabularx}{\linewidth}{lX}
\toprule
Parameter \\\midrule
<\texttt{ALIGNMENTS}>&The stochastic language of alignments.\\
&\textit{Mandatory:} \quad yes, though it can be given on STDIN by giving a `-' on the command line.\\
&\textit{Accepted values:}\quad \hyperref[filehandler:stochastic language of alignments]{stochastic language of alignments (.sali)}.\\
\texttt{-o} or \texttt{--output} <\texttt{FILE}> &
The fraction file to which the result must be written. If the parameter is not given, the results will be written to STDOUT.\\
&\textit{Mandatory:} \quad no\\
\texttt{-a} or \texttt{--approximate} & Use approximate arithmetic instead of exact arithmetic.\\
&\textit{Mandatory:}\quad no\\
\bottomrule
\end{tabularx}
\noindent Output: fraction, which can be written as a fraction.
\\This command is not available in Java and ProM.
\subsection{\texttt{Ebi convert finite-stochastic-language}}
\label{command:Ebi convert finite-stochastic-language}
Alias: \texttt{Ebi conv slang}.\\
Convert an object to a finite stochastic language.\\
\begin{tabularx}{\linewidth}{lX}
\toprule
Parameter \\\midrule
<\texttt{FILE}>&Any file supported by Ebi that can be converted.\\
&\textit{Mandatory:} \quad yes, though it can be given on STDIN by giving a `-' on the command line.\\
&\textit{Accepted values:}\quad \hyperref[filehandler:finite stochastic language]{finite stochastic language (.slang)} and \hyperref[filehandler:event log]{event log (.xes)}.\\
\texttt{-o} or \texttt{--output} <\texttt{FILE}> &
The file to which the results must be written. Based on the file extension, Ebi will output either a deterministic finite automaton (.dfa -- Section~\ref{filehandler:deterministic finite automaton}), a finite language (.lang -- Section~\ref{filehandler:finite language}), a finite stochastic language (.slang -- Section~\ref{filehandler:finite stochastic language}) or a stochastic deterministic finite automaton (.sdfa -- Section~\ref{filehandler:stochastic deterministic finite automaton}).
If the parameter is not given, the results will be written to STDOUT as a finite stochastic language (.slang).\\
&\textit{Mandatory:} \quad no\\
\texttt{-a} or \texttt{--approximate} & Use approximate arithmetic instead of exact arithmetic.\\
&\textit{Mandatory:}\quad no\\
\bottomrule
\end{tabularx}
\noindent Output: finite stochastic language, which can be written as a deterministic finite automaton (.dfa -- Section~\ref{filehandler:deterministic finite automaton}), a finite language (.lang -- Section~\ref{filehandler:finite language}), a finite stochastic language (.slang -- Section~\ref{filehandler:finite stochastic language}) or a stochastic deterministic finite automaton (.sdfa -- Section~\ref{filehandler:stochastic deterministic finite automaton}).
\\This command is not available in Java and ProM.
\subsection{\texttt{Ebi convert labelled-petri-net}}
\label{command:Ebi convert labelled-petri-net}
Alias: \texttt{Ebi conv lpn}.\\
Convert an object to a labelled Petri net.\\
\begin{tabularx}{\linewidth}{lX}
\toprule
Parameter \\\midrule
<\texttt{FILE}>&Any file supported by Ebi that can be converted.\\
&\textit{Mandatory:} \quad yes, though it can be given on STDIN by giving a `-' on the command line.\\
&\textit{Accepted values:}\quad \hyperref[filehandler:Petri net markup language]{Petri net markup language (.pnml)}, \hyperref[filehandler:stochastic deterministic finite automaton]{stochastic deterministic finite automaton (.sdfa)}, \hyperref[filehandler:directly follows model]{directly follows model (.dfm)}, \hyperref[filehandler:process tree]{process tree (.ptree)}, \hyperref[filehandler:labelled Petri net]{labelled Petri net (.lpn)}, \hyperref[filehandler:process tree markup language]{process tree markup language (.ptml)}, \hyperref[filehandler:stochastic directly follows model]{stochastic directly follows model (.sdfm)}, \hyperref[filehandler:deterministic finite automaton]{deterministic finite automaton (.dfa)}, \hyperref[filehandler:stochastic labelled Petri net]{stochastic labelled Petri net (.slpn)}, \hyperref[filehandler:stochastic process tree]{stochastic process tree (.sptree)} and \hyperref[filehandler:directly follows graph]{directly follows graph (.dfg)}.\\
\texttt{-o} or \texttt{--output} <\texttt{FILE}> &
The file to which the results must be written. Based on the file extension, Ebi will output either a labelled Petri net (.lpn -- Section~\ref{filehandler:labelled Petri net}), a LoLa Petri net (.lola -- Section~\ref{filehandler:LoLa Petri net}), a Petri net markup language (.pnml -- Section~\ref{filehandler:Petri net markup language}), a portable document format (.pdf -- Section~\ref{filehandler:portable document format}) or a scalable vector graphics (.svg -- Section~\ref{filehandler:scalable vector graphics}).
If the parameter is not given, the results will be written to STDOUT as a labelled Petri net (.lpn).\\
&\textit{Mandatory:} \quad no\\
\texttt{-a} or \texttt{--approximate} & Use approximate arithmetic instead of exact arithmetic.\\
&\textit{Mandatory:}\quad no\\
\bottomrule
\end{tabularx}
\noindent Output: labelled Petri net, which can be written as a labelled Petri net (.lpn -- Section~\ref{filehandler:labelled Petri net}), a LoLa Petri net (.lola -- Section~\ref{filehandler:LoLa Petri net}), a Petri net markup language (.pnml -- Section~\ref{filehandler:Petri net markup language}), a portable document format (.pdf -- Section~\ref{filehandler:portable document format}) or a scalable vector graphics (.svg -- Section~\ref{filehandler:scalable vector graphics}).
\\This command is available in Java and ProM.
\subsection{\texttt{Ebi convert stochastic-finite-deterministic-automaton}}
\label{command:Ebi convert stochastic-finite-deterministic-automaton}
Alias: \texttt{Ebi conv sdfa}.\\
Convert an object to a stochastic deterministic finite automaton.\\
\begin{tabularx}{\linewidth}{lX}
\toprule
Parameter \\\midrule
<\texttt{FILE}>&Any file supported by Ebi that can be converted.\\
&\textit{Mandatory:} \quad yes, though it can be given on STDIN by giving a `-' on the command line.\\
&\textit{Accepted values:}\quad \hyperref[filehandler:finite stochastic language]{finite stochastic language (.slang)}, \hyperref[filehandler:stochastic deterministic finite automaton]{stochastic deterministic finite automaton (.sdfa)} and \hyperref[filehandler:event log]{event log (.xes)}.\\
\texttt{-o} or \texttt{--output} <\texttt{FILE}> &
The file to which the results must be written. Based on the file extension, Ebi will output either a deterministic finite automaton (.dfa -- Section~\ref{filehandler:deterministic finite automaton}), a labelled Petri net (.lpn -- Section~\ref{filehandler:labelled Petri net}), a LoLa Petri net (.lola -- Section~\ref{filehandler:LoLa Petri net}), a Petri net markup language (.pnml -- Section~\ref{filehandler:Petri net markup language}), a stochastic deterministic finite automaton (.sdfa -- Section~\ref{filehandler:stochastic deterministic finite automaton}), a portable document format (.pdf -- Section~\ref{filehandler:portable document format}) or a scalable vector graphics (.svg -- Section~\ref{filehandler:scalable vector graphics}).
If the parameter is not given, the results will be written to STDOUT as a stochastic deterministic finite automaton (.sdfa).\\
&\textit{Mandatory:} \quad no\\
\texttt{-a} or \texttt{--approximate} & Use approximate arithmetic instead of exact arithmetic.\\
&\textit{Mandatory:}\quad no\\
\bottomrule
\end{tabularx}
\noindent Output: stochastic deterministic finite automaton, which can be written as a deterministic finite automaton (.dfa -- Section~\ref{filehandler:deterministic finite automaton}), a labelled Petri net (.lpn -- Section~\ref{filehandler:labelled Petri net}), a LoLa Petri net (.lola -- Section~\ref{filehandler:LoLa Petri net}), a Petri net markup language (.pnml -- Section~\ref{filehandler:Petri net markup language}), a stochastic deterministic finite automaton (.sdfa -- Section~\ref{filehandler:stochastic deterministic finite automaton}), a portable document format (.pdf -- Section~\ref{filehandler:portable document format}) or a scalable vector graphics (.svg -- Section~\ref{filehandler:scalable vector graphics}).
\\This command is available in Java and ProM.
\subsection{\texttt{Ebi discover alignments}}
\label{command:Ebi discover alignments}
Alias: \texttt{Ebi disc ali}.\\
Give each transition a weight that matches the aligned occurrences of its label. The model must be livelock-free.\\
More information: ~\cite{DBLP:conf/icpm/BurkeLW20}.\\
\begin{tabularx}{\linewidth}{lX}
\toprule
Parameter \\\midrule
<\texttt{FILE\_1}>&A finite stochastic language (log) to get the occurrences from.\\
&\textit{Mandatory:} \quad yes, though it can be given on STDIN by giving a `-' on the command line.\\
&\textit{Accepted values:}\quad \hyperref[filehandler:event log]{event log (.xes)}, \hyperref[filehandler:finite stochastic language]{finite stochastic language (.slang)} and \hyperref[filehandler:compressed event log]{compressed event log (.xes.gz)}.\\
<\texttt{FILE\_2}>&A labelled Petri net with the control flow.\\
&\textit{Mandatory:} \quad yes, though it can be given on STDIN by giving a `-' on the command line.\\
&\textit{Accepted values:}\quad \hyperref[filehandler:directly follows model]{directly follows model (.dfm)}, \hyperref[filehandler:stochastic directly follows model]{stochastic directly follows model (.sdfm)}, \hyperref[filehandler:stochastic labelled Petri net]{stochastic labelled Petri net (.slpn)}, \hyperref[filehandler:labelled Petri net]{labelled Petri net (.lpn)}, \hyperref[filehandler:directly follows graph]{directly follows graph (.dfg)}, \hyperref[filehandler:Petri net markup language]{Petri net markup language (.pnml)}, \hyperref[filehandler:deterministic finite automaton]{deterministic finite automaton (.dfa)}, \hyperref[filehandler:stochastic deterministic finite automaton]{stochastic deterministic finite automaton (.sdfa)}, \hyperref[filehandler:stochastic process tree]{stochastic process tree (.sptree)}, \hyperref[filehandler:process tree markup language]{process tree markup language (.ptml)} and \hyperref[filehandler:process tree]{process tree (.ptree)}.\\
\texttt{-o} or \texttt{--output} <\texttt{FILE}> &
The file to which the results must be written. Based on the file extension, Ebi will output either a labelled Petri net (.lpn -- Section~\ref{filehandler:labelled Petri net}), a LoLa Petri net (.lola -- Section~\ref{filehandler:LoLa Petri net}), a Petri net markup language (.pnml -- Section~\ref{filehandler:Petri net markup language}), a stochastic labelled Petri net (.slpn -- Section~\ref{filehandler:stochastic labelled Petri net}), a portable document format (.pdf -- Section~\ref{filehandler:portable document format}) or a scalable vector graphics (.svg -- Section~\ref{filehandler:scalable vector graphics}).
If the parameter is not given, the results will be written to STDOUT as a stochastic labelled Petri net (.slpn).\\
&\textit{Mandatory:} \quad no\\
\texttt{-a} or \texttt{--approximate} & Use approximate arithmetic instead of exact arithmetic.\\
&\textit{Mandatory:}\quad no\\
\bottomrule
\end{tabularx}
\noindent Output: stochastic labelled Petri net, which can be written as a labelled Petri net (.lpn -- Section~\ref{filehandler:labelled Petri net}), a LoLa Petri net (.lola -- Section~\ref{filehandler:LoLa Petri net}), a Petri net markup language (.pnml -- Section~\ref{filehandler:Petri net markup language}), a stochastic labelled Petri net (.slpn -- Section~\ref{filehandler:stochastic labelled Petri net}), a portable document format (.pdf -- Section~\ref{filehandler:portable document format}) or a scalable vector graphics (.svg -- Section~\ref{filehandler:scalable vector graphics}).
\\This command is available in Java and ProM.
\subsection{\texttt{Ebi discover directly-follows-graph}}
\label{command:Ebi discover directly-follows-graph}
Alias: \texttt{Ebi disc dfg}.\\
Discover a directly follows graph.\\
More information: ~\cite{DBLP:conf/icpm/LeemansPW19}.\\
\begin{tabularx}{\linewidth}{lX}
\toprule
Parameter \\\midrule
<\texttt{LANG}>&A finite stochastic language.\\
&\textit{Mandatory:} \quad yes, though it can be given on STDIN by giving a `-' on the command line.\\
&\textit{Accepted values:}\quad \hyperref[filehandler:compressed event log]{compressed event log (.xes.gz)}, \hyperref[filehandler:event log]{event log (.xes)} and \hyperref[filehandler:finite stochastic language]{finite stochastic language (.slang)}.\\
<\texttt{MIN\_FITNESS}>&The minimum fraction of traces that should fit the resulting model.\\
&\textit{Mandatory:} \quad no: if no value is provided, a default of 1 will be used. It can also be provided on STDIN by giving a `-' on the command line.\\
&\textit{Accepted values:}\quad fraction between 0 and 1.\\
\texttt{-o} or \texttt{--output} <\texttt{FILE}> &
The file to which the results must be written. Based on the file extension, Ebi will output either a directly follows graph (.dfg -- Section~\ref{filehandler:directly follows graph}), a directly follows model (.dfm -- Section~\ref{filehandler:directly follows model}), a stochastic directly follows model (.sdfm -- Section~\ref{filehandler:stochastic directly follows model}), a labelled Petri net (.lpn -- Section~\ref{filehandler:labelled Petri net}), a LoLa Petri net (.lola -- Section~\ref{filehandler:LoLa Petri net}), a Petri net markup language (.pnml -- Section~\ref{filehandler:Petri net markup language}), a stochastic labelled Petri net (.slpn -- Section~\ref{filehandler:stochastic labelled Petri net}), a portable document format (.pdf -- Section~\ref{filehandler:portable document format}) or a scalable vector graphics (.svg -- Section~\ref{filehandler:scalable vector graphics}).
If the parameter is not given, the results will be written to STDOUT as a directly follows graph (.dfg).\\
&\textit{Mandatory:} \quad no\\
\texttt{-a} or \texttt{--approximate} & Use approximate arithmetic instead of exact arithmetic.\\
&\textit{Mandatory:}\quad no\\
\bottomrule
\end{tabularx}
\noindent Output: directly follows graph, which can be written as a directly follows graph (.dfg -- Section~\ref{filehandler:directly follows graph}), a directly follows model (.dfm -- Section~\ref{filehandler:directly follows model}), a stochastic directly follows model (.sdfm -- Section~\ref{filehandler:stochastic directly follows model}), a labelled Petri net (.lpn -- Section~\ref{filehandler:labelled Petri net}), a LoLa Petri net (.lola -- Section~\ref{filehandler:LoLa Petri net}), a Petri net markup language (.pnml -- Section~\ref{filehandler:Petri net markup language}), a stochastic labelled Petri net (.slpn -- Section~\ref{filehandler:stochastic labelled Petri net}), a portable document format (.pdf -- Section~\ref{filehandler:portable document format}) or a scalable vector graphics (.svg -- Section~\ref{filehandler:scalable vector graphics}).
\\This command is available in Java and ProM.
\subsection{\texttt{Ebi discover occurrence labelled-petri-net}}
\label{command:Ebi discover occurrence labelled-petri-net}
Alias: \texttt{Ebi disc occ lpn}.\\
Give each transition a weight that matches the occurrences of its label; silent transitions get a weight of 1.\\
More information: ~\cite{DBLP:conf/icpm/BurkeLW20}.\\
\begin{tabularx}{\linewidth}{lX}
\toprule
Parameter \\\midrule
<\texttt{FILE\_1}>&A finite stochastic language (log) to get the occurrences from.\\
&\textit{Mandatory:} \quad yes, though it can be given on STDIN by giving a `-' on the command line.\\
&\textit{Accepted values:}\quad \hyperref[filehandler:compressed event log]{compressed event log (.xes.gz)}, \hyperref[filehandler:event log]{event log (.xes)} and \hyperref[filehandler:finite stochastic language]{finite stochastic language (.slang)}.\\
<\texttt{FILE\_2}>&A labelled Petri net with the control flow.\\
&\textit{Mandatory:} \quad yes, though it can be given on STDIN by giving a `-' on the command line.\\
&\textit{Accepted values:}\quad \hyperref[filehandler:directly follows model]{directly follows model (.dfm)}, \hyperref[filehandler:process tree]{process tree (.ptree)}, \hyperref[filehandler:Petri net markup language]{Petri net markup language (.pnml)}, \hyperref[filehandler:stochastic deterministic finite automaton]{stochastic deterministic finite automaton (.sdfa)}, \hyperref[filehandler:directly follows graph]{directly follows graph (.dfg)}, \hyperref[filehandler:stochastic labelled Petri net]{stochastic labelled Petri net (.slpn)}, \hyperref[filehandler:deterministic finite automaton]{deterministic finite automaton (.dfa)}, \hyperref[filehandler:stochastic directly follows model]{stochastic directly follows model (.sdfm)}, \hyperref[filehandler:labelled Petri net]{labelled Petri net (.lpn)}, \hyperref[filehandler:stochastic process tree]{stochastic process tree (.sptree)} and \hyperref[filehandler:process tree markup language]{process tree markup language (.ptml)}.\\
\texttt{-o} or \texttt{--output} <\texttt{FILE}> &
The file to which the results must be written. Based on the file extension, Ebi will output either a labelled Petri net (.lpn -- Section~\ref{filehandler:labelled Petri net}), a LoLa Petri net (.lola -- Section~\ref{filehandler:LoLa Petri net}), a Petri net markup language (.pnml -- Section~\ref{filehandler:Petri net markup language}), a stochastic labelled Petri net (.slpn -- Section~\ref{filehandler:stochastic labelled Petri net}), a portable document format (.pdf -- Section~\ref{filehandler:portable document format}) or a scalable vector graphics (.svg -- Section~\ref{filehandler:scalable vector graphics}).
If the parameter is not given, the results will be written to STDOUT as a stochastic labelled Petri net (.slpn).\\
&\textit{Mandatory:} \quad no\\
\texttt{-a} or \texttt{--approximate} & Use approximate arithmetic instead of exact arithmetic.\\
&\textit{Mandatory:}\quad no\\
\bottomrule
\end{tabularx}
\noindent Output: stochastic labelled Petri net, which can be written as a labelled Petri net (.lpn -- Section~\ref{filehandler:labelled Petri net}), a LoLa Petri net (.lola -- Section~\ref{filehandler:LoLa Petri net}), a Petri net markup language (.pnml -- Section~\ref{filehandler:Petri net markup language}), a stochastic labelled Petri net (.slpn -- Section~\ref{filehandler:stochastic labelled Petri net}), a portable document format (.pdf -- Section~\ref{filehandler:portable document format}) or a scalable vector graphics (.svg -- Section~\ref{filehandler:scalable vector graphics}).
\\This command is available in Java and ProM.
\subsection{\texttt{Ebi discover occurrence process-tree}}
\label{command:Ebi discover occurrence process-tree}
Alias: \texttt{Ebi disc occ ptree}.\\
Give each leaf a weight that matches the occurrences of its label; silent leaves get a weight of 1.\\
More information: ~\cite{DBLP:conf/icpm/BurkeLW20}.\\
\begin{tabularx}{\linewidth}{lX}
\toprule
Parameter \\\midrule
<\texttt{LANG}>&A finite stochastic language (log) to get the occurrences from.\\
&\textit{Mandatory:} \quad yes, though it can be given on STDIN by giving a `-' on the command line.\\
&\textit{Accepted values:}\quad \hyperref[filehandler:compressed event log]{compressed event log (.xes.gz)}, \hyperref[filehandler:finite stochastic language]{finite stochastic language (.slang)} and \hyperref[filehandler:event log]{event log (.xes)}.\\
<\texttt{TREE}>&A process tree with the control flow.\\
&\textit{Mandatory:} \quad yes, though it can be given on STDIN by giving a `-' on the command line.\\
&\textit{Accepted values:}\quad \hyperref[filehandler:process tree]{process tree (.ptree)}, \hyperref[filehandler:stochastic process tree]{stochastic process tree (.sptree)} and \hyperref[filehandler:process tree markup language]{process tree markup language (.ptml)}.\\
\texttt{-o} or \texttt{--output} <\texttt{FILE}> &
The file to which the results must be written. Based on the file extension, Ebi will output either a Petri net markup language (.pnml -- Section~\ref{filehandler:Petri net markup language}), a portable document format (.pdf -- Section~\ref{filehandler:portable document format}), a scalable vector graphics (.svg -- Section~\ref{filehandler:scalable vector graphics}), a stochastic process tree (.sptree -- Section~\ref{filehandler:stochastic process tree}) or a process tree markup language (.ptml -- Section~\ref{filehandler:process tree markup language}).
If the parameter is not given, the results will be written to STDOUT as a stochastic process tree (.sptree).\\
&\textit{Mandatory:} \quad no\\
\texttt{-a} or \texttt{--approximate} & Use approximate arithmetic instead of exact arithmetic.\\
&\textit{Mandatory:}\quad no\\
\bottomrule
\end{tabularx}
\noindent Output: stochastic process tree, which can be written as a Petri net markup language (.pnml -- Section~\ref{filehandler:Petri net markup language}), a portable document format (.pdf -- Section~\ref{filehandler:portable document format}), a scalable vector graphics (.svg -- Section~\ref{filehandler:scalable vector graphics}), a stochastic process tree (.sptree -- Section~\ref{filehandler:stochastic process tree}) or a process tree markup language (.ptml -- Section~\ref{filehandler:process tree markup language}).
\\This command is not available in Java and ProM.
\subsection{\texttt{Ebi discover uniform labelled-petri-net}}
\label{command:Ebi discover uniform labelled-petri-net}
Alias: \texttt{Ebi disc uni lpn}.\\
Give each transition a weight of 1 in a labelled Petri net.\\
\begin{tabularx}{\linewidth}{lX}
\toprule
Parameter \\\midrule
<\texttt{MODEL}>&A labelled Petri net.\\
&\textit{Mandatory:} \quad yes, though it can be given on STDIN by giving a `-' on the command line.\\
&\textit{Accepted values:}\quad \hyperref[filehandler:labelled Petri net]{labelled Petri net (.lpn)}, \hyperref[filehandler:stochastic deterministic finite automaton]{stochastic deterministic finite automaton (.sdfa)}, \hyperref[filehandler:stochastic labelled Petri net]{stochastic labelled Petri net (.slpn)}, \hyperref[filehandler:directly follows model]{directly follows model (.dfm)}, \hyperref[filehandler:deterministic finite automaton]{deterministic finite automaton (.dfa)}, \hyperref[filehandler:directly follows graph]{directly follows graph (.dfg)}, \hyperref[filehandler:stochastic directly follows model]{stochastic directly follows model (.sdfm)}, \hyperref[filehandler:Petri net markup language]{Petri net markup language (.pnml)}, \hyperref[filehandler:process tree]{process tree (.ptree)}, \hyperref[filehandler:stochastic process tree]{stochastic process tree (.sptree)} and \hyperref[filehandler:process tree markup language]{process tree markup language (.ptml)}.\\
\texttt{-o} or \texttt{--output} <\texttt{FILE}> &
The file to which the results must be written. Based on the file extension, Ebi will output either a labelled Petri net (.lpn -- Section~\ref{filehandler:labelled Petri net}), a LoLa Petri net (.lola -- Section~\ref{filehandler:LoLa Petri net}), a Petri net markup language (.pnml -- Section~\ref{filehandler:Petri net markup language}), a stochastic labelled Petri net (.slpn -- Section~\ref{filehandler:stochastic labelled Petri net}), a portable document format (.pdf -- Section~\ref{filehandler:portable document format}) or a scalable vector graphics (.svg -- Section~\ref{filehandler:scalable vector graphics}).
If the parameter is not given, the results will be written to STDOUT as a stochastic labelled Petri net (.slpn).\\
&\textit{Mandatory:} \quad no\\
\texttt{-a} or \texttt{--approximate} & Use approximate arithmetic instead of exact arithmetic.\\
&\textit{Mandatory:}\quad no\\
\bottomrule
\end{tabularx}
\noindent Output: stochastic labelled Petri net, which can be written as a labelled Petri net (.lpn -- Section~\ref{filehandler:labelled Petri net}), a LoLa Petri net (.lola -- Section~\ref{filehandler:LoLa Petri net}), a Petri net markup language (.pnml -- Section~\ref{filehandler:Petri net markup language}), a stochastic labelled Petri net (.slpn -- Section~\ref{filehandler:stochastic labelled Petri net}), a portable document format (.pdf -- Section~\ref{filehandler:portable document format}) or a scalable vector graphics (.svg -- Section~\ref{filehandler:scalable vector graphics}).
\\This command is available in Java and ProM.
\subsection{\texttt{Ebi discover uniform process-tree}}
\label{command:Ebi discover uniform process-tree}
Alias: \texttt{Ebi disc uni ptree}.\\
Give each leaf a weight of 1 in a process tree.\\
\begin{tabularx}{\linewidth}{lX}
\toprule
Parameter \\\midrule
<\texttt{TREE}>&A process tree.\\
&\textit{Mandatory:} \quad yes, though it can be given on STDIN by giving a `-' on the command line.\\
&\textit{Accepted values:}\quad \hyperref[filehandler:stochastic process tree]{stochastic process tree (.sptree)}, \hyperref[filehandler:process tree]{process tree (.ptree)} and \hyperref[filehandler:process tree markup language]{process tree markup language (.ptml)}.\\
\texttt{-o} or \texttt{--output} <\texttt{FILE}> &
The file to which the results must be written. Based on the file extension, Ebi will output either a Petri net markup language (.pnml -- Section~\ref{filehandler:Petri net markup language}), a portable document format (.pdf -- Section~\ref{filehandler:portable document format}), a scalable vector graphics (.svg -- Section~\ref{filehandler:scalable vector graphics}), a stochastic process tree (.sptree -- Section~\ref{filehandler:stochastic process tree}) or a process tree markup language (.ptml -- Section~\ref{filehandler:process tree markup language}).
If the parameter is not given, the results will be written to STDOUT as a stochastic process tree (.sptree).\\
&\textit{Mandatory:} \quad no\\
\texttt{-a} or \texttt{--approximate} & Use approximate arithmetic instead of exact arithmetic.\\
&\textit{Mandatory:}\quad no\\
\bottomrule
\end{tabularx}
\noindent Output: stochastic process tree, which can be written as a Petri net markup language (.pnml -- Section~\ref{filehandler:Petri net markup language}), a portable document format (.pdf -- Section~\ref{filehandler:portable document format}), a scalable vector graphics (.svg -- Section~\ref{filehandler:scalable vector graphics}), a stochastic process tree (.sptree -- Section~\ref{filehandler:stochastic process tree}) or a process tree markup language (.ptml -- Section~\ref{filehandler:process tree markup language}).
\\This command is not available in Java and ProM.
\subsection{\texttt{Ebi discover-non-stochastic flower deterministic-finite-automaton}}
\label{command:Ebi discover-non-stochastic flower deterministic-finite-automaton}
Alias: \texttt{Ebi dins flw dfa}.\\
Discover a DFA that supports any trace with the activities of the log.\\
\begin{tabularx}{\linewidth}{lX}
\toprule
Parameter \\\midrule
<\texttt{FILE}>&A file with activities.\\
&\textit{Mandatory:} \quad yes, though it can be given on STDIN by giving a `-' on the command line.\\
&\textit{Accepted values:}\quad \hyperref[filehandler:Petri net markup language]{Petri net markup language (.pnml)}, \hyperref[filehandler:stochastic labelled Petri net]{stochastic labelled Petri net (.slpn)}, \hyperref[filehandler:LoLa Petri net]{LoLa Petri net (.lola)}, \hyperref[filehandler:stochastic process tree]{stochastic process tree (.sptree)}, \hyperref[filehandler:finite stochastic language]{finite stochastic language (.slang)}, \hyperref[filehandler:directly follows graph]{directly follows graph (.dfg)}, \hyperref[filehandler:compressed event log]{compressed event log (.xes.gz)}, \hyperref[filehandler:event log]{event log (.xes)}, \hyperref[filehandler:stochastic directly follows model]{stochastic directly follows model (.sdfm)}, \hyperref[filehandler:deterministic finite automaton]{deterministic finite automaton (.dfa)}, \hyperref[filehandler:process tree]{process tree (.ptree)}, \hyperref[filehandler:stochastic language of alignments]{stochastic language of alignments (.sali)}, \hyperref[filehandler:language of alignments]{language of alignments (.ali)}, \hyperref[filehandler:labelled Petri net]{labelled Petri net (.lpn)}, \hyperref[filehandler:finite language]{finite language (.lang)}, \hyperref[filehandler:stochastic deterministic finite automaton]{stochastic deterministic finite automaton (.sdfa)} and \hyperref[filehandler:directly follows model]{directly follows model (.dfm)}.\\
\texttt{-o} or \texttt{--output} <\texttt{FILE}> &
The file to which the results must be written. Based on the file extension, Ebi will output either a deterministic finite automaton (.dfa -- Section~\ref{filehandler:deterministic finite automaton}), a labelled Petri net (.lpn -- Section~\ref{filehandler:labelled Petri net}), a LoLa Petri net (.lola -- Section~\ref{filehandler:LoLa Petri net}), a Petri net markup language (.pnml -- Section~\ref{filehandler:Petri net markup language}), a portable document format (.pdf -- Section~\ref{filehandler:portable document format}) or a scalable vector graphics (.svg -- Section~\ref{filehandler:scalable vector graphics}).
If the parameter is not given, the results will be written to STDOUT as a deterministic finite automaton (.dfa).\\
&\textit{Mandatory:} \quad no\\
\texttt{-a} or \texttt{--approximate} & Use approximate arithmetic instead of exact arithmetic.\\
&\textit{Mandatory:}\quad no\\
\bottomrule
\end{tabularx}
\noindent Output: deterministic finite automaton, which can be written as a deterministic finite automaton (.dfa -- Section~\ref{filehandler:deterministic finite automaton}), a labelled Petri net (.lpn -- Section~\ref{filehandler:labelled Petri net}), a LoLa Petri net (.lola -- Section~\ref{filehandler:LoLa Petri net}), a Petri net markup language (.pnml -- Section~\ref{filehandler:Petri net markup language}), a portable document format (.pdf -- Section~\ref{filehandler:portable document format}) or a scalable vector graphics (.svg -- Section~\ref{filehandler:scalable vector graphics}).
\\This command is available in Java and ProM.
\subsection{\texttt{Ebi discover-non-stochastic flower process-tree}}
\label{command:Ebi discover-non-stochastic flower process-tree}
Alias: \texttt{Ebi dins flw ptree}.\\
Discover a process tree that supports any trace with the activities of the log.\\
\begin{tabularx}{\linewidth}{lX}
\toprule
Parameter \\\midrule
<\texttt{FILE}>&A file with activities.\\
&\textit{Mandatory:} \quad yes, though it can be given on STDIN by giving a `-' on the command line.\\
&\textit{Accepted values:}\quad \hyperref[filehandler:language of alignments]{language of alignments (.ali)}, \hyperref[filehandler:event log]{event log (.xes)}, \hyperref[filehandler:directly follows graph]{directly follows graph (.dfg)}, \hyperref[filehandler:labelled Petri net]{labelled Petri net (.lpn)}, \hyperref[filehandler:stochastic directly follows model]{stochastic directly follows model (.sdfm)}, \hyperref[filehandler:stochastic language of alignments]{stochastic language of alignments (.sali)}, \hyperref[filehandler:stochastic process tree]{stochastic process tree (.sptree)}, \hyperref[filehandler:finite stochastic language]{finite stochastic language (.slang)}, \hyperref[filehandler:stochastic labelled Petri net]{stochastic labelled Petri net (.slpn)}, \hyperref[filehandler:compressed event log]{compressed event log (.xes.gz)}, \hyperref[filehandler:process tree]{process tree (.ptree)}, \hyperref[filehandler:Petri net markup language]{Petri net markup language (.pnml)}, \hyperref[filehandler:directly follows model]{directly follows model (.dfm)}, \hyperref[filehandler:deterministic finite automaton]{deterministic finite automaton (.dfa)}, \hyperref[filehandler:LoLa Petri net]{LoLa Petri net (.lola)}, \hyperref[filehandler:stochastic deterministic finite automaton]{stochastic deterministic finite automaton (.sdfa)} and \hyperref[filehandler:finite language]{finite language (.lang)}.\\
\texttt{-o} or \texttt{--output} <\texttt{FILE}> &
The file to which the results must be written. Based on the file extension, Ebi will output either a labelled Petri net (.lpn -- Section~\ref{filehandler:labelled Petri net}), a LoLa Petri net (.lola -- Section~\ref{filehandler:LoLa Petri net}), a Petri net markup language (.pnml -- Section~\ref{filehandler:Petri net markup language}), a process tree (.ptree -- Section~\ref{filehandler:process tree}), a portable document format (.pdf -- Section~\ref{filehandler:portable document format}), a scalable vector graphics (.svg -- Section~\ref{filehandler:scalable vector graphics}) or a process tree markup language (.ptml -- Section~\ref{filehandler:process tree markup language}).
If the parameter is not given, the results will be written to STDOUT as a process tree (.ptree).\\
&\textit{Mandatory:} \quad no\\
\texttt{-a} or \texttt{--approximate} & Use approximate arithmetic instead of exact arithmetic.\\
&\textit{Mandatory:}\quad no\\
\bottomrule
\end{tabularx}
\noindent Output: process tree, which can be written as a labelled Petri net (.lpn -- Section~\ref{filehandler:labelled Petri net}), a LoLa Petri net (.lola -- Section~\ref{filehandler:LoLa Petri net}), a Petri net markup language (.pnml -- Section~\ref{filehandler:Petri net markup language}), a process tree (.ptree -- Section~\ref{filehandler:process tree}), a portable document format (.pdf -- Section~\ref{filehandler:portable document format}), a scalable vector graphics (.svg -- Section~\ref{filehandler:scalable vector graphics}) or a process tree markup language (.ptml -- Section~\ref{filehandler:process tree markup language}).
\\This command is available in Java and ProM.
\subsection{\texttt{Ebi discover-non-stochastic prefix-tree deterministic-finite-automaton}}
\label{command:Ebi discover-non-stochastic prefix-tree deterministic-finite-automaton}
Alias: \texttt{Ebi dins pfxt dfa}.\\
Discover a DFA that is a prefix tree of the log.\\
\begin{tabularx}{\linewidth}{lX}
\toprule
Parameter \\\midrule
<\texttt{LANG}>&A finite language.\\
&\textit{Mandatory:} \quad yes, though it can be given on STDIN by giving a `-' on the command line.\\
&\textit{Accepted values:}\quad \hyperref[filehandler:finite stochastic language]{finite stochastic language (.slang)}, \hyperref[filehandler:finite language]{finite language (.lang)}, \hyperref[filehandler:event log]{event log (.xes)} and \hyperref[filehandler:compressed event log]{compressed event log (.xes.gz)}.\\
\texttt{-o} or \texttt{--output} <\texttt{FILE}> &
The file to which the results must be written. Based on the file extension, Ebi will output either a deterministic finite automaton (.dfa -- Section~\ref{filehandler:deterministic finite automaton}), a labelled Petri net (.lpn -- Section~\ref{filehandler:labelled Petri net}), a LoLa Petri net (.lola -- Section~\ref{filehandler:LoLa Petri net}), a Petri net markup language (.pnml -- Section~\ref{filehandler:Petri net markup language}), a portable document format (.pdf -- Section~\ref{filehandler:portable document format}) or a scalable vector graphics (.svg -- Section~\ref{filehandler:scalable vector graphics}).
If the parameter is not given, the results will be written to STDOUT as a deterministic finite automaton (.dfa).\\
&\textit{Mandatory:} \quad no\\
\texttt{-a} or \texttt{--approximate} & Use approximate arithmetic instead of exact arithmetic.\\
&\textit{Mandatory:}\quad no\\
\bottomrule
\end{tabularx}
\noindent Output: deterministic finite automaton, which can be written as a deterministic finite automaton (.dfa -- Section~\ref{filehandler:deterministic finite automaton}), a labelled Petri net (.lpn -- Section~\ref{filehandler:labelled Petri net}), a LoLa Petri net (.lola -- Section~\ref{filehandler:LoLa Petri net}), a Petri net markup language (.pnml -- Section~\ref{filehandler:Petri net markup language}), a portable document format (.pdf -- Section~\ref{filehandler:portable document format}) or a scalable vector graphics (.svg -- Section~\ref{filehandler:scalable vector graphics}).
\\This command is available in Java and ProM.
\subsection{\texttt{Ebi discover-non-stochastic prefix-tree process-tree}}
\label{command:Ebi discover-non-stochastic prefix-tree process-tree}
Alias: \texttt{Ebi dins pfxt tree}.\\
Discover a process tree that is a prefix tree of the log.\\
\begin{tabularx}{\linewidth}{lX}
\toprule
Parameter \\\midrule
<\texttt{LANG}>&A finite language.\\
&\textit{Mandatory:} \quad yes, though it can be given on STDIN by giving a `-' on the command line.\\
&\textit{Accepted values:}\quad \hyperref[filehandler:finite stochastic language]{finite stochastic language (.slang)}, \hyperref[filehandler:compressed event log]{compressed event log (.xes.gz)}, \hyperref[filehandler:event log]{event log (.xes)} and \hyperref[filehandler:finite language]{finite language (.lang)}.\\
\texttt{-o} or \texttt{--output} <\texttt{FILE}> &
The file to which the results must be written. Based on the file extension, Ebi will output either a labelled Petri net (.lpn -- Section~\ref{filehandler:labelled Petri net}), a LoLa Petri net (.lola -- Section~\ref{filehandler:LoLa Petri net}), a Petri net markup language (.pnml -- Section~\ref{filehandler:Petri net markup language}), a process tree (.ptree -- Section~\ref{filehandler:process tree}), a portable document format (.pdf -- Section~\ref{filehandler:portable document format}), a scalable vector graphics (.svg -- Section~\ref{filehandler:scalable vector graphics}) or a process tree markup language (.ptml -- Section~\ref{filehandler:process tree markup language}).
If the parameter is not given, the results will be written to STDOUT as a process tree (.ptree).\\
&\textit{Mandatory:} \quad no\\
\texttt{-a} or \texttt{--approximate} & Use approximate arithmetic instead of exact arithmetic.\\
&\textit{Mandatory:}\quad no\\
\bottomrule
\end{tabularx}
\noindent Output: process tree, which can be written as a labelled Petri net (.lpn -- Section~\ref{filehandler:labelled Petri net}), a LoLa Petri net (.lola -- Section~\ref{filehandler:LoLa Petri net}), a Petri net markup language (.pnml -- Section~\ref{filehandler:Petri net markup language}), a process tree (.ptree -- Section~\ref{filehandler:process tree}), a portable document format (.pdf -- Section~\ref{filehandler:portable document format}), a scalable vector graphics (.svg -- Section~\ref{filehandler:scalable vector graphics}) or a process tree markup language (.ptml -- Section~\ref{filehandler:process tree markup language}).
\\This command is available in Java and ProM.
\subsection{\texttt{Ebi information}}
\label{command:Ebi information}
Alias: \texttt{Ebi info}.\\
Show information about a file.\\
\begin{tabularx}{\linewidth}{lX}
\toprule
Parameter \\\midrule
<\texttt{FILE}>&Any file supported by Ebi.\\
&\textit{Mandatory:} \quad yes, though it can be given on STDIN by giving a `-' on the command line.\\
&\textit{Accepted values:}\quad \hyperref[filehandler:process tree]{process tree (.ptree)}, \hyperref[filehandler:Petri net markup language]{Petri net markup language (.pnml)}, \hyperref[filehandler:scalable vector graphics]{scalable vector graphics (.svg)}, \hyperref[filehandler:stochastic directly follows model]{stochastic directly follows model (.sdfm)}, \hyperref[filehandler:compressed event log]{compressed event log (.xes.gz)}, \hyperref[filehandler:stochastic deterministic finite automaton]{stochastic deterministic finite automaton (.sdfa)}, \hyperref[filehandler:labelled Petri net]{labelled Petri net (.lpn)}, \hyperref[filehandler:LoLa Petri net]{LoLa Petri net (.lola)}, \hyperref[filehandler:directly follows graph]{directly follows graph (.dfg)}, \hyperref[filehandler:deterministic finite automaton]{deterministic finite automaton (.dfa)}, \hyperref[filehandler:event log]{event log (.xes)}, \hyperref[filehandler:process tree markup language]{process tree markup language (.ptml)}, \hyperref[filehandler:directly follows model]{directly follows model (.dfm)}, \hyperref[filehandler:finite stochastic language]{finite stochastic language (.slang)}, \hyperref[filehandler:finite language]{finite language (.lang)}, \hyperref[filehandler:language of alignments]{language of alignments (.ali)}, \hyperref[filehandler:stochastic process tree]{stochastic process tree (.sptree)}, \hyperref[filehandler:executions]{executions (.exs)}, \hyperref[filehandler:portable document format]{portable document format (.pdf)}, \hyperref[filehandler:stochastic language of alignments]{stochastic language of alignments (.sali)} and \hyperref[filehandler:stochastic labelled Petri net]{stochastic labelled Petri net (.slpn)}.\\
\texttt{-o} or \texttt{--output} <\texttt{FILE}> &
The text file to which the result must be written. If the parameter is not given, the results will be written to STDOUT.\\
&\textit{Mandatory:} \quad no\\
\texttt{-a} or \texttt{--approximate} & Use approximate arithmetic instead of exact arithmetic.\\
&\textit{Mandatory:}\quad no\\
\bottomrule
\end{tabularx}
\noindent Output: text, which can be written as  text.
\\This command is available in Java and ProM.
\subsection{\texttt{Ebi itself graph}}
\label{command:Ebi itself graph}
Print the graph of Ebi.\\
\begin{tabularx}{\linewidth}{lX}
\toprule
Parameter \\\midrule
\texttt{-o} or \texttt{--output} <\texttt{FILE}> &
The file to which the results must be written. Based on the file extension, Ebi will output either a portable document format (.pdf -- Section~\ref{filehandler:portable document format}) or a scalable vector graphics (.svg -- Section~\ref{filehandler:scalable vector graphics}).
If the parameter is not given, the results will be written to STDOUT as a scalable vector graphics (.svg).\\
&\textit{Mandatory:} \quad no\\
\bottomrule
\end{tabularx}
\noindent Output: scalable vector graphics, which can be written as a portable document format (.pdf -- Section~\ref{filehandler:portable document format}) or a scalable vector graphics (.svg -- Section~\ref{filehandler:scalable vector graphics}).
\\This command is not available in Java and ProM.
\subsection{\texttt{Ebi itself html}}
\label{command:Ebi itself html}
Print parts of the website.\\
\begin{tabularx}{\linewidth}{lX}
\toprule
Parameter \\\midrule
\texttt{-o} or \texttt{--output} <\texttt{FILE}> &
The text file to which the result must be written. If the parameter is not given, the results will be written to STDOUT.\\
&\textit{Mandatory:} \quad no\\
\bottomrule
\end{tabularx}
\noindent Output: text, which can be written as  text.
\\This command is available in Java and ProM.
\subsection{\texttt{Ebi itself java}}
\label{command:Ebi itself java}
Print the classes for Java.\\
\begin{tabularx}{\linewidth}{lX}
\toprule
Parameter \\\midrule
\texttt{-o} or \texttt{--output} <\texttt{FILE}> &
The text file to which the result must be written. If the parameter is not given, the results will be written to STDOUT.\\
&\textit{Mandatory:} \quad no\\
\bottomrule
\end{tabularx}
\noindent Output: text, which can be written as  text.
\\This command is available in Java and ProM.
\subsection{\texttt{Ebi itself logo}}
\label{command:Ebi itself logo}
Alias: \texttt{Ebi it log}.\\
Print the logo of Ebi.\\
\begin{tabularx}{\linewidth}{lX}
\toprule
Parameter \\\midrule
\texttt{-o} or \texttt{--output} <\texttt{FILE}> &
The text file to which the result must be written. If the parameter is not given, the results will be written to STDOUT.\\
&\textit{Mandatory:} \quad no\\
\bottomrule
\end{tabularx}
\noindent Output: text, which can be written as  text.
\\This command is available in Java and ProM.
\subsection{\texttt{Ebi itself manual}}
\label{command:Ebi itself manual}
Alias: \texttt{Ebi it man}.\\
Print the automatically generated parts of the manual of Ebi in Latex format.\\
\begin{tabularx}{\linewidth}{lX}
\toprule
Parameter \\\midrule
\texttt{-o} or \texttt{--output} <\texttt{FILE}> &
The text file to which the result must be written. If the parameter is not given, the results will be written to STDOUT.\\
&\textit{Mandatory:} \quad no\\
\bottomrule
\end{tabularx}
\noindent Output: text, which can be written as  text.
\\This command is available in Java and ProM.
\subsection{\texttt{Ebi probability explain-trace}}
\label{command:Ebi probability explain-trace}
Alias: \texttt{Ebi prob exptra}.\\
Compute the most likely explanation of a trace given the stochastic model.\\
\begin{tabularx}{\linewidth}{lX}
\toprule
Parameter \\\midrule
<\texttt{FILE}>&The model.\\
&\textit{Mandatory:} \quad yes, though it can be given on STDIN by giving a `-' on the command line.\\
&\textit{Accepted values:}\quad \hyperref[filehandler:compressed event log]{compressed event log (.xes.gz)}, \hyperref[filehandler:stochastic directly follows model]{stochastic directly follows model (.sdfm)}, \hyperref[filehandler:stochastic labelled Petri net]{stochastic labelled Petri net (.slpn)}, \hyperref[filehandler:stochastic deterministic finite automaton]{stochastic deterministic finite automaton (.sdfa)}, \hyperref[filehandler:stochastic process tree]{stochastic process tree (.sptree)}, \hyperref[filehandler:event log]{event log (.xes)}, \hyperref[filehandler:directly follows graph]{directly follows graph (.dfg)} and \hyperref[filehandler:finite stochastic language]{finite stochastic language (.slang)}.\\
<\texttt{VALUE}>&Balance between 0 (=only consider deviations) to 1 (=only consider weight in the model)\\
&\textit{Mandatory:} \quad yes, though it can be given on STDIN by giving a `-' on the command line.\\
&\textit{Accepted values:}\quad fraction between 0 and 1.\\
<\texttt{TRACE}>
&The trace.\\
&\textit{Mandatory:}\quad yes\\
\texttt{-o} or \texttt{--output} <\texttt{FILE}> &
The language of alignments (.ali) file to which the result must be written. If the parameter is not given, the results will be written to STDOUT.\\
&\textit{Mandatory:} \quad no\\
\texttt{-a} or \texttt{--approximate} & Use approximate arithmetic instead of exact arithmetic.\\
&\textit{Mandatory:}\quad no\\
\bottomrule
\end{tabularx}
\noindent Output: alignments, which can be written as a language of alignments (.ali -- Section~\ref{filehandler:language of alignments}).
\\This command is not available in Java and ProM.
\subsection{\texttt{Ebi probability log}}
\label{command:Ebi probability log}
Compute the probability that a stochastic model produces any trace of a log.\\
More information: ~\cite{DBLP:journals/is/LeemansMM24}.\\
\begin{tabularx}{\linewidth}{lX}
\toprule
Parameter \\\midrule
<\texttt{FILE\_1}>&The queriable stochastic language (model).\\
&\textit{Mandatory:} \quad yes, though it can be given on STDIN by giving a `-' on the command line.\\
&\textit{Accepted values:}\quad \hyperref[filehandler:finite stochastic language]{finite stochastic language (.slang)}, \hyperref[filehandler:stochastic deterministic finite automaton]{stochastic deterministic finite automaton (.sdfa)}, \hyperref[filehandler:compressed event log]{compressed event log (.xes.gz)}, \hyperref[filehandler:event log]{event log (.xes)}, \hyperref[filehandler:stochastic labelled Petri net]{stochastic labelled Petri net (.slpn)} and \hyperref[filehandler:stochastic process tree]{stochastic process tree (.sptree)}.\\
<\texttt{FILE\_2}>&The finite language (log).\\
&\textit{Mandatory:} \quad yes, though it can be given on STDIN by giving a `-' on the command line.\\
&\textit{Accepted values:}\quad \hyperref[filehandler:finite stochastic language]{finite stochastic language (.slang)}, \hyperref[filehandler:finite language]{finite language (.lang)}, \hyperref[filehandler:compressed event log]{compressed event log (.xes.gz)} and \hyperref[filehandler:event log]{event log (.xes)}.\\
\texttt{-o} or \texttt{--output} <\texttt{FILE}> &
The fraction file to which the result must be written. If the parameter is not given, the results will be written to STDOUT.\\
&\textit{Mandatory:} \quad no\\
\texttt{-a} or \texttt{--approximate} & Use approximate arithmetic instead of exact arithmetic.\\
&\textit{Mandatory:}\quad no\\
\bottomrule
\end{tabularx}
\noindent Output: fraction, which can be written as a fraction.
\\This command is available in Java and ProM.
\subsection{\texttt{Ebi probability trace}}
\label{command:Ebi probability trace}
Alias: \texttt{Ebi prob trac}.\\
Compute the probability of a trace in a stochastic model.\\
More information: ~\cite{DBLP:journals/is/LeemansMM24}.\\
\begin{tabularx}{\linewidth}{lX}
\toprule
Parameter \\\midrule
<\texttt{FILE}>&The queriable stochastic language (model).\\
&\textit{Mandatory:} \quad yes, though it can be given on STDIN by giving a `-' on the command line.\\
&\textit{Accepted values:}\quad \hyperref[filehandler:event log]{event log (.xes)}, \hyperref[filehandler:compressed event log]{compressed event log (.xes.gz)}, \hyperref[filehandler:finite stochastic language]{finite stochastic language (.slang)}, \hyperref[filehandler:stochastic process tree]{stochastic process tree (.sptree)}, \hyperref[filehandler:stochastic deterministic finite automaton]{stochastic deterministic finite automaton (.sdfa)} and \hyperref[filehandler:stochastic labelled Petri net]{stochastic labelled Petri net (.slpn)}.\\
<\texttt{TRACE}>
&The trace.\\
&\textit{Mandatory:}\quad yes\\
\texttt{-o} or \texttt{--output} <\texttt{FILE}> &
The fraction file to which the result must be written. If the parameter is not given, the results will be written to STDOUT.\\
&\textit{Mandatory:} \quad no\\
\texttt{-a} or \texttt{--approximate} & Use approximate arithmetic instead of exact arithmetic.\\
&\textit{Mandatory:}\quad no\\
\bottomrule
\end{tabularx}
\noindent Output: fraction, which can be written as a fraction.
\\This command is not available in Java and ProM.
\subsection{\texttt{Ebi sample}}
\label{command:Ebi sample}
Alias: \texttt{Ebi sam}.\\
Sample traces randomly. Please note that this may run forever if the model contains a livelock.\\
\begin{tabularx}{\linewidth}{lX}
\toprule
Parameter \\\midrule
<\texttt{FILE}>&The stochastic semantics (model).\\
&\textit{Mandatory:} \quad yes, though it can be given on STDIN by giving a `-' on the command line.\\
&\textit{Accepted values:}\quad \hyperref[filehandler:stochastic process tree]{stochastic process tree (.sptree)}, \hyperref[filehandler:finite stochastic language]{finite stochastic language (.slang)}, \hyperref[filehandler:stochastic directly follows model]{stochastic directly follows model (.sdfm)}, \hyperref[filehandler:directly follows graph]{directly follows graph (.dfg)}, \hyperref[filehandler:stochastic labelled Petri net]{stochastic labelled Petri net (.slpn)}, \hyperref[filehandler:event log]{event log (.xes)}, \hyperref[filehandler:compressed event log]{compressed event log (.xes.gz)} and \hyperref[filehandler:stochastic deterministic finite automaton]{stochastic deterministic finite automaton (.sdfa)}.\\
<\texttt{NUMBER\_OF\_TRACES}>&The number of traces to be sampled.\\
&\textit{Mandatory:} \quad yes, though it can be given on STDIN by giving a `-' on the command line.\\
&\textit{Accepted values:}\quad integer above 1.\\
\texttt{-o} or \texttt{--output} <\texttt{FILE}> &
The file to which the results must be written. Based on the file extension, Ebi will output either a deterministic finite automaton (.dfa -- Section~\ref{filehandler:deterministic finite automaton}), a finite language (.lang -- Section~\ref{filehandler:finite language}), a finite stochastic language (.slang -- Section~\ref{filehandler:finite stochastic language}) or a stochastic deterministic finite automaton (.sdfa -- Section~\ref{filehandler:stochastic deterministic finite automaton}).
If the parameter is not given, the results will be written to STDOUT as a finite stochastic language (.slang).\\
&\textit{Mandatory:} \quad no\\
\texttt{-a} or \texttt{--approximate} & Use approximate arithmetic instead of exact arithmetic.\\
&\textit{Mandatory:}\quad no\\
\bottomrule
\end{tabularx}
\noindent Output: finite stochastic language, which can be written as a deterministic finite automaton (.dfa -- Section~\ref{filehandler:deterministic finite automaton}), a finite language (.lang -- Section~\ref{filehandler:finite language}), a finite stochastic language (.slang -- Section~\ref{filehandler:finite stochastic language}) or a stochastic deterministic finite automaton (.sdfa -- Section~\ref{filehandler:stochastic deterministic finite automaton}).
\\This command is not available in Java and ProM.
\subsection{\texttt{Ebi test bootstrap-test}}
\label{command:Ebi test bootstrap-test}
Alias: \texttt{Ebi tst btst}.\\
Test the hypothesis that the logs are derived from identical processes.\\
More information: \cite{DBLP:journals/tkde/LeemansMPH23}.\\
\begin{tabularx}{\linewidth}{lX}
\toprule
Parameter \\\midrule
<\texttt{LANG\_1}>&The first event log for which the test is to be performed.\\
&\textit{Mandatory:} \quad yes, though it can be given on STDIN by giving a `-' on the command line.\\
&\textit{Accepted values:}\quad \hyperref[filehandler:event log]{event log (.xes)}, \hyperref[filehandler:compressed event log]{compressed event log (.xes.gz)} and \hyperref[filehandler:finite stochastic language]{finite stochastic language (.slang)}.\\
<\texttt{LANG\_2}>&The first event log for which the test is to be performed.\\
&\textit{Mandatory:} \quad yes, though it can be given on STDIN by giving a `-' on the command line.\\
&\textit{Accepted values:}\quad \hyperref[filehandler:finite stochastic language]{finite stochastic language (.slang)}, \hyperref[filehandler:event log]{event log (.xes)} and \hyperref[filehandler:compressed event log]{compressed event log (.xes.gz)}.\\
<\texttt{SAMPLES}>&The number of samples taken.\\
&\textit{Mandatory:} \quad no: if no value is provided, a default of 500 will be used. It can also be provided on STDIN by giving a `-' on the command line.\\
&\textit{Accepted values:}\quad integer above 1.\\
<\texttt{P-VALUE}>&The threshold p-value\\
&\textit{Mandatory:} \quad no: if no value is provided, a default of 1/20 will be used. It can also be provided on STDIN by giving a `-' on the command line.\\
&\textit{Accepted values:}\quad fraction between 0 and 1.\\
\texttt{-o} or \texttt{--output} <\texttt{FILE}> &
The text file to which the result must be written. If the parameter is not given, the results will be written to STDOUT.\\
&\textit{Mandatory:} \quad no\\
\texttt{-a} or \texttt{--approximate} & Use approximate arithmetic instead of exact arithmetic.\\
&\textit{Mandatory:}\quad no\\
\bottomrule
\end{tabularx}
\noindent Output: text, which can be written as  text.
\\This command is available in Java and ProM.
\subsection{\texttt{Ebi test log-categorical-attribute}}
\label{command:Ebi test log-categorical-attribute}
Alias: \texttt{Ebi tst lcat}.\\
Test the hypothesis that the sub-logs defined by the categorical attribute are derived from identical processes.\\
More information: \cite{DBLP:journals/tkde/LeemansMPH23}.\\
\begin{tabularx}{\linewidth}{lX}
\toprule
Parameter \\\midrule
<\texttt{FILE}>&The event log for which the test is to be performed.\\
&\textit{Mandatory:} \quad yes, though it can be given on STDIN by giving a `-' on the command line.\\
&\textit{Accepted values:}\quad \hyperref[filehandler:compressed event log]{compressed event log (.xes.gz)} and \hyperref[filehandler:event log]{event log (.xes)}.\\
<\texttt{ATTRIBUTE}>&The trace attribute for which the test is to be performed. The trace attributes of a log can be found using `Ebi info`.\\
&\textit{Mandatory:} \quad yes, though it can be given on STDIN by giving a `-' on the command line.\\
&\textit{Accepted values:}\quad text.\\
<\texttt{SAMPLES}>&The number of samples taken.\\
&\textit{Mandatory:} \quad no: if no value is provided, a default of 500 will be used. It can also be provided on STDIN by giving a `-' on the command line.\\
&\textit{Accepted values:}\quad integer above 1.\\
<\texttt{P-VALUE}>&The threshold p-value.\\
&\textit{Mandatory:} \quad no: if no value is provided, a default of 1/20 will be used. It can also be provided on STDIN by giving a `-' on the command line.\\
&\textit{Accepted values:}\quad fraction between 0 and 1.\\
\texttt{-o} or \texttt{--output} <\texttt{FILE}> &
The text file to which the result must be written. If the parameter is not given, the results will be written to STDOUT.\\
&\textit{Mandatory:} \quad no\\
\texttt{-a} or \texttt{--approximate} & Use approximate arithmetic instead of exact arithmetic.\\
&\textit{Mandatory:}\quad no\\
\bottomrule
\end{tabularx}
\noindent Output: text, which can be written as  text.
\\This command is available in Java and ProM.
\subsection{\texttt{Ebi validate}}
\label{command:Ebi validate}
Alias: \texttt{Ebi vali}.\\
Attempt to parse any file supported by Ebi. If you do not know the type the file should have, try `Ebi info`.\\
\begin{tabularx}{\linewidth}{lX}
\toprule
Parameter \\\midrule
<\texttt{TYPE}>&The type for which parsing should be attempted.\\
&\textit{Mandatory:} \quad yes, though it can be given on STDIN by giving a `-' on the command line.\\
&\textit{Accepted values:}\quad the file extension of any file type supported by Ebi (xes.gz, dfg, dfa, dfm, sdfm, xes, exs, lang, slang, lpn, ali, lola, pnml, sdfa, slpn, ptree, sali, sptree or ptml).\\
<\texttt{FILE}>
&The file to be parsed.\\
&\textit{Mandatory:}\quad yes\\
\texttt{-o} or \texttt{--output} <\texttt{FILE}> &
The text file to which the result must be written. If the parameter is not given, the results will be written to STDOUT.\\
&\textit{Mandatory:} \quad no\\
\texttt{-a} or \texttt{--approximate} & Use approximate arithmetic instead of exact arithmetic.\\
&\textit{Mandatory:}\quad no\\
\bottomrule
\end{tabularx}
\noindent Output: text, which can be written as  text.
\\This command is not available in Java and ProM.
\subsection{\texttt{Ebi visualise graph}}
\label{command:Ebi visualise graph}
Visualise a file as a graph.\\
\begin{tabularx}{\linewidth}{lX}
\toprule
Parameter \\\midrule
<\texttt{FILE}>&Any file that can be visualised as a graph.\\
&\textit{Mandatory:} \quad yes, though it can be given on STDIN by giving a `-' on the command line.\\
&\textit{Accepted values:}\quad \hyperref[filehandler:labelled Petri net]{labelled Petri net (.lpn)}, \hyperref[filehandler:stochastic deterministic finite automaton]{stochastic deterministic finite automaton (.sdfa)}, \hyperref[filehandler:directly follows model]{directly follows model (.dfm)}, \hyperref[filehandler:deterministic finite automaton]{deterministic finite automaton (.dfa)}, \hyperref[filehandler:process tree]{process tree (.ptree)}, \hyperref[filehandler:LoLa Petri net]{LoLa Petri net (.lola)}, \hyperref[filehandler:process tree markup language]{process tree markup language (.ptml)}, \hyperref[filehandler:directly follows graph]{directly follows graph (.dfg)}, \hyperref[filehandler:Petri net markup language]{Petri net markup language (.pnml)}, \hyperref[filehandler:stochastic labelled Petri net]{stochastic labelled Petri net (.slpn)}, \hyperref[filehandler:stochastic process tree]{stochastic process tree (.sptree)} and \hyperref[filehandler:stochastic directly follows model]{stochastic directly follows model (.sdfm)}.\\
\texttt{-o} or \texttt{--output} <\texttt{FILE}> &
The file to which the results must be written. Based on the file extension, Ebi will output either a portable document format (.pdf -- Section~\ref{filehandler:portable document format}) or a scalable vector graphics (.svg -- Section~\ref{filehandler:scalable vector graphics}).
If the parameter is not given, the results will be written to STDOUT as a scalable vector graphics (.svg).\\
&\textit{Mandatory:} \quad no\\
\texttt{-a} or \texttt{--approximate} & Use approximate arithmetic instead of exact arithmetic.\\
&\textit{Mandatory:}\quad no\\
\bottomrule
\end{tabularx}
\noindent Output: scalable vector graphics, which can be written as a portable document format (.pdf -- Section~\ref{filehandler:portable document format}) or a scalable vector graphics (.svg -- Section~\ref{filehandler:scalable vector graphics}).
\\This command is not available in Java and ProM.
\subsection{\texttt{Ebi visualise text}}
\label{command:Ebi visualise text}
Alias: \texttt{Ebi vis txt}.\\
Visualise a file as text.\\
\begin{tabularx}{\linewidth}{lX}
\toprule
Parameter \\\midrule
<\texttt{FILE}>&Any file that can be visualised textually.\\
&\textit{Mandatory:} \quad yes, though it can be given on STDIN by giving a `-' on the command line.\\
&\textit{Accepted values:}\quad \hyperref[filehandler:compressed event log]{compressed event log (.xes.gz)}, \hyperref[filehandler:event log]{event log (.xes)}, \hyperref[filehandler:stochastic process tree]{stochastic process tree (.sptree)}, \hyperref[filehandler:deterministic finite automaton]{deterministic finite automaton (.dfa)}, \hyperref[filehandler:language of alignments]{language of alignments (.ali)}, \hyperref[filehandler:stochastic labelled Petri net]{stochastic labelled Petri net (.slpn)}, \hyperref[filehandler:scalable vector graphics]{scalable vector graphics (.svg)}, \hyperref[filehandler:Petri net markup language]{Petri net markup language (.pnml)}, \hyperref[filehandler:directly follows model]{directly follows model (.dfm)}, \hyperref[filehandler:finite stochastic language]{finite stochastic language (.slang)}, \hyperref[filehandler:LoLa Petri net]{LoLa Petri net (.lola)}, \hyperref[filehandler:process tree]{process tree (.ptree)}, \hyperref[filehandler:executions]{executions (.exs)}, \hyperref[filehandler:finite language]{finite language (.lang)}, \hyperref[filehandler:stochastic deterministic finite automaton]{stochastic deterministic finite automaton (.sdfa)}, \hyperref[filehandler:portable document format]{portable document format (.pdf)}, \hyperref[filehandler:stochastic language of alignments]{stochastic language of alignments (.sali)}, \hyperref[filehandler:stochastic directly follows model]{stochastic directly follows model (.sdfm)}, \hyperref[filehandler:directly follows graph]{directly follows graph (.dfg)}, \hyperref[filehandler:process tree markup language]{process tree markup language (.ptml)} and \hyperref[filehandler:labelled Petri net]{labelled Petri net (.lpn)}.\\
\texttt{-o} or \texttt{--output} <\texttt{FILE}> &
The text file to which the result must be written. If the parameter is not given, the results will be written to STDOUT.\\
&\textit{Mandatory:} \quad no\\
\bottomrule
\end{tabularx}
\noindent Output: text, which can be written as  text.
\\This command is available in Java and ProM.
}
\long\def\ebifilehandlers{
\subsection{Compressed event log (.xes.gz)}
\label{filehandler:compressed event log}
Import as objects: event log.
\\Import as traits: activities, iterable language, finite language, finite stochastic language, queriable stochastic language, iterable stochastic language, event log, stochastic deterministic semantics, stochastic semantics, semantics.
\\Input to commands: \\\null\qquad\hyperref[command:Ebi analyse all-traces]{\texttt{Ebi analyse all-traces}} (Section~\ref{command:Ebi analyse all-traces})\\\null\qquad\hyperref[command:Ebi analyse completeness]{\texttt{Ebi analyse completeness}} (Section~\ref{command:Ebi analyse completeness})\\\null\qquad\hyperref[command:Ebi analyse coverage]{\texttt{Ebi analyse coverage}} (Section~\ref{command:Ebi analyse coverage})\\\null\qquad\hyperref[command:Ebi analyse medoid]{\texttt{Ebi analyse medoid}} (Section~\ref{command:Ebi analyse medoid})\\\null\qquad\hyperref[command:Ebi analyse minimum-probability-traces]{\texttt{Ebi analyse minimum-probability-traces}} (Section~\ref{command:Ebi analyse minimum-probability-traces})\\\null\qquad\hyperref[command:Ebi analyse mode]{\texttt{Ebi analyse mode}} (Section~\ref{command:Ebi analyse mode})\\\null\qquad\hyperref[command:Ebi analyse most-likely-traces]{\texttt{Ebi analyse most-likely-traces}} (Section~\ref{command:Ebi analyse most-likely-traces})\\\null\qquad\hyperref[command:Ebi analyse variety]{\texttt{Ebi analyse variety}} (Section~\ref{command:Ebi analyse variety})\\\null\qquad\hyperref[command:Ebi analyse-non-stochastic any-traces]{\texttt{Ebi analyse-non-stochastic any-traces}} (Section~\ref{command:Ebi analyse-non-stochastic any-traces})\\\null\qquad\hyperref[command:Ebi analyse-non-stochastic bounded]{\texttt{Ebi analyse-non-stochastic bounded}} (Section~\ref{command:Ebi analyse-non-stochastic bounded})\\\null\qquad\hyperref[command:Ebi analyse-non-stochastic cluster]{\texttt{Ebi analyse-non-stochastic cluster}} (Section~\ref{command:Ebi analyse-non-stochastic cluster})\\\null\qquad\hyperref[command:Ebi analyse-non-stochastic executions]{\texttt{Ebi analyse-non-stochastic executions}} (Section~\ref{command:Ebi analyse-non-stochastic executions})\\\null\qquad\hyperref[command:Ebi analyse-non-stochastic infinitely-many-traces]{\texttt{Ebi analyse-non-stochastic infinitely-many-traces}} (Section~\ref{command:Ebi analyse-non-stochastic infinitely-many-traces})\\\null\qquad\hyperref[command:Ebi analyse-non-stochastic medoid]{\texttt{Ebi analyse-non-stochastic medoid}} (Section~\ref{command:Ebi analyse-non-stochastic medoid})\\\null\qquad\hyperref[command:Ebi association all-trace-attributes]{\texttt{Ebi association all-trace-attributes}} (Section~\ref{command:Ebi association all-trace-attributes})\\\null\qquad\hyperref[command:Ebi association trace-attribute]{\texttt{Ebi association trace-attribute}} (Section~\ref{command:Ebi association trace-attribute})\\\null\qquad\hyperref[command:Ebi conformance earth-movers-stochastic-conformance]{\texttt{Ebi conformance earth-movers-stochastic-conformance}} (Section~\ref{command:Ebi conformance earth-movers-stochastic-conformance})\\\null\qquad\hyperref[command:Ebi conformance earth-movers-stochastic-conformance-sample]{\texttt{Ebi conformance earth-movers-stochastic-conformance-sample}} (Section~\ref{command:Ebi conformance earth-movers-stochastic-conformance-sample})\\\null\qquad\hyperref[command:Ebi conformance entropic-relevance]{\texttt{Ebi conformance entropic-relevance}} (Section~\ref{command:Ebi conformance entropic-relevance})\\\null\qquad\hyperref[command:Ebi conformance jensen-shannon]{\texttt{Ebi conformance jensen-shannon}} (Section~\ref{command:Ebi conformance jensen-shannon})\\\null\qquad\hyperref[command:Ebi conformance jensen-shannon-sample]{\texttt{Ebi conformance jensen-shannon-sample}} (Section~\ref{command:Ebi conformance jensen-shannon-sample})\\\null\qquad\hyperref[command:Ebi conformance unit-earth-movers-stochastic-conformance]{\texttt{Ebi conformance unit-earth-movers-stochastic-conformance}} (Section~\ref{command:Ebi conformance unit-earth-movers-stochastic-conformance})\\\null\qquad\hyperref[command:Ebi conformance-non-stochastic alignments]{\texttt{Ebi conformance-non-stochastic alignments}} (Section~\ref{command:Ebi conformance-non-stochastic alignments})\\\null\qquad\hyperref[command:Ebi conformance-non-stochastic escaping-edges-precision]{\texttt{Ebi conformance-non-stochastic escaping-edges-precision}} (Section~\ref{command:Ebi conformance-non-stochastic escaping-edges-precision})\\\null\qquad\hyperref[command:Ebi conformance-non-stochastic set-alignments]{\texttt{Ebi conformance-non-stochastic set-alignments}} (Section~\ref{command:Ebi conformance-non-stochastic set-alignments})\\\null\qquad\hyperref[command:Ebi discover alignments]{\texttt{Ebi discover alignments}} (Section~\ref{command:Ebi discover alignments})\\\null\qquad\hyperref[command:Ebi discover directly-follows-graph]{\texttt{Ebi discover directly-follows-graph}} (Section~\ref{command:Ebi discover directly-follows-graph})\\\null\qquad\hyperref[command:Ebi discover occurrence labelled-petri-net]{\texttt{Ebi discover occurrence labelled-petri-net}} (Section~\ref{command:Ebi discover occurrence labelled-petri-net})\\\null\qquad\hyperref[command:Ebi discover occurrence process-tree]{\texttt{Ebi discover occurrence process-tree}} (Section~\ref{command:Ebi discover occurrence process-tree})\\\null\qquad\hyperref[command:Ebi discover-non-stochastic flower deterministic-finite-automaton]{\texttt{Ebi discover-non-stochastic flower deterministic-finite-automaton}} (Section~\ref{command:Ebi discover-non-stochastic flower deterministic-finite-automaton})\\\null\qquad\hyperref[command:Ebi discover-non-stochastic flower process-tree]{\texttt{Ebi discover-non-stochastic flower process-tree}} (Section~\ref{command:Ebi discover-non-stochastic flower process-tree})\\\null\qquad\hyperref[command:Ebi discover-non-stochastic prefix-tree deterministic-finite-automaton]{\texttt{Ebi discover-non-stochastic prefix-tree deterministic-finite-automaton}} (Section~\ref{command:Ebi discover-non-stochastic prefix-tree deterministic-finite-automaton})\\\null\qquad\hyperref[command:Ebi discover-non-stochastic prefix-tree process-tree]{\texttt{Ebi discover-non-stochastic prefix-tree process-tree}} (Section~\ref{command:Ebi discover-non-stochastic prefix-tree process-tree})\\\null\qquad\hyperref[command:Ebi information]{\texttt{Ebi information}} (Section~\ref{command:Ebi information})\\\null\qquad\hyperref[command:Ebi probability explain-trace]{\texttt{Ebi probability explain-trace}} (Section~\ref{command:Ebi probability explain-trace})\\\null\qquad\hyperref[command:Ebi probability log]{\texttt{Ebi probability log}} (Section~\ref{command:Ebi probability log})\\\null\qquad\hyperref[command:Ebi probability trace]{\texttt{Ebi probability trace}} (Section~\ref{command:Ebi probability trace})\\\null\qquad\hyperref[command:Ebi sample]{\texttt{Ebi sample}} (Section~\ref{command:Ebi sample})\\\null\qquad\hyperref[command:Ebi test bootstrap-test]{\texttt{Ebi test bootstrap-test}} (Section~\ref{command:Ebi test bootstrap-test})\\\null\qquad\hyperref[command:Ebi test log-categorical-attribute]{\texttt{Ebi test log-categorical-attribute}} (Section~\ref{command:Ebi test log-categorical-attribute})\\\null\qquad\hyperref[command:Ebi validate]{\texttt{Ebi validate}} (Section~\ref{command:Ebi validate})\\\null\qquad\hyperref[command:Ebi visualise text]{\texttt{Ebi visualise text}} (Section~\ref{command:Ebi visualise text}).
\\Output of commands: none.
\\A compressed event log (.xes.gz) can be neither imported nor exported between Ebi, and ProM and Java.
\\File format specification:
A compressed event log is a gzipped event log file in the IEEE XES format~\cite{DBLP:journals/cim/AcamporaVSAGV17}.
Parsing is performed by the Rust4PM crate~\cite{DBLP:conf/bpm/KustersA24}.
\clearpage
\subsection{Directly follows graph (.dfg)}
\label{filehandler:directly follows graph}
Import as objects: directly follows graph, stochastic directly follows model, directly follows model, labelled Petri net, stochastic labelled Petri net.
\\Import as traits: activities, semantics, stochastic semantics, stochastic deterministic semantics, graphable.
\\Input to commands: \\\null\qquad\hyperref[command:Ebi analyse all-traces]{\texttt{Ebi analyse all-traces}} (Section~\ref{command:Ebi analyse all-traces})\\\null\qquad\hyperref[command:Ebi analyse coverage]{\texttt{Ebi analyse coverage}} (Section~\ref{command:Ebi analyse coverage})\\\null\qquad\hyperref[command:Ebi analyse directly-follows-edge-difference]{\texttt{Ebi analyse directly-follows-edge-difference}} (Section~\ref{command:Ebi analyse directly-follows-edge-difference})\\\null\qquad\hyperref[command:Ebi analyse minimum-probability-traces]{\texttt{Ebi analyse minimum-probability-traces}} (Section~\ref{command:Ebi analyse minimum-probability-traces})\\\null\qquad\hyperref[command:Ebi analyse mode]{\texttt{Ebi analyse mode}} (Section~\ref{command:Ebi analyse mode})\\\null\qquad\hyperref[command:Ebi analyse most-likely-traces]{\texttt{Ebi analyse most-likely-traces}} (Section~\ref{command:Ebi analyse most-likely-traces})\\\null\qquad\hyperref[command:Ebi analyse-non-stochastic any-traces]{\texttt{Ebi analyse-non-stochastic any-traces}} (Section~\ref{command:Ebi analyse-non-stochastic any-traces})\\\null\qquad\hyperref[command:Ebi analyse-non-stochastic bounded]{\texttt{Ebi analyse-non-stochastic bounded}} (Section~\ref{command:Ebi analyse-non-stochastic bounded})\\\null\qquad\hyperref[command:Ebi analyse-non-stochastic executions]{\texttt{Ebi analyse-non-stochastic executions}} (Section~\ref{command:Ebi analyse-non-stochastic executions})\\\null\qquad\hyperref[command:Ebi analyse-non-stochastic infinitely-many-traces]{\texttt{Ebi analyse-non-stochastic infinitely-many-traces}} (Section~\ref{command:Ebi analyse-non-stochastic infinitely-many-traces})\\\null\qquad\hyperref[command:Ebi conformance earth-movers-stochastic-conformance-sample]{\texttt{Ebi conformance earth-movers-stochastic-conformance-sample}} (Section~\ref{command:Ebi conformance earth-movers-stochastic-conformance-sample})\\\null\qquad\hyperref[command:Ebi conformance jensen-shannon-sample]{\texttt{Ebi conformance jensen-shannon-sample}} (Section~\ref{command:Ebi conformance jensen-shannon-sample})\\\null\qquad\hyperref[command:Ebi conformance-non-stochastic alignments]{\texttt{Ebi conformance-non-stochastic alignments}} (Section~\ref{command:Ebi conformance-non-stochastic alignments})\\\null\qquad\hyperref[command:Ebi conformance-non-stochastic escaping-edges-precision]{\texttt{Ebi conformance-non-stochastic escaping-edges-precision}} (Section~\ref{command:Ebi conformance-non-stochastic escaping-edges-precision})\\\null\qquad\hyperref[command:Ebi conformance-non-stochastic set-alignments]{\texttt{Ebi conformance-non-stochastic set-alignments}} (Section~\ref{command:Ebi conformance-non-stochastic set-alignments})\\\null\qquad\hyperref[command:Ebi convert labelled-petri-net]{\texttt{Ebi convert labelled-petri-net}} (Section~\ref{command:Ebi convert labelled-petri-net})\\\null\qquad\hyperref[command:Ebi discover alignments]{\texttt{Ebi discover alignments}} (Section~\ref{command:Ebi discover alignments})\\\null\qquad\hyperref[command:Ebi discover occurrence labelled-petri-net]{\texttt{Ebi discover occurrence labelled-petri-net}} (Section~\ref{command:Ebi discover occurrence labelled-petri-net})\\\null\qquad\hyperref[command:Ebi discover uniform labelled-petri-net]{\texttt{Ebi discover uniform labelled-petri-net}} (Section~\ref{command:Ebi discover uniform labelled-petri-net})\\\null\qquad\hyperref[command:Ebi discover-non-stochastic flower deterministic-finite-automaton]{\texttt{Ebi discover-non-stochastic flower deterministic-finite-automaton}} (Section~\ref{command:Ebi discover-non-stochastic flower deterministic-finite-automaton})\\\null\qquad\hyperref[command:Ebi discover-non-stochastic flower process-tree]{\texttt{Ebi discover-non-stochastic flower process-tree}} (Section~\ref{command:Ebi discover-non-stochastic flower process-tree})\\\null\qquad\hyperref[command:Ebi information]{\texttt{Ebi information}} (Section~\ref{command:Ebi information})\\\null\qquad\hyperref[command:Ebi probability explain-trace]{\texttt{Ebi probability explain-trace}} (Section~\ref{command:Ebi probability explain-trace})\\\null\qquad\hyperref[command:Ebi sample]{\texttt{Ebi sample}} (Section~\ref{command:Ebi sample})\\\null\qquad\hyperref[command:Ebi validate]{\texttt{Ebi validate}} (Section~\ref{command:Ebi validate})\\\null\qquad\hyperref[command:Ebi visualise graph]{\texttt{Ebi visualise graph}} (Section~\ref{command:Ebi visualise graph})\\\null\qquad\hyperref[command:Ebi visualise text]{\texttt{Ebi visualise text}} (Section~\ref{command:Ebi visualise text}).
\\Output of commands: \\\null\qquad\hyperref[command:Ebi discover directly-follows-graph]{\texttt{Ebi discover directly-follows-graph}} (Section~\ref{command:Ebi discover directly-follows-graph}).
\\A directly follows graph (.dfg) can be neither imported nor exported between Ebi, and ProM and Java.
\\File format specification:
A directly follows graph is a JSON structure.
    
    The following table gives an overview of several file types and their features:
    \begin{center}
    \begin{tabular}{lll}
        \toprule
        File type & stochastic & multiple nodes with the same label & file syntax \\
        \midrule
        \hyperref[filehandler:directly follows graph]{directly follows graph (.dfg)} & yes & no & JSON \\
        \hyperref[filehandler:directly follows model]{directly follows model (.dfm)} & no & yes & line-based \\
        \hyperref[filehandler:stochastic directly follows model]{stochastic directly follows model (.sdfm)} & yes & yes & line-based \\
        \bottomrule
    \end{tabular}
    \end{center}
\clearpage
\subsection{Deterministic finite automaton (.dfa)}
\label{filehandler:deterministic finite automaton}
Import as objects: deterministic finite automaton, labelled Petri net.
\\Import as traits: activities, semantics, graphable.
\\Input to commands: \\\null\qquad\hyperref[command:Ebi analyse-non-stochastic any-traces]{\texttt{Ebi analyse-non-stochastic any-traces}} (Section~\ref{command:Ebi analyse-non-stochastic any-traces})\\\null\qquad\hyperref[command:Ebi analyse-non-stochastic bounded]{\texttt{Ebi analyse-non-stochastic bounded}} (Section~\ref{command:Ebi analyse-non-stochastic bounded})\\\null\qquad\hyperref[command:Ebi analyse-non-stochastic executions]{\texttt{Ebi analyse-non-stochastic executions}} (Section~\ref{command:Ebi analyse-non-stochastic executions})\\\null\qquad\hyperref[command:Ebi analyse-non-stochastic infinitely-many-traces]{\texttt{Ebi analyse-non-stochastic infinitely-many-traces}} (Section~\ref{command:Ebi analyse-non-stochastic infinitely-many-traces})\\\null\qquad\hyperref[command:Ebi conformance-non-stochastic alignments]{\texttt{Ebi conformance-non-stochastic alignments}} (Section~\ref{command:Ebi conformance-non-stochastic alignments})\\\null\qquad\hyperref[command:Ebi conformance-non-stochastic escaping-edges-precision]{\texttt{Ebi conformance-non-stochastic escaping-edges-precision}} (Section~\ref{command:Ebi conformance-non-stochastic escaping-edges-precision})\\\null\qquad\hyperref[command:Ebi conformance-non-stochastic set-alignments]{\texttt{Ebi conformance-non-stochastic set-alignments}} (Section~\ref{command:Ebi conformance-non-stochastic set-alignments})\\\null\qquad\hyperref[command:Ebi convert labelled-petri-net]{\texttt{Ebi convert labelled-petri-net}} (Section~\ref{command:Ebi convert labelled-petri-net})\\\null\qquad\hyperref[command:Ebi discover alignments]{\texttt{Ebi discover alignments}} (Section~\ref{command:Ebi discover alignments})\\\null\qquad\hyperref[command:Ebi discover occurrence labelled-petri-net]{\texttt{Ebi discover occurrence labelled-petri-net}} (Section~\ref{command:Ebi discover occurrence labelled-petri-net})\\\null\qquad\hyperref[command:Ebi discover uniform labelled-petri-net]{\texttt{Ebi discover uniform labelled-petri-net}} (Section~\ref{command:Ebi discover uniform labelled-petri-net})\\\null\qquad\hyperref[command:Ebi discover-non-stochastic flower deterministic-finite-automaton]{\texttt{Ebi discover-non-stochastic flower deterministic-finite-automaton}} (Section~\ref{command:Ebi discover-non-stochastic flower deterministic-finite-automaton})\\\null\qquad\hyperref[command:Ebi discover-non-stochastic flower process-tree]{\texttt{Ebi discover-non-stochastic flower process-tree}} (Section~\ref{command:Ebi discover-non-stochastic flower process-tree})\\\null\qquad\hyperref[command:Ebi information]{\texttt{Ebi information}} (Section~\ref{command:Ebi information})\\\null\qquad\hyperref[command:Ebi validate]{\texttt{Ebi validate}} (Section~\ref{command:Ebi validate})\\\null\qquad\hyperref[command:Ebi visualise graph]{\texttt{Ebi visualise graph}} (Section~\ref{command:Ebi visualise graph})\\\null\qquad\hyperref[command:Ebi visualise text]{\texttt{Ebi visualise text}} (Section~\ref{command:Ebi visualise text}).
\\Output of commands: \\\null\qquad\hyperref[command:Ebi analyse all-traces]{\texttt{Ebi analyse all-traces}} (Section~\ref{command:Ebi analyse all-traces})\\\null\qquad\hyperref[command:Ebi analyse coverage]{\texttt{Ebi analyse coverage}} (Section~\ref{command:Ebi analyse coverage})\\\null\qquad\hyperref[command:Ebi analyse medoid]{\texttt{Ebi analyse medoid}} (Section~\ref{command:Ebi analyse medoid})\\\null\qquad\hyperref[command:Ebi analyse minimum-probability-traces]{\texttt{Ebi analyse minimum-probability-traces}} (Section~\ref{command:Ebi analyse minimum-probability-traces})\\\null\qquad\hyperref[command:Ebi analyse mode]{\texttt{Ebi analyse mode}} (Section~\ref{command:Ebi analyse mode})\\\null\qquad\hyperref[command:Ebi analyse most-likely-traces]{\texttt{Ebi analyse most-likely-traces}} (Section~\ref{command:Ebi analyse most-likely-traces})\\\null\qquad\hyperref[command:Ebi analyse-non-stochastic cluster]{\texttt{Ebi analyse-non-stochastic cluster}} (Section~\ref{command:Ebi analyse-non-stochastic cluster})\\\null\qquad\hyperref[command:Ebi analyse-non-stochastic medoid]{\texttt{Ebi analyse-non-stochastic medoid}} (Section~\ref{command:Ebi analyse-non-stochastic medoid})\\\null\qquad\hyperref[command:Ebi convert finite-stochastic-language]{\texttt{Ebi convert finite-stochastic-language}} (Section~\ref{command:Ebi convert finite-stochastic-language})\\\null\qquad\hyperref[command:Ebi convert stochastic-finite-deterministic-automaton]{\texttt{Ebi convert stochastic-finite-deterministic-automaton}} (Section~\ref{command:Ebi convert stochastic-finite-deterministic-automaton})\\\null\qquad\hyperref[command:Ebi discover-non-stochastic flower deterministic-finite-automaton]{\texttt{Ebi discover-non-stochastic flower deterministic-finite-automaton}} (Section~\ref{command:Ebi discover-non-stochastic flower deterministic-finite-automaton})\\\null\qquad\hyperref[command:Ebi discover-non-stochastic prefix-tree deterministic-finite-automaton]{\texttt{Ebi discover-non-stochastic prefix-tree deterministic-finite-automaton}} (Section~\ref{command:Ebi discover-non-stochastic prefix-tree deterministic-finite-automaton})\\\null\qquad\hyperref[command:Ebi sample]{\texttt{Ebi sample}} (Section~\ref{command:Ebi sample}).
\\A deterministic finite automaton (.dfa) can be neither imported nor exported between Ebi, and ProM and Java.
\\File format specification:
A deterministic finite automaton is a JSON structure with the top level being an object.
    This object contains the following key-value pairs:
    \begin{itemize}
    \item \texttt{initialState} being the index of the initial state. This field is optional: if omitted, the DFA has an empty language.
    \item \texttt{finalStates} being a list of indices of the final states.
    A final state is not necessarily a deadlock state.
    \item \texttt{transitions} being a list of transitions. 
    Each transition is an object with \texttt{from} being the source state index of the transition, \texttt{to} being the target state index of the transition, and 	exttt{{label}} being the activity of the transition. 
    Silent transitions are not supported.
    The file format supports deadlocks and livelocks.
    \end{itemize}
    For instance:
    \lstinputlisting[language=json, style=boxed]{../testfiles/aa-ab-ba.dfa}
\clearpage
\subsection{Directly follows model (.dfm)}
\label{filehandler:directly follows model}
Import as objects: directly follows model, labelled Petri net.
\\Import as traits: activities, semantics, graphable.
\\Input to commands: \\\null\qquad\hyperref[command:Ebi analyse-non-stochastic any-traces]{\texttt{Ebi analyse-non-stochastic any-traces}} (Section~\ref{command:Ebi analyse-non-stochastic any-traces})\\\null\qquad\hyperref[command:Ebi analyse-non-stochastic bounded]{\texttt{Ebi analyse-non-stochastic bounded}} (Section~\ref{command:Ebi analyse-non-stochastic bounded})\\\null\qquad\hyperref[command:Ebi analyse-non-stochastic executions]{\texttt{Ebi analyse-non-stochastic executions}} (Section~\ref{command:Ebi analyse-non-stochastic executions})\\\null\qquad\hyperref[command:Ebi analyse-non-stochastic infinitely-many-traces]{\texttt{Ebi analyse-non-stochastic infinitely-many-traces}} (Section~\ref{command:Ebi analyse-non-stochastic infinitely-many-traces})\\\null\qquad\hyperref[command:Ebi conformance-non-stochastic alignments]{\texttt{Ebi conformance-non-stochastic alignments}} (Section~\ref{command:Ebi conformance-non-stochastic alignments})\\\null\qquad\hyperref[command:Ebi conformance-non-stochastic escaping-edges-precision]{\texttt{Ebi conformance-non-stochastic escaping-edges-precision}} (Section~\ref{command:Ebi conformance-non-stochastic escaping-edges-precision})\\\null\qquad\hyperref[command:Ebi conformance-non-stochastic set-alignments]{\texttt{Ebi conformance-non-stochastic set-alignments}} (Section~\ref{command:Ebi conformance-non-stochastic set-alignments})\\\null\qquad\hyperref[command:Ebi convert labelled-petri-net]{\texttt{Ebi convert labelled-petri-net}} (Section~\ref{command:Ebi convert labelled-petri-net})\\\null\qquad\hyperref[command:Ebi discover alignments]{\texttt{Ebi discover alignments}} (Section~\ref{command:Ebi discover alignments})\\\null\qquad\hyperref[command:Ebi discover occurrence labelled-petri-net]{\texttt{Ebi discover occurrence labelled-petri-net}} (Section~\ref{command:Ebi discover occurrence labelled-petri-net})\\\null\qquad\hyperref[command:Ebi discover uniform labelled-petri-net]{\texttt{Ebi discover uniform labelled-petri-net}} (Section~\ref{command:Ebi discover uniform labelled-petri-net})\\\null\qquad\hyperref[command:Ebi discover-non-stochastic flower deterministic-finite-automaton]{\texttt{Ebi discover-non-stochastic flower deterministic-finite-automaton}} (Section~\ref{command:Ebi discover-non-stochastic flower deterministic-finite-automaton})\\\null\qquad\hyperref[command:Ebi discover-non-stochastic flower process-tree]{\texttt{Ebi discover-non-stochastic flower process-tree}} (Section~\ref{command:Ebi discover-non-stochastic flower process-tree})\\\null\qquad\hyperref[command:Ebi information]{\texttt{Ebi information}} (Section~\ref{command:Ebi information})\\\null\qquad\hyperref[command:Ebi validate]{\texttt{Ebi validate}} (Section~\ref{command:Ebi validate})\\\null\qquad\hyperref[command:Ebi visualise graph]{\texttt{Ebi visualise graph}} (Section~\ref{command:Ebi visualise graph})\\\null\qquad\hyperref[command:Ebi visualise text]{\texttt{Ebi visualise text}} (Section~\ref{command:Ebi visualise text}).
\\Output of commands: \\\null\qquad\hyperref[command:Ebi discover directly-follows-graph]{\texttt{Ebi discover directly-follows-graph}} (Section~\ref{command:Ebi discover directly-follows-graph}).
\\A directly follows model (.dfm) can be neither imported nor exported between Ebi, and ProM and Java.
\\File format specification:
A directly follows model is a line-based structure. Lines starting with a \# are ignored.
    This first line is exactly `directly follows model'.\
    The second line is a boolean indicating whether the model supports empty traces.\
    The third line is the number of activities in the model.\
    The following lines each contain an activity. Duplicated labels are accepted.\
    The next line contains the number of start activities, followed by, for each start activity, a line with the index of the start activity.\
    The next line contains the number of end activities, followed by, for each end activity, a line with the index of the end activity.\
    The next line contains the number of edges, followed by, for each edge, a line with first the index of the source activity, then the `>` symbol, then the index of the target activity.
    
    For instance:
    \lstinputlisting[language=ebilines, style=boxed]{../testfiles/a-b_star.dfm}
    
    The following table gives an overview of several file types and their features:
    \begin{center}
    \begin{tabular}{lll}
        \toprule
        File type & stochastic & multiple nodes with the same label & file syntax \\
        \midrule
        \hyperref[filehandler:directly follows graph]{directly follows graph (.dfg)} & yes & no & JSON \\
        \hyperref[filehandler:directly follows model]{directly follows model (.dfm)} & no & yes & line-based \\
        \hyperref[filehandler:stochastic directly follows model]{stochastic directly follows model (.sdfm)} & yes & yes & line-based \\
        \bottomrule
    \end{tabular}
    \end{center}
\clearpage
\subsection{Stochastic directly follows model (.sdfm)}
\label{filehandler:stochastic directly follows model}
Import as objects: stochastic directly follows model, directly follows model, labelled Petri net, stochastic labelled Petri net.
\\Import as traits: activities, semantics, stochastic semantics, stochastic deterministic semantics, graphable.
\\Input to commands: \\\null\qquad\hyperref[command:Ebi analyse all-traces]{\texttt{Ebi analyse all-traces}} (Section~\ref{command:Ebi analyse all-traces})\\\null\qquad\hyperref[command:Ebi analyse coverage]{\texttt{Ebi analyse coverage}} (Section~\ref{command:Ebi analyse coverage})\\\null\qquad\hyperref[command:Ebi analyse minimum-probability-traces]{\texttt{Ebi analyse minimum-probability-traces}} (Section~\ref{command:Ebi analyse minimum-probability-traces})\\\null\qquad\hyperref[command:Ebi analyse mode]{\texttt{Ebi analyse mode}} (Section~\ref{command:Ebi analyse mode})\\\null\qquad\hyperref[command:Ebi analyse most-likely-traces]{\texttt{Ebi analyse most-likely-traces}} (Section~\ref{command:Ebi analyse most-likely-traces})\\\null\qquad\hyperref[command:Ebi analyse-non-stochastic any-traces]{\texttt{Ebi analyse-non-stochastic any-traces}} (Section~\ref{command:Ebi analyse-non-stochastic any-traces})\\\null\qquad\hyperref[command:Ebi analyse-non-stochastic bounded]{\texttt{Ebi analyse-non-stochastic bounded}} (Section~\ref{command:Ebi analyse-non-stochastic bounded})\\\null\qquad\hyperref[command:Ebi analyse-non-stochastic executions]{\texttt{Ebi analyse-non-stochastic executions}} (Section~\ref{command:Ebi analyse-non-stochastic executions})\\\null\qquad\hyperref[command:Ebi analyse-non-stochastic infinitely-many-traces]{\texttt{Ebi analyse-non-stochastic infinitely-many-traces}} (Section~\ref{command:Ebi analyse-non-stochastic infinitely-many-traces})\\\null\qquad\hyperref[command:Ebi conformance earth-movers-stochastic-conformance-sample]{\texttt{Ebi conformance earth-movers-stochastic-conformance-sample}} (Section~\ref{command:Ebi conformance earth-movers-stochastic-conformance-sample})\\\null\qquad\hyperref[command:Ebi conformance jensen-shannon-sample]{\texttt{Ebi conformance jensen-shannon-sample}} (Section~\ref{command:Ebi conformance jensen-shannon-sample})\\\null\qquad\hyperref[command:Ebi conformance-non-stochastic alignments]{\texttt{Ebi conformance-non-stochastic alignments}} (Section~\ref{command:Ebi conformance-non-stochastic alignments})\\\null\qquad\hyperref[command:Ebi conformance-non-stochastic escaping-edges-precision]{\texttt{Ebi conformance-non-stochastic escaping-edges-precision}} (Section~\ref{command:Ebi conformance-non-stochastic escaping-edges-precision})\\\null\qquad\hyperref[command:Ebi conformance-non-stochastic set-alignments]{\texttt{Ebi conformance-non-stochastic set-alignments}} (Section~\ref{command:Ebi conformance-non-stochastic set-alignments})\\\null\qquad\hyperref[command:Ebi convert labelled-petri-net]{\texttt{Ebi convert labelled-petri-net}} (Section~\ref{command:Ebi convert labelled-petri-net})\\\null\qquad\hyperref[command:Ebi discover alignments]{\texttt{Ebi discover alignments}} (Section~\ref{command:Ebi discover alignments})\\\null\qquad\hyperref[command:Ebi discover occurrence labelled-petri-net]{\texttt{Ebi discover occurrence labelled-petri-net}} (Section~\ref{command:Ebi discover occurrence labelled-petri-net})\\\null\qquad\hyperref[command:Ebi discover uniform labelled-petri-net]{\texttt{Ebi discover uniform labelled-petri-net}} (Section~\ref{command:Ebi discover uniform labelled-petri-net})\\\null\qquad\hyperref[command:Ebi discover-non-stochastic flower deterministic-finite-automaton]{\texttt{Ebi discover-non-stochastic flower deterministic-finite-automaton}} (Section~\ref{command:Ebi discover-non-stochastic flower deterministic-finite-automaton})\\\null\qquad\hyperref[command:Ebi discover-non-stochastic flower process-tree]{\texttt{Ebi discover-non-stochastic flower process-tree}} (Section~\ref{command:Ebi discover-non-stochastic flower process-tree})\\\null\qquad\hyperref[command:Ebi information]{\texttt{Ebi information}} (Section~\ref{command:Ebi information})\\\null\qquad\hyperref[command:Ebi probability explain-trace]{\texttt{Ebi probability explain-trace}} (Section~\ref{command:Ebi probability explain-trace})\\\null\qquad\hyperref[command:Ebi sample]{\texttt{Ebi sample}} (Section~\ref{command:Ebi sample})\\\null\qquad\hyperref[command:Ebi validate]{\texttt{Ebi validate}} (Section~\ref{command:Ebi validate})\\\null\qquad\hyperref[command:Ebi visualise graph]{\texttt{Ebi visualise graph}} (Section~\ref{command:Ebi visualise graph})\\\null\qquad\hyperref[command:Ebi visualise text]{\texttt{Ebi visualise text}} (Section~\ref{command:Ebi visualise text}).
\\Output of commands: \\\null\qquad\hyperref[command:Ebi discover directly-follows-graph]{\texttt{Ebi discover directly-follows-graph}} (Section~\ref{command:Ebi discover directly-follows-graph}).
\\A stochastic directly follows model (.sdfm) can be neither imported nor exported between Ebi, and ProM and Java.
\\File format specification:
A stochstic directly follows model is a line-based structure. Lines starting with a \# are ignored.
    This first line is exactly `directly follows model'.\
    The second line is a boolean indicating whether the model supports empty traces.\
    The third line is the number of activities in the model.\
    The following lines each contain an activity. Duplicated labels are accepted.\
    The next line contains the number of start activities, followed by, for each start activity, a line with the index of the start activity, followed by a `w` and the weight of the start activity.\
    The next line contains the number of end activities, followed by, for each end activity, a line with the index of the end activity, followed by a `w` and the weight of the end activity.\
    The next line contains the number of edges, followed by, for each edge, a line with first the index of the source activity, then the `>` symbol, then the index of the target activity, then a `w`, and then the weight of the transition.
    
    For instance:
    \lstinputlisting[language=ebilines, style=boxed]{../testfiles/aa-ab-ba.sdfm}
    
    The following table gives an overview of several file types and their features:
    \begin{center}
    \begin{tabular}{lll}
        \toprule
        File type & stochastic & multiple nodes with the same label & file syntax \\
        \midrule
        \hyperref[filehandler:directly follows graph]{directly follows graph (.dfg)} & yes & no & JSON \\
        \hyperref[filehandler:directly follows model]{directly follows model (.dfm)} & no & yes & line-based \\
        \hyperref[filehandler:stochastic directly follows model]{stochastic directly follows model (.sdfm)} & yes & yes & line-based \\
        \bottomrule
    \end{tabular}
    \end{center}
\clearpage
\subsection{Event log (.xes)}
\label{filehandler:event log}
Import as objects: event log, finite stochastic language, stochastic deterministic finite automaton.
\\Import as traits: activities, iterable language, finite language, finite stochastic language, queriable stochastic language, iterable stochastic language, event log, stochastic semantics, stochastic deterministic semantics, stochastic semantics, semantics.
\\Input to commands: \\\null\qquad\hyperref[command:Ebi analyse all-traces]{\texttt{Ebi analyse all-traces}} (Section~\ref{command:Ebi analyse all-traces})\\\null\qquad\hyperref[command:Ebi analyse completeness]{\texttt{Ebi analyse completeness}} (Section~\ref{command:Ebi analyse completeness})\\\null\qquad\hyperref[command:Ebi analyse coverage]{\texttt{Ebi analyse coverage}} (Section~\ref{command:Ebi analyse coverage})\\\null\qquad\hyperref[command:Ebi analyse medoid]{\texttt{Ebi analyse medoid}} (Section~\ref{command:Ebi analyse medoid})\\\null\qquad\hyperref[command:Ebi analyse minimum-probability-traces]{\texttt{Ebi analyse minimum-probability-traces}} (Section~\ref{command:Ebi analyse minimum-probability-traces})\\\null\qquad\hyperref[command:Ebi analyse mode]{\texttt{Ebi analyse mode}} (Section~\ref{command:Ebi analyse mode})\\\null\qquad\hyperref[command:Ebi analyse most-likely-traces]{\texttt{Ebi analyse most-likely-traces}} (Section~\ref{command:Ebi analyse most-likely-traces})\\\null\qquad\hyperref[command:Ebi analyse variety]{\texttt{Ebi analyse variety}} (Section~\ref{command:Ebi analyse variety})\\\null\qquad\hyperref[command:Ebi analyse-non-stochastic any-traces]{\texttt{Ebi analyse-non-stochastic any-traces}} (Section~\ref{command:Ebi analyse-non-stochastic any-traces})\\\null\qquad\hyperref[command:Ebi analyse-non-stochastic bounded]{\texttt{Ebi analyse-non-stochastic bounded}} (Section~\ref{command:Ebi analyse-non-stochastic bounded})\\\null\qquad\hyperref[command:Ebi analyse-non-stochastic cluster]{\texttt{Ebi analyse-non-stochastic cluster}} (Section~\ref{command:Ebi analyse-non-stochastic cluster})\\\null\qquad\hyperref[command:Ebi analyse-non-stochastic executions]{\texttt{Ebi analyse-non-stochastic executions}} (Section~\ref{command:Ebi analyse-non-stochastic executions})\\\null\qquad\hyperref[command:Ebi analyse-non-stochastic infinitely-many-traces]{\texttt{Ebi analyse-non-stochastic infinitely-many-traces}} (Section~\ref{command:Ebi analyse-non-stochastic infinitely-many-traces})\\\null\qquad\hyperref[command:Ebi analyse-non-stochastic medoid]{\texttt{Ebi analyse-non-stochastic medoid}} (Section~\ref{command:Ebi analyse-non-stochastic medoid})\\\null\qquad\hyperref[command:Ebi association all-trace-attributes]{\texttt{Ebi association all-trace-attributes}} (Section~\ref{command:Ebi association all-trace-attributes})\\\null\qquad\hyperref[command:Ebi association trace-attribute]{\texttt{Ebi association trace-attribute}} (Section~\ref{command:Ebi association trace-attribute})\\\null\qquad\hyperref[command:Ebi conformance earth-movers-stochastic-conformance]{\texttt{Ebi conformance earth-movers-stochastic-conformance}} (Section~\ref{command:Ebi conformance earth-movers-stochastic-conformance})\\\null\qquad\hyperref[command:Ebi conformance earth-movers-stochastic-conformance-sample]{\texttt{Ebi conformance earth-movers-stochastic-conformance-sample}} (Section~\ref{command:Ebi conformance earth-movers-stochastic-conformance-sample})\\\null\qquad\hyperref[command:Ebi conformance entropic-relevance]{\texttt{Ebi conformance entropic-relevance}} (Section~\ref{command:Ebi conformance entropic-relevance})\\\null\qquad\hyperref[command:Ebi conformance jensen-shannon]{\texttt{Ebi conformance jensen-shannon}} (Section~\ref{command:Ebi conformance jensen-shannon})\\\null\qquad\hyperref[command:Ebi conformance jensen-shannon-sample]{\texttt{Ebi conformance jensen-shannon-sample}} (Section~\ref{command:Ebi conformance jensen-shannon-sample})\\\null\qquad\hyperref[command:Ebi conformance unit-earth-movers-stochastic-conformance]{\texttt{Ebi conformance unit-earth-movers-stochastic-conformance}} (Section~\ref{command:Ebi conformance unit-earth-movers-stochastic-conformance})\\\null\qquad\hyperref[command:Ebi conformance-non-stochastic alignments]{\texttt{Ebi conformance-non-stochastic alignments}} (Section~\ref{command:Ebi conformance-non-stochastic alignments})\\\null\qquad\hyperref[command:Ebi conformance-non-stochastic escaping-edges-precision]{\texttt{Ebi conformance-non-stochastic escaping-edges-precision}} (Section~\ref{command:Ebi conformance-non-stochastic escaping-edges-precision})\\\null\qquad\hyperref[command:Ebi conformance-non-stochastic set-alignments]{\texttt{Ebi conformance-non-stochastic set-alignments}} (Section~\ref{command:Ebi conformance-non-stochastic set-alignments})\\\null\qquad\hyperref[command:Ebi convert finite-stochastic-language]{\texttt{Ebi convert finite-stochastic-language}} (Section~\ref{command:Ebi convert finite-stochastic-language})\\\null\qquad\hyperref[command:Ebi convert stochastic-finite-deterministic-automaton]{\texttt{Ebi convert stochastic-finite-deterministic-automaton}} (Section~\ref{command:Ebi convert stochastic-finite-deterministic-automaton})\\\null\qquad\hyperref[command:Ebi discover alignments]{\texttt{Ebi discover alignments}} (Section~\ref{command:Ebi discover alignments})\\\null\qquad\hyperref[command:Ebi discover directly-follows-graph]{\texttt{Ebi discover directly-follows-graph}} (Section~\ref{command:Ebi discover directly-follows-graph})\\\null\qquad\hyperref[command:Ebi discover occurrence labelled-petri-net]{\texttt{Ebi discover occurrence labelled-petri-net}} (Section~\ref{command:Ebi discover occurrence labelled-petri-net})\\\null\qquad\hyperref[command:Ebi discover occurrence process-tree]{\texttt{Ebi discover occurrence process-tree}} (Section~\ref{command:Ebi discover occurrence process-tree})\\\null\qquad\hyperref[command:Ebi discover-non-stochastic flower deterministic-finite-automaton]{\texttt{Ebi discover-non-stochastic flower deterministic-finite-automaton}} (Section~\ref{command:Ebi discover-non-stochastic flower deterministic-finite-automaton})\\\null\qquad\hyperref[command:Ebi discover-non-stochastic flower process-tree]{\texttt{Ebi discover-non-stochastic flower process-tree}} (Section~\ref{command:Ebi discover-non-stochastic flower process-tree})\\\null\qquad\hyperref[command:Ebi discover-non-stochastic prefix-tree deterministic-finite-automaton]{\texttt{Ebi discover-non-stochastic prefix-tree deterministic-finite-automaton}} (Section~\ref{command:Ebi discover-non-stochastic prefix-tree deterministic-finite-automaton})\\\null\qquad\hyperref[command:Ebi discover-non-stochastic prefix-tree process-tree]{\texttt{Ebi discover-non-stochastic prefix-tree process-tree}} (Section~\ref{command:Ebi discover-non-stochastic prefix-tree process-tree})\\\null\qquad\hyperref[command:Ebi information]{\texttt{Ebi information}} (Section~\ref{command:Ebi information})\\\null\qquad\hyperref[command:Ebi probability explain-trace]{\texttt{Ebi probability explain-trace}} (Section~\ref{command:Ebi probability explain-trace})\\\null\qquad\hyperref[command:Ebi probability log]{\texttt{Ebi probability log}} (Section~\ref{command:Ebi probability log})\\\null\qquad\hyperref[command:Ebi probability trace]{\texttt{Ebi probability trace}} (Section~\ref{command:Ebi probability trace})\\\null\qquad\hyperref[command:Ebi sample]{\texttt{Ebi sample}} (Section~\ref{command:Ebi sample})\\\null\qquad\hyperref[command:Ebi test bootstrap-test]{\texttt{Ebi test bootstrap-test}} (Section~\ref{command:Ebi test bootstrap-test})\\\null\qquad\hyperref[command:Ebi test log-categorical-attribute]{\texttt{Ebi test log-categorical-attribute}} (Section~\ref{command:Ebi test log-categorical-attribute})\\\null\qquad\hyperref[command:Ebi validate]{\texttt{Ebi validate}} (Section~\ref{command:Ebi validate})\\\null\qquad\hyperref[command:Ebi visualise text]{\texttt{Ebi visualise text}} (Section~\ref{command:Ebi visualise text}).
\\Output of commands: none.
\\An event log (.xes) can be imported and exported between Ebi, and ProM and Java.
\\File format specification:
An event log file follows the IEEE XES format~\cite{DBLP:journals/cim/AcamporaVSAGV17}. 
Parsing is performed by the Rust4PM crate~\cite{DBLP:conf/bpm/KustersA24}.
For instance:
    \lstinputlisting[language=xml, style=boxed]{../testfiles/a-b.xes}
\clearpage
\subsection{Executions (.exs)}
\label{filehandler:executions}
Import as objects: executions.
\\Import as traits: none.
\\Input to commands: \\\null\qquad\hyperref[command:Ebi information]{\texttt{Ebi information}} (Section~\ref{command:Ebi information})\\\null\qquad\hyperref[command:Ebi validate]{\texttt{Ebi validate}} (Section~\ref{command:Ebi validate})\\\null\qquad\hyperref[command:Ebi visualise text]{\texttt{Ebi visualise text}} (Section~\ref{command:Ebi visualise text}).
\\Output of commands: \\\null\qquad\hyperref[command:Ebi analyse-non-stochastic executions]{\texttt{Ebi analyse-non-stochastic executions}} (Section~\ref{command:Ebi analyse-non-stochastic executions}).
\\ executions (.exs) can be neither imported nor exported between Ebi, and ProM and Java.
\\File format specification:
not yet finalised.
\clearpage
\subsection{Finite language (.lang)}
\label{filehandler:finite language}
Import as objects: finite language, deterministic finite automaton.
\\Import as traits: activities, iterable language, finite language, semantics.
\\Input to commands: \\\null\qquad\hyperref[command:Ebi analyse-non-stochastic any-traces]{\texttt{Ebi analyse-non-stochastic any-traces}} (Section~\ref{command:Ebi analyse-non-stochastic any-traces})\\\null\qquad\hyperref[command:Ebi analyse-non-stochastic bounded]{\texttt{Ebi analyse-non-stochastic bounded}} (Section~\ref{command:Ebi analyse-non-stochastic bounded})\\\null\qquad\hyperref[command:Ebi analyse-non-stochastic cluster]{\texttt{Ebi analyse-non-stochastic cluster}} (Section~\ref{command:Ebi analyse-non-stochastic cluster})\\\null\qquad\hyperref[command:Ebi analyse-non-stochastic executions]{\texttt{Ebi analyse-non-stochastic executions}} (Section~\ref{command:Ebi analyse-non-stochastic executions})\\\null\qquad\hyperref[command:Ebi analyse-non-stochastic infinitely-many-traces]{\texttt{Ebi analyse-non-stochastic infinitely-many-traces}} (Section~\ref{command:Ebi analyse-non-stochastic infinitely-many-traces})\\\null\qquad\hyperref[command:Ebi analyse-non-stochastic medoid]{\texttt{Ebi analyse-non-stochastic medoid}} (Section~\ref{command:Ebi analyse-non-stochastic medoid})\\\null\qquad\hyperref[command:Ebi conformance-non-stochastic alignments]{\texttt{Ebi conformance-non-stochastic alignments}} (Section~\ref{command:Ebi conformance-non-stochastic alignments})\\\null\qquad\hyperref[command:Ebi conformance-non-stochastic escaping-edges-precision]{\texttt{Ebi conformance-non-stochastic escaping-edges-precision}} (Section~\ref{command:Ebi conformance-non-stochastic escaping-edges-precision})\\\null\qquad\hyperref[command:Ebi conformance-non-stochastic set-alignments]{\texttt{Ebi conformance-non-stochastic set-alignments}} (Section~\ref{command:Ebi conformance-non-stochastic set-alignments})\\\null\qquad\hyperref[command:Ebi discover-non-stochastic flower deterministic-finite-automaton]{\texttt{Ebi discover-non-stochastic flower deterministic-finite-automaton}} (Section~\ref{command:Ebi discover-non-stochastic flower deterministic-finite-automaton})\\\null\qquad\hyperref[command:Ebi discover-non-stochastic flower process-tree]{\texttt{Ebi discover-non-stochastic flower process-tree}} (Section~\ref{command:Ebi discover-non-stochastic flower process-tree})\\\null\qquad\hyperref[command:Ebi discover-non-stochastic prefix-tree deterministic-finite-automaton]{\texttt{Ebi discover-non-stochastic prefix-tree deterministic-finite-automaton}} (Section~\ref{command:Ebi discover-non-stochastic prefix-tree deterministic-finite-automaton})\\\null\qquad\hyperref[command:Ebi discover-non-stochastic prefix-tree process-tree]{\texttt{Ebi discover-non-stochastic prefix-tree process-tree}} (Section~\ref{command:Ebi discover-non-stochastic prefix-tree process-tree})\\\null\qquad\hyperref[command:Ebi information]{\texttt{Ebi information}} (Section~\ref{command:Ebi information})\\\null\qquad\hyperref[command:Ebi probability log]{\texttt{Ebi probability log}} (Section~\ref{command:Ebi probability log})\\\null\qquad\hyperref[command:Ebi validate]{\texttt{Ebi validate}} (Section~\ref{command:Ebi validate})\\\null\qquad\hyperref[command:Ebi visualise text]{\texttt{Ebi visualise text}} (Section~\ref{command:Ebi visualise text}).
\\Output of commands: \\\null\qquad\hyperref[command:Ebi analyse all-traces]{\texttt{Ebi analyse all-traces}} (Section~\ref{command:Ebi analyse all-traces})\\\null\qquad\hyperref[command:Ebi analyse coverage]{\texttt{Ebi analyse coverage}} (Section~\ref{command:Ebi analyse coverage})\\\null\qquad\hyperref[command:Ebi analyse medoid]{\texttt{Ebi analyse medoid}} (Section~\ref{command:Ebi analyse medoid})\\\null\qquad\hyperref[command:Ebi analyse minimum-probability-traces]{\texttt{Ebi analyse minimum-probability-traces}} (Section~\ref{command:Ebi analyse minimum-probability-traces})\\\null\qquad\hyperref[command:Ebi analyse mode]{\texttt{Ebi analyse mode}} (Section~\ref{command:Ebi analyse mode})\\\null\qquad\hyperref[command:Ebi analyse most-likely-traces]{\texttt{Ebi analyse most-likely-traces}} (Section~\ref{command:Ebi analyse most-likely-traces})\\\null\qquad\hyperref[command:Ebi analyse-non-stochastic cluster]{\texttt{Ebi analyse-non-stochastic cluster}} (Section~\ref{command:Ebi analyse-non-stochastic cluster})\\\null\qquad\hyperref[command:Ebi analyse-non-stochastic medoid]{\texttt{Ebi analyse-non-stochastic medoid}} (Section~\ref{command:Ebi analyse-non-stochastic medoid})\\\null\qquad\hyperref[command:Ebi convert finite-stochastic-language]{\texttt{Ebi convert finite-stochastic-language}} (Section~\ref{command:Ebi convert finite-stochastic-language})\\\null\qquad\hyperref[command:Ebi sample]{\texttt{Ebi sample}} (Section~\ref{command:Ebi sample}).
\\A finite language (.lang) can be neither imported nor exported between Ebi, and ProM and Java.
\\File format specification:
A finite language is a line-based structure. Lines starting with a \# are ignored.
    This first line is exactly `finite language'.
    The second line is the number of traces in the language.
    For each trace, the first line contains the number of events in the trace.
    Then, each subsequent line contains the activity name of one event.
    
    For instance:
    \lstinputlisting[language=ebilines, style=boxed]{../testfiles/aa-ab-ba.lang}
\clearpage
\subsection{Finite stochastic language (.slang)}
\label{filehandler:finite stochastic language}
Import as objects: finite stochastic language, stochastic deterministic finite automaton.
\\Import as traits: activities, iterable language, finite language, finite stochastic language, queriable stochastic language, iterable stochastic language, stochastic semantics, stochastic deterministic semantics, semantics.
\\Input to commands: \\\null\qquad\hyperref[command:Ebi analyse all-traces]{\texttt{Ebi analyse all-traces}} (Section~\ref{command:Ebi analyse all-traces})\\\null\qquad\hyperref[command:Ebi analyse coverage]{\texttt{Ebi analyse coverage}} (Section~\ref{command:Ebi analyse coverage})\\\null\qquad\hyperref[command:Ebi analyse medoid]{\texttt{Ebi analyse medoid}} (Section~\ref{command:Ebi analyse medoid})\\\null\qquad\hyperref[command:Ebi analyse minimum-probability-traces]{\texttt{Ebi analyse minimum-probability-traces}} (Section~\ref{command:Ebi analyse minimum-probability-traces})\\\null\qquad\hyperref[command:Ebi analyse mode]{\texttt{Ebi analyse mode}} (Section~\ref{command:Ebi analyse mode})\\\null\qquad\hyperref[command:Ebi analyse most-likely-traces]{\texttt{Ebi analyse most-likely-traces}} (Section~\ref{command:Ebi analyse most-likely-traces})\\\null\qquad\hyperref[command:Ebi analyse variety]{\texttt{Ebi analyse variety}} (Section~\ref{command:Ebi analyse variety})\\\null\qquad\hyperref[command:Ebi analyse-non-stochastic any-traces]{\texttt{Ebi analyse-non-stochastic any-traces}} (Section~\ref{command:Ebi analyse-non-stochastic any-traces})\\\null\qquad\hyperref[command:Ebi analyse-non-stochastic bounded]{\texttt{Ebi analyse-non-stochastic bounded}} (Section~\ref{command:Ebi analyse-non-stochastic bounded})\\\null\qquad\hyperref[command:Ebi analyse-non-stochastic cluster]{\texttt{Ebi analyse-non-stochastic cluster}} (Section~\ref{command:Ebi analyse-non-stochastic cluster})\\\null\qquad\hyperref[command:Ebi analyse-non-stochastic executions]{\texttt{Ebi analyse-non-stochastic executions}} (Section~\ref{command:Ebi analyse-non-stochastic executions})\\\null\qquad\hyperref[command:Ebi analyse-non-stochastic infinitely-many-traces]{\texttt{Ebi analyse-non-stochastic infinitely-many-traces}} (Section~\ref{command:Ebi analyse-non-stochastic infinitely-many-traces})\\\null\qquad\hyperref[command:Ebi analyse-non-stochastic medoid]{\texttt{Ebi analyse-non-stochastic medoid}} (Section~\ref{command:Ebi analyse-non-stochastic medoid})\\\null\qquad\hyperref[command:Ebi conformance earth-movers-stochastic-conformance]{\texttt{Ebi conformance earth-movers-stochastic-conformance}} (Section~\ref{command:Ebi conformance earth-movers-stochastic-conformance})\\\null\qquad\hyperref[command:Ebi conformance earth-movers-stochastic-conformance-sample]{\texttt{Ebi conformance earth-movers-stochastic-conformance-sample}} (Section~\ref{command:Ebi conformance earth-movers-stochastic-conformance-sample})\\\null\qquad\hyperref[command:Ebi conformance entropic-relevance]{\texttt{Ebi conformance entropic-relevance}} (Section~\ref{command:Ebi conformance entropic-relevance})\\\null\qquad\hyperref[command:Ebi conformance jensen-shannon]{\texttt{Ebi conformance jensen-shannon}} (Section~\ref{command:Ebi conformance jensen-shannon})\\\null\qquad\hyperref[command:Ebi conformance jensen-shannon-sample]{\texttt{Ebi conformance jensen-shannon-sample}} (Section~\ref{command:Ebi conformance jensen-shannon-sample})\\\null\qquad\hyperref[command:Ebi conformance unit-earth-movers-stochastic-conformance]{\texttt{Ebi conformance unit-earth-movers-stochastic-conformance}} (Section~\ref{command:Ebi conformance unit-earth-movers-stochastic-conformance})\\\null\qquad\hyperref[command:Ebi conformance-non-stochastic alignments]{\texttt{Ebi conformance-non-stochastic alignments}} (Section~\ref{command:Ebi conformance-non-stochastic alignments})\\\null\qquad\hyperref[command:Ebi conformance-non-stochastic escaping-edges-precision]{\texttt{Ebi conformance-non-stochastic escaping-edges-precision}} (Section~\ref{command:Ebi conformance-non-stochastic escaping-edges-precision})\\\null\qquad\hyperref[command:Ebi conformance-non-stochastic set-alignments]{\texttt{Ebi conformance-non-stochastic set-alignments}} (Section~\ref{command:Ebi conformance-non-stochastic set-alignments})\\\null\qquad\hyperref[command:Ebi convert finite-stochastic-language]{\texttt{Ebi convert finite-stochastic-language}} (Section~\ref{command:Ebi convert finite-stochastic-language})\\\null\qquad\hyperref[command:Ebi convert stochastic-finite-deterministic-automaton]{\texttt{Ebi convert stochastic-finite-deterministic-automaton}} (Section~\ref{command:Ebi convert stochastic-finite-deterministic-automaton})\\\null\qquad\hyperref[command:Ebi discover alignments]{\texttt{Ebi discover alignments}} (Section~\ref{command:Ebi discover alignments})\\\null\qquad\hyperref[command:Ebi discover directly-follows-graph]{\texttt{Ebi discover directly-follows-graph}} (Section~\ref{command:Ebi discover directly-follows-graph})\\\null\qquad\hyperref[command:Ebi discover occurrence labelled-petri-net]{\texttt{Ebi discover occurrence labelled-petri-net}} (Section~\ref{command:Ebi discover occurrence labelled-petri-net})\\\null\qquad\hyperref[command:Ebi discover occurrence process-tree]{\texttt{Ebi discover occurrence process-tree}} (Section~\ref{command:Ebi discover occurrence process-tree})\\\null\qquad\hyperref[command:Ebi discover-non-stochastic flower deterministic-finite-automaton]{\texttt{Ebi discover-non-stochastic flower deterministic-finite-automaton}} (Section~\ref{command:Ebi discover-non-stochastic flower deterministic-finite-automaton})\\\null\qquad\hyperref[command:Ebi discover-non-stochastic flower process-tree]{\texttt{Ebi discover-non-stochastic flower process-tree}} (Section~\ref{command:Ebi discover-non-stochastic flower process-tree})\\\null\qquad\hyperref[command:Ebi discover-non-stochastic prefix-tree deterministic-finite-automaton]{\texttt{Ebi discover-non-stochastic prefix-tree deterministic-finite-automaton}} (Section~\ref{command:Ebi discover-non-stochastic prefix-tree deterministic-finite-automaton})\\\null\qquad\hyperref[command:Ebi discover-non-stochastic prefix-tree process-tree]{\texttt{Ebi discover-non-stochastic prefix-tree process-tree}} (Section~\ref{command:Ebi discover-non-stochastic prefix-tree process-tree})\\\null\qquad\hyperref[command:Ebi information]{\texttt{Ebi information}} (Section~\ref{command:Ebi information})\\\null\qquad\hyperref[command:Ebi probability explain-trace]{\texttt{Ebi probability explain-trace}} (Section~\ref{command:Ebi probability explain-trace})\\\null\qquad\hyperref[command:Ebi probability log]{\texttt{Ebi probability log}} (Section~\ref{command:Ebi probability log})\\\null\qquad\hyperref[command:Ebi probability trace]{\texttt{Ebi probability trace}} (Section~\ref{command:Ebi probability trace})\\\null\qquad\hyperref[command:Ebi sample]{\texttt{Ebi sample}} (Section~\ref{command:Ebi sample})\\\null\qquad\hyperref[command:Ebi test bootstrap-test]{\texttt{Ebi test bootstrap-test}} (Section~\ref{command:Ebi test bootstrap-test})\\\null\qquad\hyperref[command:Ebi validate]{\texttt{Ebi validate}} (Section~\ref{command:Ebi validate})\\\null\qquad\hyperref[command:Ebi visualise text]{\texttt{Ebi visualise text}} (Section~\ref{command:Ebi visualise text}).
\\Output of commands: \\\null\qquad\hyperref[command:Ebi analyse all-traces]{\texttt{Ebi analyse all-traces}} (Section~\ref{command:Ebi analyse all-traces})\\\null\qquad\hyperref[command:Ebi analyse coverage]{\texttt{Ebi analyse coverage}} (Section~\ref{command:Ebi analyse coverage})\\\null\qquad\hyperref[command:Ebi analyse minimum-probability-traces]{\texttt{Ebi analyse minimum-probability-traces}} (Section~\ref{command:Ebi analyse minimum-probability-traces})\\\null\qquad\hyperref[command:Ebi analyse mode]{\texttt{Ebi analyse mode}} (Section~\ref{command:Ebi analyse mode})\\\null\qquad\hyperref[command:Ebi analyse most-likely-traces]{\texttt{Ebi analyse most-likely-traces}} (Section~\ref{command:Ebi analyse most-likely-traces})\\\null\qquad\hyperref[command:Ebi convert finite-stochastic-language]{\texttt{Ebi convert finite-stochastic-language}} (Section~\ref{command:Ebi convert finite-stochastic-language})\\\null\qquad\hyperref[command:Ebi sample]{\texttt{Ebi sample}} (Section~\ref{command:Ebi sample}).
\\A finite stochastic language (.slang) can be neither imported nor exported between Ebi, and ProM and Java.
\\File format specification:
A finite language is a line-based structure. Lines starting with a \# are ignored.
    This first line is exactly `finite stochastic language'.
    The second line is the number of traces in the language.
    For each trace, the first line is the probability of the trace as a positive fraction or a decimal value.
    The second line contains the number of events in the trace.
    Then, each subsequent line contains the activity name of one event.

    The sum of the probabilities of the traces in the language needs to be $\leq$ 1.
    
    For instance:
    \lstinputlisting[language=ebilines, style=boxed]{../testfiles/aa-ab-ba.slang}
\clearpage
\subsection{Labelled petri net (.lpn)}
\label{filehandler:labelled Petri net}
Import as objects: labelled Petri net.
\\Import as traits: activities, semantics, graphable.
\\Input to commands: \\\null\qquad\hyperref[command:Ebi analyse-non-stochastic any-traces]{\texttt{Ebi analyse-non-stochastic any-traces}} (Section~\ref{command:Ebi analyse-non-stochastic any-traces})\\\null\qquad\hyperref[command:Ebi analyse-non-stochastic bounded]{\texttt{Ebi analyse-non-stochastic bounded}} (Section~\ref{command:Ebi analyse-non-stochastic bounded})\\\null\qquad\hyperref[command:Ebi analyse-non-stochastic executions]{\texttt{Ebi analyse-non-stochastic executions}} (Section~\ref{command:Ebi analyse-non-stochastic executions})\\\null\qquad\hyperref[command:Ebi analyse-non-stochastic infinitely-many-traces]{\texttt{Ebi analyse-non-stochastic infinitely-many-traces}} (Section~\ref{command:Ebi analyse-non-stochastic infinitely-many-traces})\\\null\qquad\hyperref[command:Ebi conformance-non-stochastic alignments]{\texttt{Ebi conformance-non-stochastic alignments}} (Section~\ref{command:Ebi conformance-non-stochastic alignments})\\\null\qquad\hyperref[command:Ebi conformance-non-stochastic escaping-edges-precision]{\texttt{Ebi conformance-non-stochastic escaping-edges-precision}} (Section~\ref{command:Ebi conformance-non-stochastic escaping-edges-precision})\\\null\qquad\hyperref[command:Ebi conformance-non-stochastic set-alignments]{\texttt{Ebi conformance-non-stochastic set-alignments}} (Section~\ref{command:Ebi conformance-non-stochastic set-alignments})\\\null\qquad\hyperref[command:Ebi convert labelled-petri-net]{\texttt{Ebi convert labelled-petri-net}} (Section~\ref{command:Ebi convert labelled-petri-net})\\\null\qquad\hyperref[command:Ebi discover alignments]{\texttt{Ebi discover alignments}} (Section~\ref{command:Ebi discover alignments})\\\null\qquad\hyperref[command:Ebi discover occurrence labelled-petri-net]{\texttt{Ebi discover occurrence labelled-petri-net}} (Section~\ref{command:Ebi discover occurrence labelled-petri-net})\\\null\qquad\hyperref[command:Ebi discover uniform labelled-petri-net]{\texttt{Ebi discover uniform labelled-petri-net}} (Section~\ref{command:Ebi discover uniform labelled-petri-net})\\\null\qquad\hyperref[command:Ebi discover-non-stochastic flower deterministic-finite-automaton]{\texttt{Ebi discover-non-stochastic flower deterministic-finite-automaton}} (Section~\ref{command:Ebi discover-non-stochastic flower deterministic-finite-automaton})\\\null\qquad\hyperref[command:Ebi discover-non-stochastic flower process-tree]{\texttt{Ebi discover-non-stochastic flower process-tree}} (Section~\ref{command:Ebi discover-non-stochastic flower process-tree})\\\null\qquad\hyperref[command:Ebi information]{\texttt{Ebi information}} (Section~\ref{command:Ebi information})\\\null\qquad\hyperref[command:Ebi validate]{\texttt{Ebi validate}} (Section~\ref{command:Ebi validate})\\\null\qquad\hyperref[command:Ebi visualise graph]{\texttt{Ebi visualise graph}} (Section~\ref{command:Ebi visualise graph})\\\null\qquad\hyperref[command:Ebi visualise text]{\texttt{Ebi visualise text}} (Section~\ref{command:Ebi visualise text}).
\\Output of commands: \\\null\qquad\hyperref[command:Ebi convert labelled-petri-net]{\texttt{Ebi convert labelled-petri-net}} (Section~\ref{command:Ebi convert labelled-petri-net})\\\null\qquad\hyperref[command:Ebi convert stochastic-finite-deterministic-automaton]{\texttt{Ebi convert stochastic-finite-deterministic-automaton}} (Section~\ref{command:Ebi convert stochastic-finite-deterministic-automaton})\\\null\qquad\hyperref[command:Ebi discover alignments]{\texttt{Ebi discover alignments}} (Section~\ref{command:Ebi discover alignments})\\\null\qquad\hyperref[command:Ebi discover directly-follows-graph]{\texttt{Ebi discover directly-follows-graph}} (Section~\ref{command:Ebi discover directly-follows-graph})\\\null\qquad\hyperref[command:Ebi discover occurrence labelled-petri-net]{\texttt{Ebi discover occurrence labelled-petri-net}} (Section~\ref{command:Ebi discover occurrence labelled-petri-net})\\\null\qquad\hyperref[command:Ebi discover uniform labelled-petri-net]{\texttt{Ebi discover uniform labelled-petri-net}} (Section~\ref{command:Ebi discover uniform labelled-petri-net})\\\null\qquad\hyperref[command:Ebi discover-non-stochastic flower deterministic-finite-automaton]{\texttt{Ebi discover-non-stochastic flower deterministic-finite-automaton}} (Section~\ref{command:Ebi discover-non-stochastic flower deterministic-finite-automaton})\\\null\qquad\hyperref[command:Ebi discover-non-stochastic flower process-tree]{\texttt{Ebi discover-non-stochastic flower process-tree}} (Section~\ref{command:Ebi discover-non-stochastic flower process-tree})\\\null\qquad\hyperref[command:Ebi discover-non-stochastic prefix-tree deterministic-finite-automaton]{\texttt{Ebi discover-non-stochastic prefix-tree deterministic-finite-automaton}} (Section~\ref{command:Ebi discover-non-stochastic prefix-tree deterministic-finite-automaton})\\\null\qquad\hyperref[command:Ebi discover-non-stochastic prefix-tree process-tree]{\texttt{Ebi discover-non-stochastic prefix-tree process-tree}} (Section~\ref{command:Ebi discover-non-stochastic prefix-tree process-tree}).
\\A labelled Petri net (.lpn) can be imported and exported between Ebi, and ProM and Java.
\\File format specification:
A labelled Petri net is a line-based structure. Lines starting with a \# are ignored.
    This first line is exactly `labelled Petri net'.
    The second line is the number of places in the net.
    The lines thereafter contain the initial marking: each place has its own line with the number of tokens on that place in the initial marking.
    The next line is the number of transitions in the net.
    Then, for each transition, the following lines are next: 
    (i) the word `silent' or the word `label' followed by a space and the name of the activity with which the transition is labelled;
    (ii) the number of input places, followed by a line for each input place with the index of the place;
    (iii) the number of output places, followed by a line for each output place with the index of the place.
    
    For instance:
    \lstinputlisting[language=ebilines, style=boxed]{../testfiles/aa-ab-ba.lpn}
\clearpage
\subsection{Language of alignments (.ali)}
\label{filehandler:language of alignments}
Import as objects: alignments.
\\Import as traits: activities.
\\Input to commands: \\\null\qquad\hyperref[command:Ebi discover-non-stochastic flower deterministic-finite-automaton]{\texttt{Ebi discover-non-stochastic flower deterministic-finite-automaton}} (Section~\ref{command:Ebi discover-non-stochastic flower deterministic-finite-automaton})\\\null\qquad\hyperref[command:Ebi discover-non-stochastic flower process-tree]{\texttt{Ebi discover-non-stochastic flower process-tree}} (Section~\ref{command:Ebi discover-non-stochastic flower process-tree})\\\null\qquad\hyperref[command:Ebi information]{\texttt{Ebi information}} (Section~\ref{command:Ebi information})\\\null\qquad\hyperref[command:Ebi validate]{\texttt{Ebi validate}} (Section~\ref{command:Ebi validate})\\\null\qquad\hyperref[command:Ebi visualise text]{\texttt{Ebi visualise text}} (Section~\ref{command:Ebi visualise text}).
\\Output of commands: \\\null\qquad\hyperref[command:Ebi conformance-non-stochastic alignments]{\texttt{Ebi conformance-non-stochastic alignments}} (Section~\ref{command:Ebi conformance-non-stochastic alignments})\\\null\qquad\hyperref[command:Ebi conformance-non-stochastic set-alignments]{\texttt{Ebi conformance-non-stochastic set-alignments}} (Section~\ref{command:Ebi conformance-non-stochastic set-alignments})\\\null\qquad\hyperref[command:Ebi probability explain-trace]{\texttt{Ebi probability explain-trace}} (Section~\ref{command:Ebi probability explain-trace}).
\\A language of alignments (.ali) can be neither imported nor exported between Ebi, and ProM and Java.
\\File format specification:
A language of alignments is a line-based structure. Lines starting with a \# are ignored.
    This first line is exactly `language of alignments'.
    The second line is the number of alignments in the language.
    For each alignment, the first line contains the number of moves in the alignment.
    Then, each move is given as either 
    \begin{itemize}
        \item `synchronous move', followed by a line with the word `label' followed by a space and the activity label, which is followed with a line with the index of the involved transition.
        \item `silent move', followed by a line with the index of the silent transition.
        \item `log move', followed by a line with the word `label', then a space, and then the activity label.
        \item `model move', followed by a line with the word `label' followed by a space and the activity label, which is followed with a line with the index of the involved ransition.
    \end{itemize}
    Note that the Semantics trait of Ebi, which is what most alignment computations use, requires that every final marking is a deadlock.
    Consequently, an implicit silent transition may be added by the Semantics trait that is not in the model.
    
    For instance:
    \lstinputlisting[language=ebilines, style=boxed]{../testfiles/aa-ab-ba.ali}
\clearpage
\subsection{Lo la petri net (.lola)}
\label{filehandler:LoLa Petri net}
Import as objects: none.
\\Import as traits: activities, semantics, graphable.
\\Input to commands: \\\null\qquad\hyperref[command:Ebi analyse-non-stochastic executions]{\texttt{Ebi analyse-non-stochastic executions}} (Section~\ref{command:Ebi analyse-non-stochastic executions})\\\null\qquad\hyperref[command:Ebi conformance-non-stochastic alignments]{\texttt{Ebi conformance-non-stochastic alignments}} (Section~\ref{command:Ebi conformance-non-stochastic alignments})\\\null\qquad\hyperref[command:Ebi conformance-non-stochastic escaping-edges-precision]{\texttt{Ebi conformance-non-stochastic escaping-edges-precision}} (Section~\ref{command:Ebi conformance-non-stochastic escaping-edges-precision})\\\null\qquad\hyperref[command:Ebi conformance-non-stochastic set-alignments]{\texttt{Ebi conformance-non-stochastic set-alignments}} (Section~\ref{command:Ebi conformance-non-stochastic set-alignments})\\\null\qquad\hyperref[command:Ebi discover-non-stochastic flower deterministic-finite-automaton]{\texttt{Ebi discover-non-stochastic flower deterministic-finite-automaton}} (Section~\ref{command:Ebi discover-non-stochastic flower deterministic-finite-automaton})\\\null\qquad\hyperref[command:Ebi discover-non-stochastic flower process-tree]{\texttt{Ebi discover-non-stochastic flower process-tree}} (Section~\ref{command:Ebi discover-non-stochastic flower process-tree})\\\null\qquad\hyperref[command:Ebi validate]{\texttt{Ebi validate}} (Section~\ref{command:Ebi validate})\\\null\qquad\hyperref[command:Ebi visualise graph]{\texttt{Ebi visualise graph}} (Section~\ref{command:Ebi visualise graph}).
\\Output of commands: \\\null\qquad\hyperref[command:Ebi convert labelled-petri-net]{\texttt{Ebi convert labelled-petri-net}} (Section~\ref{command:Ebi convert labelled-petri-net})\\\null\qquad\hyperref[command:Ebi convert stochastic-finite-deterministic-automaton]{\texttt{Ebi convert stochastic-finite-deterministic-automaton}} (Section~\ref{command:Ebi convert stochastic-finite-deterministic-automaton})\\\null\qquad\hyperref[command:Ebi discover alignments]{\texttt{Ebi discover alignments}} (Section~\ref{command:Ebi discover alignments})\\\null\qquad\hyperref[command:Ebi discover directly-follows-graph]{\texttt{Ebi discover directly-follows-graph}} (Section~\ref{command:Ebi discover directly-follows-graph})\\\null\qquad\hyperref[command:Ebi discover occurrence labelled-petri-net]{\texttt{Ebi discover occurrence labelled-petri-net}} (Section~\ref{command:Ebi discover occurrence labelled-petri-net})\\\null\qquad\hyperref[command:Ebi discover uniform labelled-petri-net]{\texttt{Ebi discover uniform labelled-petri-net}} (Section~\ref{command:Ebi discover uniform labelled-petri-net})\\\null\qquad\hyperref[command:Ebi discover-non-stochastic flower deterministic-finite-automaton]{\texttt{Ebi discover-non-stochastic flower deterministic-finite-automaton}} (Section~\ref{command:Ebi discover-non-stochastic flower deterministic-finite-automaton})\\\null\qquad\hyperref[command:Ebi discover-non-stochastic flower process-tree]{\texttt{Ebi discover-non-stochastic flower process-tree}} (Section~\ref{command:Ebi discover-non-stochastic flower process-tree})\\\null\qquad\hyperref[command:Ebi discover-non-stochastic prefix-tree deterministic-finite-automaton]{\texttt{Ebi discover-non-stochastic prefix-tree deterministic-finite-automaton}} (Section~\ref{command:Ebi discover-non-stochastic prefix-tree deterministic-finite-automaton})\\\null\qquad\hyperref[command:Ebi discover-non-stochastic prefix-tree process-tree]{\texttt{Ebi discover-non-stochastic prefix-tree process-tree}} (Section~\ref{command:Ebi discover-non-stochastic prefix-tree process-tree}).
\\A LoLa Petri net (.lola) can be neither imported nor exported between Ebi, and ProM and Java.
\\File format specification:
A LoLA Petri net language file adheres to the grammar described in~\cite{DBLP:conf/apn/Wolf18a}.
    Note that Ebi does not support place bounds or fairness, and that LoLA nets do not support silent transitions, and has some restrictions on labeling.
For instance:
    \lstinputlisting[language=xml, style=boxed]{../testfiles/mutex.lola}
\clearpage
\subsection{Petri net markup language (.pnml)}
\label{filehandler:Petri net markup language}
Import as objects: labelled Petri net.
\\Import as traits: activities, semantics, graphable.
\\Input to commands: \\\null\qquad\hyperref[command:Ebi analyse-non-stochastic any-traces]{\texttt{Ebi analyse-non-stochastic any-traces}} (Section~\ref{command:Ebi analyse-non-stochastic any-traces})\\\null\qquad\hyperref[command:Ebi analyse-non-stochastic bounded]{\texttt{Ebi analyse-non-stochastic bounded}} (Section~\ref{command:Ebi analyse-non-stochastic bounded})\\\null\qquad\hyperref[command:Ebi analyse-non-stochastic executions]{\texttt{Ebi analyse-non-stochastic executions}} (Section~\ref{command:Ebi analyse-non-stochastic executions})\\\null\qquad\hyperref[command:Ebi analyse-non-stochastic infinitely-many-traces]{\texttt{Ebi analyse-non-stochastic infinitely-many-traces}} (Section~\ref{command:Ebi analyse-non-stochastic infinitely-many-traces})\\\null\qquad\hyperref[command:Ebi conformance-non-stochastic alignments]{\texttt{Ebi conformance-non-stochastic alignments}} (Section~\ref{command:Ebi conformance-non-stochastic alignments})\\\null\qquad\hyperref[command:Ebi conformance-non-stochastic escaping-edges-precision]{\texttt{Ebi conformance-non-stochastic escaping-edges-precision}} (Section~\ref{command:Ebi conformance-non-stochastic escaping-edges-precision})\\\null\qquad\hyperref[command:Ebi conformance-non-stochastic set-alignments]{\texttt{Ebi conformance-non-stochastic set-alignments}} (Section~\ref{command:Ebi conformance-non-stochastic set-alignments})\\\null\qquad\hyperref[command:Ebi convert labelled-petri-net]{\texttt{Ebi convert labelled-petri-net}} (Section~\ref{command:Ebi convert labelled-petri-net})\\\null\qquad\hyperref[command:Ebi discover alignments]{\texttt{Ebi discover alignments}} (Section~\ref{command:Ebi discover alignments})\\\null\qquad\hyperref[command:Ebi discover occurrence labelled-petri-net]{\texttt{Ebi discover occurrence labelled-petri-net}} (Section~\ref{command:Ebi discover occurrence labelled-petri-net})\\\null\qquad\hyperref[command:Ebi discover uniform labelled-petri-net]{\texttt{Ebi discover uniform labelled-petri-net}} (Section~\ref{command:Ebi discover uniform labelled-petri-net})\\\null\qquad\hyperref[command:Ebi discover-non-stochastic flower deterministic-finite-automaton]{\texttt{Ebi discover-non-stochastic flower deterministic-finite-automaton}} (Section~\ref{command:Ebi discover-non-stochastic flower deterministic-finite-automaton})\\\null\qquad\hyperref[command:Ebi discover-non-stochastic flower process-tree]{\texttt{Ebi discover-non-stochastic flower process-tree}} (Section~\ref{command:Ebi discover-non-stochastic flower process-tree})\\\null\qquad\hyperref[command:Ebi information]{\texttt{Ebi information}} (Section~\ref{command:Ebi information})\\\null\qquad\hyperref[command:Ebi validate]{\texttt{Ebi validate}} (Section~\ref{command:Ebi validate})\\\null\qquad\hyperref[command:Ebi visualise graph]{\texttt{Ebi visualise graph}} (Section~\ref{command:Ebi visualise graph})\\\null\qquad\hyperref[command:Ebi visualise text]{\texttt{Ebi visualise text}} (Section~\ref{command:Ebi visualise text}).
\\Output of commands: \\\null\qquad\hyperref[command:Ebi convert labelled-petri-net]{\texttt{Ebi convert labelled-petri-net}} (Section~\ref{command:Ebi convert labelled-petri-net})\\\null\qquad\hyperref[command:Ebi convert stochastic-finite-deterministic-automaton]{\texttt{Ebi convert stochastic-finite-deterministic-automaton}} (Section~\ref{command:Ebi convert stochastic-finite-deterministic-automaton})\\\null\qquad\hyperref[command:Ebi discover alignments]{\texttt{Ebi discover alignments}} (Section~\ref{command:Ebi discover alignments})\\\null\qquad\hyperref[command:Ebi discover directly-follows-graph]{\texttt{Ebi discover directly-follows-graph}} (Section~\ref{command:Ebi discover directly-follows-graph})\\\null\qquad\hyperref[command:Ebi discover occurrence labelled-petri-net]{\texttt{Ebi discover occurrence labelled-petri-net}} (Section~\ref{command:Ebi discover occurrence labelled-petri-net})\\\null\qquad\hyperref[command:Ebi discover occurrence process-tree]{\texttt{Ebi discover occurrence process-tree}} (Section~\ref{command:Ebi discover occurrence process-tree})\\\null\qquad\hyperref[command:Ebi discover uniform labelled-petri-net]{\texttt{Ebi discover uniform labelled-petri-net}} (Section~\ref{command:Ebi discover uniform labelled-petri-net})\\\null\qquad\hyperref[command:Ebi discover uniform process-tree]{\texttt{Ebi discover uniform process-tree}} (Section~\ref{command:Ebi discover uniform process-tree})\\\null\qquad\hyperref[command:Ebi discover-non-stochastic flower deterministic-finite-automaton]{\texttt{Ebi discover-non-stochastic flower deterministic-finite-automaton}} (Section~\ref{command:Ebi discover-non-stochastic flower deterministic-finite-automaton})\\\null\qquad\hyperref[command:Ebi discover-non-stochastic flower process-tree]{\texttt{Ebi discover-non-stochastic flower process-tree}} (Section~\ref{command:Ebi discover-non-stochastic flower process-tree})\\\null\qquad\hyperref[command:Ebi discover-non-stochastic prefix-tree deterministic-finite-automaton]{\texttt{Ebi discover-non-stochastic prefix-tree deterministic-finite-automaton}} (Section~\ref{command:Ebi discover-non-stochastic prefix-tree deterministic-finite-automaton})\\\null\qquad\hyperref[command:Ebi discover-non-stochastic prefix-tree process-tree]{\texttt{Ebi discover-non-stochastic prefix-tree process-tree}} (Section~\ref{command:Ebi discover-non-stochastic prefix-tree process-tree}).
\\A Petri net markup language (.pnml) can be neither imported nor exported between Ebi, and ProM and Java.
\\File format specification:
A Petri net markup language file follows the ISO 15909-2:2011 format~\cite{pnml}. 
Parsing is performed by the Rust4PM crate~\cite{DBLP:conf/bpm/KustersA24}.

    Please note that Ebi ignores any final markings.
    Instead, every deadlock is considered a final marking.

    For instance:
    \lstinputlisting[language=xml, style=boxed]{../testfiles/a.pnml}
\clearpage
\subsection{Stochastic deterministic finite automaton (.sdfa)}
\label{filehandler:stochastic deterministic finite automaton}
Import as objects: stochastic deterministic finite automaton, labelled Petri net.
\\Import as traits: activities, queriable stochastic language, stochastic deterministic semantics, stochastic semantics, semantics, graphable.
\\Input to commands: \\\null\qquad\hyperref[command:Ebi analyse all-traces]{\texttt{Ebi analyse all-traces}} (Section~\ref{command:Ebi analyse all-traces})\\\null\qquad\hyperref[command:Ebi analyse coverage]{\texttt{Ebi analyse coverage}} (Section~\ref{command:Ebi analyse coverage})\\\null\qquad\hyperref[command:Ebi analyse minimum-probability-traces]{\texttt{Ebi analyse minimum-probability-traces}} (Section~\ref{command:Ebi analyse minimum-probability-traces})\\\null\qquad\hyperref[command:Ebi analyse mode]{\texttt{Ebi analyse mode}} (Section~\ref{command:Ebi analyse mode})\\\null\qquad\hyperref[command:Ebi analyse most-likely-traces]{\texttt{Ebi analyse most-likely-traces}} (Section~\ref{command:Ebi analyse most-likely-traces})\\\null\qquad\hyperref[command:Ebi analyse-non-stochastic any-traces]{\texttt{Ebi analyse-non-stochastic any-traces}} (Section~\ref{command:Ebi analyse-non-stochastic any-traces})\\\null\qquad\hyperref[command:Ebi analyse-non-stochastic bounded]{\texttt{Ebi analyse-non-stochastic bounded}} (Section~\ref{command:Ebi analyse-non-stochastic bounded})\\\null\qquad\hyperref[command:Ebi analyse-non-stochastic executions]{\texttt{Ebi analyse-non-stochastic executions}} (Section~\ref{command:Ebi analyse-non-stochastic executions})\\\null\qquad\hyperref[command:Ebi analyse-non-stochastic infinitely-many-traces]{\texttt{Ebi analyse-non-stochastic infinitely-many-traces}} (Section~\ref{command:Ebi analyse-non-stochastic infinitely-many-traces})\\\null\qquad\hyperref[command:Ebi conformance earth-movers-stochastic-conformance-sample]{\texttt{Ebi conformance earth-movers-stochastic-conformance-sample}} (Section~\ref{command:Ebi conformance earth-movers-stochastic-conformance-sample})\\\null\qquad\hyperref[command:Ebi conformance entropic-relevance]{\texttt{Ebi conformance entropic-relevance}} (Section~\ref{command:Ebi conformance entropic-relevance})\\\null\qquad\hyperref[command:Ebi conformance jensen-shannon]{\texttt{Ebi conformance jensen-shannon}} (Section~\ref{command:Ebi conformance jensen-shannon})\\\null\qquad\hyperref[command:Ebi conformance jensen-shannon-sample]{\texttt{Ebi conformance jensen-shannon-sample}} (Section~\ref{command:Ebi conformance jensen-shannon-sample})\\\null\qquad\hyperref[command:Ebi conformance unit-earth-movers-stochastic-conformance]{\texttt{Ebi conformance unit-earth-movers-stochastic-conformance}} (Section~\ref{command:Ebi conformance unit-earth-movers-stochastic-conformance})\\\null\qquad\hyperref[command:Ebi conformance-non-stochastic alignments]{\texttt{Ebi conformance-non-stochastic alignments}} (Section~\ref{command:Ebi conformance-non-stochastic alignments})\\\null\qquad\hyperref[command:Ebi conformance-non-stochastic escaping-edges-precision]{\texttt{Ebi conformance-non-stochastic escaping-edges-precision}} (Section~\ref{command:Ebi conformance-non-stochastic escaping-edges-precision})\\\null\qquad\hyperref[command:Ebi conformance-non-stochastic set-alignments]{\texttt{Ebi conformance-non-stochastic set-alignments}} (Section~\ref{command:Ebi conformance-non-stochastic set-alignments})\\\null\qquad\hyperref[command:Ebi convert labelled-petri-net]{\texttt{Ebi convert labelled-petri-net}} (Section~\ref{command:Ebi convert labelled-petri-net})\\\null\qquad\hyperref[command:Ebi convert stochastic-finite-deterministic-automaton]{\texttt{Ebi convert stochastic-finite-deterministic-automaton}} (Section~\ref{command:Ebi convert stochastic-finite-deterministic-automaton})\\\null\qquad\hyperref[command:Ebi discover alignments]{\texttt{Ebi discover alignments}} (Section~\ref{command:Ebi discover alignments})\\\null\qquad\hyperref[command:Ebi discover occurrence labelled-petri-net]{\texttt{Ebi discover occurrence labelled-petri-net}} (Section~\ref{command:Ebi discover occurrence labelled-petri-net})\\\null\qquad\hyperref[command:Ebi discover uniform labelled-petri-net]{\texttt{Ebi discover uniform labelled-petri-net}} (Section~\ref{command:Ebi discover uniform labelled-petri-net})\\\null\qquad\hyperref[command:Ebi discover-non-stochastic flower deterministic-finite-automaton]{\texttt{Ebi discover-non-stochastic flower deterministic-finite-automaton}} (Section~\ref{command:Ebi discover-non-stochastic flower deterministic-finite-automaton})\\\null\qquad\hyperref[command:Ebi discover-non-stochastic flower process-tree]{\texttt{Ebi discover-non-stochastic flower process-tree}} (Section~\ref{command:Ebi discover-non-stochastic flower process-tree})\\\null\qquad\hyperref[command:Ebi information]{\texttt{Ebi information}} (Section~\ref{command:Ebi information})\\\null\qquad\hyperref[command:Ebi probability explain-trace]{\texttt{Ebi probability explain-trace}} (Section~\ref{command:Ebi probability explain-trace})\\\null\qquad\hyperref[command:Ebi probability log]{\texttt{Ebi probability log}} (Section~\ref{command:Ebi probability log})\\\null\qquad\hyperref[command:Ebi probability trace]{\texttt{Ebi probability trace}} (Section~\ref{command:Ebi probability trace})\\\null\qquad\hyperref[command:Ebi sample]{\texttt{Ebi sample}} (Section~\ref{command:Ebi sample})\\\null\qquad\hyperref[command:Ebi validate]{\texttt{Ebi validate}} (Section~\ref{command:Ebi validate})\\\null\qquad\hyperref[command:Ebi visualise graph]{\texttt{Ebi visualise graph}} (Section~\ref{command:Ebi visualise graph})\\\null\qquad\hyperref[command:Ebi visualise text]{\texttt{Ebi visualise text}} (Section~\ref{command:Ebi visualise text}).
\\Output of commands: \\\null\qquad\hyperref[command:Ebi analyse all-traces]{\texttt{Ebi analyse all-traces}} (Section~\ref{command:Ebi analyse all-traces})\\\null\qquad\hyperref[command:Ebi analyse coverage]{\texttt{Ebi analyse coverage}} (Section~\ref{command:Ebi analyse coverage})\\\null\qquad\hyperref[command:Ebi analyse minimum-probability-traces]{\texttt{Ebi analyse minimum-probability-traces}} (Section~\ref{command:Ebi analyse minimum-probability-traces})\\\null\qquad\hyperref[command:Ebi analyse mode]{\texttt{Ebi analyse mode}} (Section~\ref{command:Ebi analyse mode})\\\null\qquad\hyperref[command:Ebi analyse most-likely-traces]{\texttt{Ebi analyse most-likely-traces}} (Section~\ref{command:Ebi analyse most-likely-traces})\\\null\qquad\hyperref[command:Ebi convert finite-stochastic-language]{\texttt{Ebi convert finite-stochastic-language}} (Section~\ref{command:Ebi convert finite-stochastic-language})\\\null\qquad\hyperref[command:Ebi convert stochastic-finite-deterministic-automaton]{\texttt{Ebi convert stochastic-finite-deterministic-automaton}} (Section~\ref{command:Ebi convert stochastic-finite-deterministic-automaton})\\\null\qquad\hyperref[command:Ebi sample]{\texttt{Ebi sample}} (Section~\ref{command:Ebi sample}).
\\A stochastic deterministic finite automaton (.sdfa) can be neither imported nor exported between Ebi, and ProM and Java.
\\File format specification:
A stochastic deterministic finite automaton is a JSON structure with the top level being an object.
    This object contains the following key-value pairs:
    \begin{itemize}
    \item \texttt{initialState} being the index of the initial state. This field is optional: if omitted, the SDFA has an empty stochastic language.
    \item \texttt{transitions} being a list of transitions. 
    Each transition is an object with \texttt{from} being the source state index of the transition, 
    \texttt{to} being the target state index of the transition, 
    \texttt{label} being the activity of the transition, and
    \texttt{prob} being the probability of the transition (may be given as a fraction in a string or a float value. Must be $\leq 1$). 
    Silent transitions are not supported.
    The file format supports deadlocks and livelocks.
    The probability that a trace terminates in a state is 1 - the sum probability of the outgoing transitions of the state.
    \end{itemize}
    For instance:
    \lstinputlisting[language=json, style=boxed]{../testfiles/aa-ab-ba.sdfa}
\clearpage
\subsection{Stochastic labelled petri net (.slpn)}
\label{filehandler:stochastic labelled Petri net}
Import as objects: stochastic labelled Petri net, labelled Petri net.
\\Import as traits: activities, queriable stochastic language, stochastic deterministic semantics, stochastic semantics, semantics, graphable.
\\Input to commands: \\\null\qquad\hyperref[command:Ebi analyse all-traces]{\texttt{Ebi analyse all-traces}} (Section~\ref{command:Ebi analyse all-traces})\\\null\qquad\hyperref[command:Ebi analyse coverage]{\texttt{Ebi analyse coverage}} (Section~\ref{command:Ebi analyse coverage})\\\null\qquad\hyperref[command:Ebi analyse minimum-probability-traces]{\texttt{Ebi analyse minimum-probability-traces}} (Section~\ref{command:Ebi analyse minimum-probability-traces})\\\null\qquad\hyperref[command:Ebi analyse mode]{\texttt{Ebi analyse mode}} (Section~\ref{command:Ebi analyse mode})\\\null\qquad\hyperref[command:Ebi analyse most-likely-traces]{\texttt{Ebi analyse most-likely-traces}} (Section~\ref{command:Ebi analyse most-likely-traces})\\\null\qquad\hyperref[command:Ebi analyse-non-stochastic any-traces]{\texttt{Ebi analyse-non-stochastic any-traces}} (Section~\ref{command:Ebi analyse-non-stochastic any-traces})\\\null\qquad\hyperref[command:Ebi analyse-non-stochastic bounded]{\texttt{Ebi analyse-non-stochastic bounded}} (Section~\ref{command:Ebi analyse-non-stochastic bounded})\\\null\qquad\hyperref[command:Ebi analyse-non-stochastic executions]{\texttt{Ebi analyse-non-stochastic executions}} (Section~\ref{command:Ebi analyse-non-stochastic executions})\\\null\qquad\hyperref[command:Ebi analyse-non-stochastic infinitely-many-traces]{\texttt{Ebi analyse-non-stochastic infinitely-many-traces}} (Section~\ref{command:Ebi analyse-non-stochastic infinitely-many-traces})\\\null\qquad\hyperref[command:Ebi conformance earth-movers-stochastic-conformance-sample]{\texttt{Ebi conformance earth-movers-stochastic-conformance-sample}} (Section~\ref{command:Ebi conformance earth-movers-stochastic-conformance-sample})\\\null\qquad\hyperref[command:Ebi conformance entropic-relevance]{\texttt{Ebi conformance entropic-relevance}} (Section~\ref{command:Ebi conformance entropic-relevance})\\\null\qquad\hyperref[command:Ebi conformance jensen-shannon]{\texttt{Ebi conformance jensen-shannon}} (Section~\ref{command:Ebi conformance jensen-shannon})\\\null\qquad\hyperref[command:Ebi conformance jensen-shannon-sample]{\texttt{Ebi conformance jensen-shannon-sample}} (Section~\ref{command:Ebi conformance jensen-shannon-sample})\\\null\qquad\hyperref[command:Ebi conformance unit-earth-movers-stochastic-conformance]{\texttt{Ebi conformance unit-earth-movers-stochastic-conformance}} (Section~\ref{command:Ebi conformance unit-earth-movers-stochastic-conformance})\\\null\qquad\hyperref[command:Ebi conformance-non-stochastic alignments]{\texttt{Ebi conformance-non-stochastic alignments}} (Section~\ref{command:Ebi conformance-non-stochastic alignments})\\\null\qquad\hyperref[command:Ebi conformance-non-stochastic escaping-edges-precision]{\texttt{Ebi conformance-non-stochastic escaping-edges-precision}} (Section~\ref{command:Ebi conformance-non-stochastic escaping-edges-precision})\\\null\qquad\hyperref[command:Ebi conformance-non-stochastic set-alignments]{\texttt{Ebi conformance-non-stochastic set-alignments}} (Section~\ref{command:Ebi conformance-non-stochastic set-alignments})\\\null\qquad\hyperref[command:Ebi convert labelled-petri-net]{\texttt{Ebi convert labelled-petri-net}} (Section~\ref{command:Ebi convert labelled-petri-net})\\\null\qquad\hyperref[command:Ebi discover alignments]{\texttt{Ebi discover alignments}} (Section~\ref{command:Ebi discover alignments})\\\null\qquad\hyperref[command:Ebi discover occurrence labelled-petri-net]{\texttt{Ebi discover occurrence labelled-petri-net}} (Section~\ref{command:Ebi discover occurrence labelled-petri-net})\\\null\qquad\hyperref[command:Ebi discover uniform labelled-petri-net]{\texttt{Ebi discover uniform labelled-petri-net}} (Section~\ref{command:Ebi discover uniform labelled-petri-net})\\\null\qquad\hyperref[command:Ebi discover-non-stochastic flower deterministic-finite-automaton]{\texttt{Ebi discover-non-stochastic flower deterministic-finite-automaton}} (Section~\ref{command:Ebi discover-non-stochastic flower deterministic-finite-automaton})\\\null\qquad\hyperref[command:Ebi discover-non-stochastic flower process-tree]{\texttt{Ebi discover-non-stochastic flower process-tree}} (Section~\ref{command:Ebi discover-non-stochastic flower process-tree})\\\null\qquad\hyperref[command:Ebi information]{\texttt{Ebi information}} (Section~\ref{command:Ebi information})\\\null\qquad\hyperref[command:Ebi probability explain-trace]{\texttt{Ebi probability explain-trace}} (Section~\ref{command:Ebi probability explain-trace})\\\null\qquad\hyperref[command:Ebi probability log]{\texttt{Ebi probability log}} (Section~\ref{command:Ebi probability log})\\\null\qquad\hyperref[command:Ebi probability trace]{\texttt{Ebi probability trace}} (Section~\ref{command:Ebi probability trace})\\\null\qquad\hyperref[command:Ebi sample]{\texttt{Ebi sample}} (Section~\ref{command:Ebi sample})\\\null\qquad\hyperref[command:Ebi validate]{\texttt{Ebi validate}} (Section~\ref{command:Ebi validate})\\\null\qquad\hyperref[command:Ebi visualise graph]{\texttt{Ebi visualise graph}} (Section~\ref{command:Ebi visualise graph})\\\null\qquad\hyperref[command:Ebi visualise text]{\texttt{Ebi visualise text}} (Section~\ref{command:Ebi visualise text}).
\\Output of commands: \\\null\qquad\hyperref[command:Ebi discover alignments]{\texttt{Ebi discover alignments}} (Section~\ref{command:Ebi discover alignments})\\\null\qquad\hyperref[command:Ebi discover directly-follows-graph]{\texttt{Ebi discover directly-follows-graph}} (Section~\ref{command:Ebi discover directly-follows-graph})\\\null\qquad\hyperref[command:Ebi discover occurrence labelled-petri-net]{\texttt{Ebi discover occurrence labelled-petri-net}} (Section~\ref{command:Ebi discover occurrence labelled-petri-net})\\\null\qquad\hyperref[command:Ebi discover uniform labelled-petri-net]{\texttt{Ebi discover uniform labelled-petri-net}} (Section~\ref{command:Ebi discover uniform labelled-petri-net}).
\\A stochastic labelled Petri net (.slpn) can be imported and exported between Ebi, and ProM and Java.
\\File format specification:
A stochastic labelled Petri net is a line-based structure. Lines starting with a \# are ignored.
    This first line is exactly `stochastic labelled Petri net'.
    The second line is the number of places in the net.
    The lines thereafter contain the initial marking: each place has its own line with the number of tokens on that place in the initial marking.
    The next line is the number of transitions in the net.
    Then, for each transition, the following lines are next: 
    (i) the word `silent' or the word `label' followed by a space and the name of the activity with which the transition is labelled;
    (ii) the weight of the transition, which may be any fraction or decimal number, even 0 or negative;
    (iii) the number of input places, followed by a line for each input place with the index of the place;
    (iiii) the number of output places, followed by a line for each output place with the index of the place.
    
    For instance:
    \lstinputlisting[language=ebilines, style=boxed]{../testfiles/aa-ab-ba_ali.slpn}
\clearpage
\subsection{Process tree (.ptree)}
\label{filehandler:process tree}
Import as objects: process tree, labelled Petri net.
\\Import as traits: activities, semantics, graphable.
\\Input to commands: \\\null\qquad\hyperref[command:Ebi analyse-non-stochastic any-traces]{\texttt{Ebi analyse-non-stochastic any-traces}} (Section~\ref{command:Ebi analyse-non-stochastic any-traces})\\\null\qquad\hyperref[command:Ebi analyse-non-stochastic bounded]{\texttt{Ebi analyse-non-stochastic bounded}} (Section~\ref{command:Ebi analyse-non-stochastic bounded})\\\null\qquad\hyperref[command:Ebi analyse-non-stochastic executions]{\texttt{Ebi analyse-non-stochastic executions}} (Section~\ref{command:Ebi analyse-non-stochastic executions})\\\null\qquad\hyperref[command:Ebi analyse-non-stochastic infinitely-many-traces]{\texttt{Ebi analyse-non-stochastic infinitely-many-traces}} (Section~\ref{command:Ebi analyse-non-stochastic infinitely-many-traces})\\\null\qquad\hyperref[command:Ebi conformance-non-stochastic alignments]{\texttt{Ebi conformance-non-stochastic alignments}} (Section~\ref{command:Ebi conformance-non-stochastic alignments})\\\null\qquad\hyperref[command:Ebi conformance-non-stochastic escaping-edges-precision]{\texttt{Ebi conformance-non-stochastic escaping-edges-precision}} (Section~\ref{command:Ebi conformance-non-stochastic escaping-edges-precision})\\\null\qquad\hyperref[command:Ebi conformance-non-stochastic set-alignments]{\texttt{Ebi conformance-non-stochastic set-alignments}} (Section~\ref{command:Ebi conformance-non-stochastic set-alignments})\\\null\qquad\hyperref[command:Ebi convert labelled-petri-net]{\texttt{Ebi convert labelled-petri-net}} (Section~\ref{command:Ebi convert labelled-petri-net})\\\null\qquad\hyperref[command:Ebi discover alignments]{\texttt{Ebi discover alignments}} (Section~\ref{command:Ebi discover alignments})\\\null\qquad\hyperref[command:Ebi discover occurrence labelled-petri-net]{\texttt{Ebi discover occurrence labelled-petri-net}} (Section~\ref{command:Ebi discover occurrence labelled-petri-net})\\\null\qquad\hyperref[command:Ebi discover occurrence process-tree]{\texttt{Ebi discover occurrence process-tree}} (Section~\ref{command:Ebi discover occurrence process-tree})\\\null\qquad\hyperref[command:Ebi discover uniform labelled-petri-net]{\texttt{Ebi discover uniform labelled-petri-net}} (Section~\ref{command:Ebi discover uniform labelled-petri-net})\\\null\qquad\hyperref[command:Ebi discover uniform process-tree]{\texttt{Ebi discover uniform process-tree}} (Section~\ref{command:Ebi discover uniform process-tree})\\\null\qquad\hyperref[command:Ebi discover-non-stochastic flower deterministic-finite-automaton]{\texttt{Ebi discover-non-stochastic flower deterministic-finite-automaton}} (Section~\ref{command:Ebi discover-non-stochastic flower deterministic-finite-automaton})\\\null\qquad\hyperref[command:Ebi discover-non-stochastic flower process-tree]{\texttt{Ebi discover-non-stochastic flower process-tree}} (Section~\ref{command:Ebi discover-non-stochastic flower process-tree})\\\null\qquad\hyperref[command:Ebi information]{\texttt{Ebi information}} (Section~\ref{command:Ebi information})\\\null\qquad\hyperref[command:Ebi validate]{\texttt{Ebi validate}} (Section~\ref{command:Ebi validate})\\\null\qquad\hyperref[command:Ebi visualise graph]{\texttt{Ebi visualise graph}} (Section~\ref{command:Ebi visualise graph})\\\null\qquad\hyperref[command:Ebi visualise text]{\texttt{Ebi visualise text}} (Section~\ref{command:Ebi visualise text}).
\\Output of commands: \\\null\qquad\hyperref[command:Ebi discover-non-stochastic flower process-tree]{\texttt{Ebi discover-non-stochastic flower process-tree}} (Section~\ref{command:Ebi discover-non-stochastic flower process-tree})\\\null\qquad\hyperref[command:Ebi discover-non-stochastic prefix-tree process-tree]{\texttt{Ebi discover-non-stochastic prefix-tree process-tree}} (Section~\ref{command:Ebi discover-non-stochastic prefix-tree process-tree}).
\\A process tree (.ptree) can be imported and exported between Ebi, and ProM and Java.
\\File format specification:
A process tree is a line-based structure. Lines starting with a \# are ignored.
    This first line is exactly `process tree'.
    The subsequent lines contain the nodes:
    Each node is either:
    \begin{itemize}
        \item A line with the word `activity' followed on the same line by a space and the label of the activity leaf;
        \item The word `tau';
        \item The name of an operator (`sequence', `xor', `concurrent', `loop', `interleaved', or `or') on its own line.
        The line thereafter contains the number of children of the node, after which the nodes are given.
        An operator node must have at least one child.
    \end{itemize}
    Indentation of nodes is allowed, but not mandatory.
    
    For instance:
    \lstinputlisting[language=ebilines, style=boxed]{../testfiles/all_operators.ptree}
\clearpage
\subsection{Portable document format (.pdf)}
\label{filehandler:portable document format}
Import as objects: none.
\\Import as traits: none.
\\Input to commands: none.
\\Output of commands: \\\null\qquad\hyperref[command:Ebi convert labelled-petri-net]{\texttt{Ebi convert labelled-petri-net}} (Section~\ref{command:Ebi convert labelled-petri-net})\\\null\qquad\hyperref[command:Ebi convert stochastic-finite-deterministic-automaton]{\texttt{Ebi convert stochastic-finite-deterministic-automaton}} (Section~\ref{command:Ebi convert stochastic-finite-deterministic-automaton})\\\null\qquad\hyperref[command:Ebi discover alignments]{\texttt{Ebi discover alignments}} (Section~\ref{command:Ebi discover alignments})\\\null\qquad\hyperref[command:Ebi discover directly-follows-graph]{\texttt{Ebi discover directly-follows-graph}} (Section~\ref{command:Ebi discover directly-follows-graph})\\\null\qquad\hyperref[command:Ebi discover occurrence labelled-petri-net]{\texttt{Ebi discover occurrence labelled-petri-net}} (Section~\ref{command:Ebi discover occurrence labelled-petri-net})\\\null\qquad\hyperref[command:Ebi discover occurrence process-tree]{\texttt{Ebi discover occurrence process-tree}} (Section~\ref{command:Ebi discover occurrence process-tree})\\\null\qquad\hyperref[command:Ebi discover uniform labelled-petri-net]{\texttt{Ebi discover uniform labelled-petri-net}} (Section~\ref{command:Ebi discover uniform labelled-petri-net})\\\null\qquad\hyperref[command:Ebi discover uniform process-tree]{\texttt{Ebi discover uniform process-tree}} (Section~\ref{command:Ebi discover uniform process-tree})\\\null\qquad\hyperref[command:Ebi discover-non-stochastic flower deterministic-finite-automaton]{\texttt{Ebi discover-non-stochastic flower deterministic-finite-automaton}} (Section~\ref{command:Ebi discover-non-stochastic flower deterministic-finite-automaton})\\\null\qquad\hyperref[command:Ebi discover-non-stochastic flower process-tree]{\texttt{Ebi discover-non-stochastic flower process-tree}} (Section~\ref{command:Ebi discover-non-stochastic flower process-tree})\\\null\qquad\hyperref[command:Ebi discover-non-stochastic prefix-tree deterministic-finite-automaton]{\texttt{Ebi discover-non-stochastic prefix-tree deterministic-finite-automaton}} (Section~\ref{command:Ebi discover-non-stochastic prefix-tree deterministic-finite-automaton})\\\null\qquad\hyperref[command:Ebi discover-non-stochastic prefix-tree process-tree]{\texttt{Ebi discover-non-stochastic prefix-tree process-tree}} (Section~\ref{command:Ebi discover-non-stochastic prefix-tree process-tree})\\\null\qquad\hyperref[command:Ebi itself graph]{\texttt{Ebi itself graph}} (Section~\ref{command:Ebi itself graph})\\\null\qquad\hyperref[command:Ebi visualise graph]{\texttt{Ebi visualise graph}} (Section~\ref{command:Ebi visualise graph}).
\\A portable document format (.pdf) can be neither imported nor exported between Ebi, and ProM and Java.
\\File format specification:
Ebi does not support importing of PDF files.
\clearpage
\subsection{Scalable vector graphics (.svg)}
\label{filehandler:scalable vector graphics}
Import as objects: none.
\\Import as traits: none.
\\Input to commands: none.
\\Output of commands: \\\null\qquad\hyperref[command:Ebi convert labelled-petri-net]{\texttt{Ebi convert labelled-petri-net}} (Section~\ref{command:Ebi convert labelled-petri-net})\\\null\qquad\hyperref[command:Ebi convert stochastic-finite-deterministic-automaton]{\texttt{Ebi convert stochastic-finite-deterministic-automaton}} (Section~\ref{command:Ebi convert stochastic-finite-deterministic-automaton})\\\null\qquad\hyperref[command:Ebi discover alignments]{\texttt{Ebi discover alignments}} (Section~\ref{command:Ebi discover alignments})\\\null\qquad\hyperref[command:Ebi discover directly-follows-graph]{\texttt{Ebi discover directly-follows-graph}} (Section~\ref{command:Ebi discover directly-follows-graph})\\\null\qquad\hyperref[command:Ebi discover occurrence labelled-petri-net]{\texttt{Ebi discover occurrence labelled-petri-net}} (Section~\ref{command:Ebi discover occurrence labelled-petri-net})\\\null\qquad\hyperref[command:Ebi discover occurrence process-tree]{\texttt{Ebi discover occurrence process-tree}} (Section~\ref{command:Ebi discover occurrence process-tree})\\\null\qquad\hyperref[command:Ebi discover uniform labelled-petri-net]{\texttt{Ebi discover uniform labelled-petri-net}} (Section~\ref{command:Ebi discover uniform labelled-petri-net})\\\null\qquad\hyperref[command:Ebi discover uniform process-tree]{\texttt{Ebi discover uniform process-tree}} (Section~\ref{command:Ebi discover uniform process-tree})\\\null\qquad\hyperref[command:Ebi discover-non-stochastic flower deterministic-finite-automaton]{\texttt{Ebi discover-non-stochastic flower deterministic-finite-automaton}} (Section~\ref{command:Ebi discover-non-stochastic flower deterministic-finite-automaton})\\\null\qquad\hyperref[command:Ebi discover-non-stochastic flower process-tree]{\texttt{Ebi discover-non-stochastic flower process-tree}} (Section~\ref{command:Ebi discover-non-stochastic flower process-tree})\\\null\qquad\hyperref[command:Ebi discover-non-stochastic prefix-tree deterministic-finite-automaton]{\texttt{Ebi discover-non-stochastic prefix-tree deterministic-finite-automaton}} (Section~\ref{command:Ebi discover-non-stochastic prefix-tree deterministic-finite-automaton})\\\null\qquad\hyperref[command:Ebi discover-non-stochastic prefix-tree process-tree]{\texttt{Ebi discover-non-stochastic prefix-tree process-tree}} (Section~\ref{command:Ebi discover-non-stochastic prefix-tree process-tree})\\\null\qquad\hyperref[command:Ebi itself graph]{\texttt{Ebi itself graph}} (Section~\ref{command:Ebi itself graph})\\\null\qquad\hyperref[command:Ebi visualise graph]{\texttt{Ebi visualise graph}} (Section~\ref{command:Ebi visualise graph}).
\\A scalable vector graphics (.svg) can be neither imported nor exported between Ebi, and ProM and Java.
\\File format specification:
Ebi does not support importing of SVG files.
\clearpage
\subsection{Stochastic language of alignments (.sali)}
\label{filehandler:stochastic language of alignments}
Import as objects: stochastic language of alignments.
\\Import as traits: activities.
\\Input to commands: \\\null\qquad\hyperref[command:Ebi conformance-non-stochastic escaping-edges-precision]{\texttt{Ebi conformance-non-stochastic escaping-edges-precision}} (Section~\ref{command:Ebi conformance-non-stochastic escaping-edges-precision})\\\null\qquad\hyperref[command:Ebi conformance-non-stochastic trace-fitness]{\texttt{Ebi conformance-non-stochastic trace-fitness}} (Section~\ref{command:Ebi conformance-non-stochastic trace-fitness})\\\null\qquad\hyperref[command:Ebi discover-non-stochastic flower deterministic-finite-automaton]{\texttt{Ebi discover-non-stochastic flower deterministic-finite-automaton}} (Section~\ref{command:Ebi discover-non-stochastic flower deterministic-finite-automaton})\\\null\qquad\hyperref[command:Ebi discover-non-stochastic flower process-tree]{\texttt{Ebi discover-non-stochastic flower process-tree}} (Section~\ref{command:Ebi discover-non-stochastic flower process-tree})\\\null\qquad\hyperref[command:Ebi information]{\texttt{Ebi information}} (Section~\ref{command:Ebi information})\\\null\qquad\hyperref[command:Ebi validate]{\texttt{Ebi validate}} (Section~\ref{command:Ebi validate})\\\null\qquad\hyperref[command:Ebi visualise text]{\texttt{Ebi visualise text}} (Section~\ref{command:Ebi visualise text}).
\\Output of commands: \\\null\qquad\hyperref[command:Ebi conformance-non-stochastic alignments]{\texttt{Ebi conformance-non-stochastic alignments}} (Section~\ref{command:Ebi conformance-non-stochastic alignments}).
\\A stochastic language of alignments (.sali) can be neither imported nor exported between Ebi, and ProM and Java.
\\File format specification:
A stochastic language of alignments is a line-based structure. Lines starting with a \# are ignored.
    This first line is exactly `stochastic language of alignments'.
    The second line is the number of alignments in the language.
    For each alignment, the first line contains the probability of the alignment in the language, which is either a positive fraction or a decimal value.
    The second line contains the number of moves in the alignment.
    Then, each move is given as either 
    \begin{itemize}
        \item `synchronous move', followed by a line with the word `label' followed by a space and the activity label, which is followed with a line with the index of the involved transition.
        \item `silent move', followed by a line with the index of the silent transition.
        \item `log move', followed by a line with the word `label', then a space, and then the activity label.
        \item `model move', followed by a line with the word `label' followed by a space and the activity label, which is followed with a line with the index of the involved ransition.
    \end{itemize}
    
    For instance:
    \lstinputlisting[language=ebilines, style=boxed]{../testfiles/aa-ab-ba.sali}
\clearpage
\subsection{Stochastic process tree (.sptree)}
\label{filehandler:stochastic process tree}
Import as objects: stochastic process tree, process tree, labelled Petri net.
\\Import as traits: activities, queriable stochastic language, stochastic deterministic semantics, semantics, stochastic semantics, graphable.
\\Input to commands: \\\null\qquad\hyperref[command:Ebi analyse all-traces]{\texttt{Ebi analyse all-traces}} (Section~\ref{command:Ebi analyse all-traces})\\\null\qquad\hyperref[command:Ebi analyse coverage]{\texttt{Ebi analyse coverage}} (Section~\ref{command:Ebi analyse coverage})\\\null\qquad\hyperref[command:Ebi analyse minimum-probability-traces]{\texttt{Ebi analyse minimum-probability-traces}} (Section~\ref{command:Ebi analyse minimum-probability-traces})\\\null\qquad\hyperref[command:Ebi analyse mode]{\texttt{Ebi analyse mode}} (Section~\ref{command:Ebi analyse mode})\\\null\qquad\hyperref[command:Ebi analyse most-likely-traces]{\texttt{Ebi analyse most-likely-traces}} (Section~\ref{command:Ebi analyse most-likely-traces})\\\null\qquad\hyperref[command:Ebi analyse-non-stochastic any-traces]{\texttt{Ebi analyse-non-stochastic any-traces}} (Section~\ref{command:Ebi analyse-non-stochastic any-traces})\\\null\qquad\hyperref[command:Ebi analyse-non-stochastic bounded]{\texttt{Ebi analyse-non-stochastic bounded}} (Section~\ref{command:Ebi analyse-non-stochastic bounded})\\\null\qquad\hyperref[command:Ebi analyse-non-stochastic executions]{\texttt{Ebi analyse-non-stochastic executions}} (Section~\ref{command:Ebi analyse-non-stochastic executions})\\\null\qquad\hyperref[command:Ebi analyse-non-stochastic infinitely-many-traces]{\texttt{Ebi analyse-non-stochastic infinitely-many-traces}} (Section~\ref{command:Ebi analyse-non-stochastic infinitely-many-traces})\\\null\qquad\hyperref[command:Ebi conformance earth-movers-stochastic-conformance-sample]{\texttt{Ebi conformance earth-movers-stochastic-conformance-sample}} (Section~\ref{command:Ebi conformance earth-movers-stochastic-conformance-sample})\\\null\qquad\hyperref[command:Ebi conformance entropic-relevance]{\texttt{Ebi conformance entropic-relevance}} (Section~\ref{command:Ebi conformance entropic-relevance})\\\null\qquad\hyperref[command:Ebi conformance jensen-shannon]{\texttt{Ebi conformance jensen-shannon}} (Section~\ref{command:Ebi conformance jensen-shannon})\\\null\qquad\hyperref[command:Ebi conformance jensen-shannon-sample]{\texttt{Ebi conformance jensen-shannon-sample}} (Section~\ref{command:Ebi conformance jensen-shannon-sample})\\\null\qquad\hyperref[command:Ebi conformance unit-earth-movers-stochastic-conformance]{\texttt{Ebi conformance unit-earth-movers-stochastic-conformance}} (Section~\ref{command:Ebi conformance unit-earth-movers-stochastic-conformance})\\\null\qquad\hyperref[command:Ebi conformance-non-stochastic alignments]{\texttt{Ebi conformance-non-stochastic alignments}} (Section~\ref{command:Ebi conformance-non-stochastic alignments})\\\null\qquad\hyperref[command:Ebi conformance-non-stochastic escaping-edges-precision]{\texttt{Ebi conformance-non-stochastic escaping-edges-precision}} (Section~\ref{command:Ebi conformance-non-stochastic escaping-edges-precision})\\\null\qquad\hyperref[command:Ebi conformance-non-stochastic set-alignments]{\texttt{Ebi conformance-non-stochastic set-alignments}} (Section~\ref{command:Ebi conformance-non-stochastic set-alignments})\\\null\qquad\hyperref[command:Ebi convert labelled-petri-net]{\texttt{Ebi convert labelled-petri-net}} (Section~\ref{command:Ebi convert labelled-petri-net})\\\null\qquad\hyperref[command:Ebi discover alignments]{\texttt{Ebi discover alignments}} (Section~\ref{command:Ebi discover alignments})\\\null\qquad\hyperref[command:Ebi discover occurrence labelled-petri-net]{\texttt{Ebi discover occurrence labelled-petri-net}} (Section~\ref{command:Ebi discover occurrence labelled-petri-net})\\\null\qquad\hyperref[command:Ebi discover occurrence process-tree]{\texttt{Ebi discover occurrence process-tree}} (Section~\ref{command:Ebi discover occurrence process-tree})\\\null\qquad\hyperref[command:Ebi discover uniform labelled-petri-net]{\texttt{Ebi discover uniform labelled-petri-net}} (Section~\ref{command:Ebi discover uniform labelled-petri-net})\\\null\qquad\hyperref[command:Ebi discover uniform process-tree]{\texttt{Ebi discover uniform process-tree}} (Section~\ref{command:Ebi discover uniform process-tree})\\\null\qquad\hyperref[command:Ebi discover-non-stochastic flower deterministic-finite-automaton]{\texttt{Ebi discover-non-stochastic flower deterministic-finite-automaton}} (Section~\ref{command:Ebi discover-non-stochastic flower deterministic-finite-automaton})\\\null\qquad\hyperref[command:Ebi discover-non-stochastic flower process-tree]{\texttt{Ebi discover-non-stochastic flower process-tree}} (Section~\ref{command:Ebi discover-non-stochastic flower process-tree})\\\null\qquad\hyperref[command:Ebi information]{\texttt{Ebi information}} (Section~\ref{command:Ebi information})\\\null\qquad\hyperref[command:Ebi probability explain-trace]{\texttt{Ebi probability explain-trace}} (Section~\ref{command:Ebi probability explain-trace})\\\null\qquad\hyperref[command:Ebi probability log]{\texttt{Ebi probability log}} (Section~\ref{command:Ebi probability log})\\\null\qquad\hyperref[command:Ebi probability trace]{\texttt{Ebi probability trace}} (Section~\ref{command:Ebi probability trace})\\\null\qquad\hyperref[command:Ebi sample]{\texttt{Ebi sample}} (Section~\ref{command:Ebi sample})\\\null\qquad\hyperref[command:Ebi validate]{\texttt{Ebi validate}} (Section~\ref{command:Ebi validate})\\\null\qquad\hyperref[command:Ebi visualise graph]{\texttt{Ebi visualise graph}} (Section~\ref{command:Ebi visualise graph})\\\null\qquad\hyperref[command:Ebi visualise text]{\texttt{Ebi visualise text}} (Section~\ref{command:Ebi visualise text}).
\\Output of commands: \\\null\qquad\hyperref[command:Ebi discover occurrence process-tree]{\texttt{Ebi discover occurrence process-tree}} (Section~\ref{command:Ebi discover occurrence process-tree})\\\null\qquad\hyperref[command:Ebi discover uniform process-tree]{\texttt{Ebi discover uniform process-tree}} (Section~\ref{command:Ebi discover uniform process-tree}).
\\A stochastic process tree (.sptree) can be neither imported nor exported between Ebi, and ProM and Java.
\\File format specification:
A stochastic process tree is a line-based structure. Lines starting with a \# are ignored.
    This first line is exactly `stochastic process tree'.
    The subsequent lines contain the nodes:
    Each node is either:
    \begin{itemize}
        \item A line with the word `activity' followed on the same line by a space and the label of the activity leaf. The next line contains the weight of the activity;
        \item The word `tau', followed on the next line by the weight of the leaf;
        \item The name of an operator (`sequence', `xor', `concurrent', `loop', `interleaved', or `or') on its own line.
        The line thereafter contains the number of children of the node, after which the nodes are given.
        An operator node must have at least one child.
    \end{itemize}
    Indentation of nodes is allowed, but not mandatory.\
    The last line of the file contains the weight of termination.
    
    For instance:
    \lstinputlisting[language=ebilines, style=boxed]{../testfiles/all_operators.sptree}
\clearpage
\subsection{Process tree markup language (.ptml)}
\label{filehandler:process tree markup language}
Import as objects: process tree, labelled Petri net.
\\Import as traits: semantics, graphable.
\\Input to commands: \\\null\qquad\hyperref[command:Ebi analyse-non-stochastic any-traces]{\texttt{Ebi analyse-non-stochastic any-traces}} (Section~\ref{command:Ebi analyse-non-stochastic any-traces})\\\null\qquad\hyperref[command:Ebi analyse-non-stochastic bounded]{\texttt{Ebi analyse-non-stochastic bounded}} (Section~\ref{command:Ebi analyse-non-stochastic bounded})\\\null\qquad\hyperref[command:Ebi analyse-non-stochastic executions]{\texttt{Ebi analyse-non-stochastic executions}} (Section~\ref{command:Ebi analyse-non-stochastic executions})\\\null\qquad\hyperref[command:Ebi analyse-non-stochastic infinitely-many-traces]{\texttt{Ebi analyse-non-stochastic infinitely-many-traces}} (Section~\ref{command:Ebi analyse-non-stochastic infinitely-many-traces})\\\null\qquad\hyperref[command:Ebi conformance-non-stochastic alignments]{\texttt{Ebi conformance-non-stochastic alignments}} (Section~\ref{command:Ebi conformance-non-stochastic alignments})\\\null\qquad\hyperref[command:Ebi conformance-non-stochastic escaping-edges-precision]{\texttt{Ebi conformance-non-stochastic escaping-edges-precision}} (Section~\ref{command:Ebi conformance-non-stochastic escaping-edges-precision})\\\null\qquad\hyperref[command:Ebi conformance-non-stochastic set-alignments]{\texttt{Ebi conformance-non-stochastic set-alignments}} (Section~\ref{command:Ebi conformance-non-stochastic set-alignments})\\\null\qquad\hyperref[command:Ebi convert labelled-petri-net]{\texttt{Ebi convert labelled-petri-net}} (Section~\ref{command:Ebi convert labelled-petri-net})\\\null\qquad\hyperref[command:Ebi discover alignments]{\texttt{Ebi discover alignments}} (Section~\ref{command:Ebi discover alignments})\\\null\qquad\hyperref[command:Ebi discover occurrence labelled-petri-net]{\texttt{Ebi discover occurrence labelled-petri-net}} (Section~\ref{command:Ebi discover occurrence labelled-petri-net})\\\null\qquad\hyperref[command:Ebi discover occurrence process-tree]{\texttt{Ebi discover occurrence process-tree}} (Section~\ref{command:Ebi discover occurrence process-tree})\\\null\qquad\hyperref[command:Ebi discover uniform labelled-petri-net]{\texttt{Ebi discover uniform labelled-petri-net}} (Section~\ref{command:Ebi discover uniform labelled-petri-net})\\\null\qquad\hyperref[command:Ebi discover uniform process-tree]{\texttt{Ebi discover uniform process-tree}} (Section~\ref{command:Ebi discover uniform process-tree})\\\null\qquad\hyperref[command:Ebi information]{\texttt{Ebi information}} (Section~\ref{command:Ebi information})\\\null\qquad\hyperref[command:Ebi validate]{\texttt{Ebi validate}} (Section~\ref{command:Ebi validate})\\\null\qquad\hyperref[command:Ebi visualise graph]{\texttt{Ebi visualise graph}} (Section~\ref{command:Ebi visualise graph})\\\null\qquad\hyperref[command:Ebi visualise text]{\texttt{Ebi visualise text}} (Section~\ref{command:Ebi visualise text}).
\\Output of commands: \\\null\qquad\hyperref[command:Ebi discover occurrence process-tree]{\texttt{Ebi discover occurrence process-tree}} (Section~\ref{command:Ebi discover occurrence process-tree})\\\null\qquad\hyperref[command:Ebi discover uniform process-tree]{\texttt{Ebi discover uniform process-tree}} (Section~\ref{command:Ebi discover uniform process-tree})\\\null\qquad\hyperref[command:Ebi discover-non-stochastic flower process-tree]{\texttt{Ebi discover-non-stochastic flower process-tree}} (Section~\ref{command:Ebi discover-non-stochastic flower process-tree})\\\null\qquad\hyperref[command:Ebi discover-non-stochastic prefix-tree process-tree]{\texttt{Ebi discover-non-stochastic prefix-tree process-tree}} (Section~\ref{command:Ebi discover-non-stochastic prefix-tree process-tree}).
\\A process tree markup language (.ptml) can be neither imported nor exported between Ebi, and ProM and Java.
\\File format specification:
A process tree markup language file.
For instance:
    \lstinputlisting[language=xml, style=boxed]{../testfiles/aa-ab-ba.ptml}
}
\def\ebifilehandlerlist{\begin{itemize}
\item compressed event log (.xes.gz) (Section~\ref{filehandler:compressed event log})
\item directly follows graph (.dfg) (Section~\ref{filehandler:directly follows graph})
\item deterministic finite automaton (.dfa) (Section~\ref{filehandler:deterministic finite automaton})
\item directly follows model (.dfm) (Section~\ref{filehandler:directly follows model})
\item stochastic directly follows model (.sdfm) (Section~\ref{filehandler:stochastic directly follows model})
\item event log (.xes) (Section~\ref{filehandler:event log})
\item executions (.exs) (Section~\ref{filehandler:executions})
\item finite language (.lang) (Section~\ref{filehandler:finite language})
\item finite stochastic language (.slang) (Section~\ref{filehandler:finite stochastic language})
\item labelled Petri net (.lpn) (Section~\ref{filehandler:labelled Petri net})
\item language of alignments (.ali) (Section~\ref{filehandler:language of alignments})
\item LoLa Petri net (.lola) (Section~\ref{filehandler:LoLa Petri net})
\item Petri net markup language (.pnml) (Section~\ref{filehandler:Petri net markup language})
\item stochastic deterministic finite automaton (.sdfa) (Section~\ref{filehandler:stochastic deterministic finite automaton})
\item stochastic labelled Petri net (.slpn) (Section~\ref{filehandler:stochastic labelled Petri net})
\item process tree (.ptree) (Section~\ref{filehandler:process tree})
\item portable document format (.pdf) (Section~\ref{filehandler:portable document format})
\item scalable vector graphics (.svg) (Section~\ref{filehandler:scalable vector graphics})
\item stochastic language of alignments (.sali) (Section~\ref{filehandler:stochastic language of alignments})
\item stochastic process tree (.sptree) (Section~\ref{filehandler:stochastic process tree})
\item process tree markup language (.ptml) (Section~\ref{filehandler:process tree markup language})
\end{itemize}}
\def\ebitraitlist{\begin{itemize}
\item Activities.
\\Has activities
\\File types that can be imported as  activities: \\\null\qquad\hyperref[filehandler:LoLa Petri net]{LoLa Petri net} (.lola -- Section~\ref{filehandler:LoLa Petri net}), \\\null\qquad\hyperref[filehandler:Petri net markup language]{Petri net markup language} (.pnml -- Section~\ref{filehandler:Petri net markup language}), \\\null\qquad\hyperref[filehandler:compressed event log]{compressed event log} (.xes.gz -- Section~\ref{filehandler:compressed event log}), \\\null\qquad\hyperref[filehandler:deterministic finite automaton]{deterministic finite automaton} (.dfa -- Section~\ref{filehandler:deterministic finite automaton}), \\\null\qquad\hyperref[filehandler:directly follows graph]{directly follows graph} (.dfg -- Section~\ref{filehandler:directly follows graph}), \\\null\qquad\hyperref[filehandler:directly follows model]{directly follows model} (.dfm -- Section~\ref{filehandler:directly follows model}), \\\null\qquad\hyperref[filehandler:event log]{event log} (.xes -- Section~\ref{filehandler:event log}), \\\null\qquad\hyperref[filehandler:finite language]{finite language} (.lang -- Section~\ref{filehandler:finite language}), \\\null\qquad\hyperref[filehandler:finite stochastic language]{finite stochastic language} (.slang -- Section~\ref{filehandler:finite stochastic language}), \\\null\qquad\hyperref[filehandler:labelled Petri net]{labelled Petri net} (.lpn -- Section~\ref{filehandler:labelled Petri net}), \\\null\qquad\hyperref[filehandler:language of alignments]{language of alignments} (.ali -- Section~\ref{filehandler:language of alignments}), \\\null\qquad\hyperref[filehandler:process tree]{process tree} (.ptree -- Section~\ref{filehandler:process tree}), \\\null\qquad\hyperref[filehandler:stochastic deterministic finite automaton]{stochastic deterministic finite automaton} (.sdfa -- Section~\ref{filehandler:stochastic deterministic finite automaton}), \\\null\qquad\hyperref[filehandler:stochastic directly follows model]{stochastic directly follows model} (.sdfm -- Section~\ref{filehandler:stochastic directly follows model}), \\\null\qquad\hyperref[filehandler:stochastic labelled Petri net]{stochastic labelled Petri net} (.slpn -- Section~\ref{filehandler:stochastic labelled Petri net}), \\\null\qquad\hyperref[filehandler:stochastic language of alignments]{stochastic language of alignments} (.sali -- Section~\ref{filehandler:stochastic language of alignments}), \\\null\qquad\hyperref[filehandler:stochastic process tree]{stochastic process tree} (.sptree -- Section~\ref{filehandler:stochastic process tree}).
\\Commands that accept  activities as input: \\\null\qquad\hyperref[command:Ebi discover-non-stochastic flower deterministic-finite-automaton]{\texttt{Ebi discover-non-stochastic flower deterministic-finite-automaton}} (Section~\ref{command:Ebi discover-non-stochastic flower deterministic-finite-automaton})\\\null\qquad\hyperref[command:Ebi discover-non-stochastic flower process-tree]{\texttt{Ebi discover-non-stochastic flower process-tree}} (Section~\ref{command:Ebi discover-non-stochastic flower process-tree})
\item Event log.
\\Has traces and attached event and trace attributes.
\\File types that can be imported as an event log: \\\null\qquad\hyperref[filehandler:compressed event log]{compressed event log} (.xes.gz -- Section~\ref{filehandler:compressed event log}), \\\null\qquad\hyperref[filehandler:event log]{event log} (.xes -- Section~\ref{filehandler:event log}).
\\Commands that accept an event log as input: \\\null\qquad\hyperref[command:Ebi analyse completeness]{\texttt{Ebi analyse completeness}} (Section~\ref{command:Ebi analyse completeness})\\\null\qquad\hyperref[command:Ebi analyse-non-stochastic executions]{\texttt{Ebi analyse-non-stochastic executions}} (Section~\ref{command:Ebi analyse-non-stochastic executions})\\\null\qquad\hyperref[command:Ebi association all-trace-attributes]{\texttt{Ebi association all-trace-attributes}} (Section~\ref{command:Ebi association all-trace-attributes})\\\null\qquad\hyperref[command:Ebi association trace-attribute]{\texttt{Ebi association trace-attribute}} (Section~\ref{command:Ebi association trace-attribute})\\\null\qquad\hyperref[command:Ebi discover directly-follows-graph]{\texttt{Ebi discover directly-follows-graph}} (Section~\ref{command:Ebi discover directly-follows-graph})\\\null\qquad\hyperref[command:Ebi test log-categorical-attribute]{\texttt{Ebi test log-categorical-attribute}} (Section~\ref{command:Ebi test log-categorical-attribute})
\item Finite language.
\\Finite set of traces.
\\File types that can be imported as a finite language: \\\null\qquad\hyperref[filehandler:compressed event log]{compressed event log} (.xes.gz -- Section~\ref{filehandler:compressed event log}), \\\null\qquad\hyperref[filehandler:event log]{event log} (.xes -- Section~\ref{filehandler:event log}), \\\null\qquad\hyperref[filehandler:finite language]{finite language} (.lang -- Section~\ref{filehandler:finite language}), \\\null\qquad\hyperref[filehandler:finite stochastic language]{finite stochastic language} (.slang -- Section~\ref{filehandler:finite stochastic language}).
\\Commands that accept a finite language as input: \\\null\qquad\hyperref[command:Ebi analyse-non-stochastic cluster]{\texttt{Ebi analyse-non-stochastic cluster}} (Section~\ref{command:Ebi analyse-non-stochastic cluster})\\\null\qquad\hyperref[command:Ebi analyse-non-stochastic medoid]{\texttt{Ebi analyse-non-stochastic medoid}} (Section~\ref{command:Ebi analyse-non-stochastic medoid})\\\null\qquad\hyperref[command:Ebi conformance-non-stochastic set-alignments]{\texttt{Ebi conformance-non-stochastic set-alignments}} (Section~\ref{command:Ebi conformance-non-stochastic set-alignments})\\\null\qquad\hyperref[command:Ebi discover-non-stochastic flower deterministic-finite-automaton]{\texttt{Ebi discover-non-stochastic flower deterministic-finite-automaton}} (Section~\ref{command:Ebi discover-non-stochastic flower deterministic-finite-automaton})\\\null\qquad\hyperref[command:Ebi discover-non-stochastic flower process-tree]{\texttt{Ebi discover-non-stochastic flower process-tree}} (Section~\ref{command:Ebi discover-non-stochastic flower process-tree})\\\null\qquad\hyperref[command:Ebi discover-non-stochastic prefix-tree deterministic-finite-automaton]{\texttt{Ebi discover-non-stochastic prefix-tree deterministic-finite-automaton}} (Section~\ref{command:Ebi discover-non-stochastic prefix-tree deterministic-finite-automaton})\\\null\qquad\hyperref[command:Ebi discover-non-stochastic prefix-tree process-tree]{\texttt{Ebi discover-non-stochastic prefix-tree process-tree}} (Section~\ref{command:Ebi discover-non-stochastic prefix-tree process-tree})\\\null\qquad\hyperref[command:Ebi probability log]{\texttt{Ebi probability log}} (Section~\ref{command:Ebi probability log})
\item Finite stochastic language.
\\Finite distribution of traces.
\\File types that can be imported as a finite stochastic language: \\\null\qquad\hyperref[filehandler:compressed event log]{compressed event log} (.xes.gz -- Section~\ref{filehandler:compressed event log}), \\\null\qquad\hyperref[filehandler:event log]{event log} (.xes -- Section~\ref{filehandler:event log}), \\\null\qquad\hyperref[filehandler:finite stochastic language]{finite stochastic language} (.slang -- Section~\ref{filehandler:finite stochastic language}).
\\Commands that accept a finite stochastic language as input: \\\null\qquad\hyperref[command:Ebi analyse all-traces]{\texttt{Ebi analyse all-traces}} (Section~\ref{command:Ebi analyse all-traces})\\\null\qquad\hyperref[command:Ebi analyse coverage]{\texttt{Ebi analyse coverage}} (Section~\ref{command:Ebi analyse coverage})\\\null\qquad\hyperref[command:Ebi analyse medoid]{\texttt{Ebi analyse medoid}} (Section~\ref{command:Ebi analyse medoid})\\\null\qquad\hyperref[command:Ebi analyse mode]{\texttt{Ebi analyse mode}} (Section~\ref{command:Ebi analyse mode})\\\null\qquad\hyperref[command:Ebi analyse most-likely-traces]{\texttt{Ebi analyse most-likely-traces}} (Section~\ref{command:Ebi analyse most-likely-traces})\\\null\qquad\hyperref[command:Ebi analyse variety]{\texttt{Ebi analyse variety}} (Section~\ref{command:Ebi analyse variety})\\\null\qquad\hyperref[command:Ebi conformance earth-movers-stochastic-conformance]{\texttt{Ebi conformance earth-movers-stochastic-conformance}} (Section~\ref{command:Ebi conformance earth-movers-stochastic-conformance})\\\null\qquad\hyperref[command:Ebi conformance earth-movers-stochastic-conformance-sample]{\texttt{Ebi conformance earth-movers-stochastic-conformance-sample}} (Section~\ref{command:Ebi conformance earth-movers-stochastic-conformance-sample})\\\null\qquad\hyperref[command:Ebi conformance entropic-relevance]{\texttt{Ebi conformance entropic-relevance}} (Section~\ref{command:Ebi conformance entropic-relevance})\\\null\qquad\hyperref[command:Ebi conformance jensen-shannon]{\texttt{Ebi conformance jensen-shannon}} (Section~\ref{command:Ebi conformance jensen-shannon})\\\null\qquad\hyperref[command:Ebi conformance jensen-shannon-sample]{\texttt{Ebi conformance jensen-shannon-sample}} (Section~\ref{command:Ebi conformance jensen-shannon-sample})\\\null\qquad\hyperref[command:Ebi conformance unit-earth-movers-stochastic-conformance]{\texttt{Ebi conformance unit-earth-movers-stochastic-conformance}} (Section~\ref{command:Ebi conformance unit-earth-movers-stochastic-conformance})\\\null\qquad\hyperref[command:Ebi conformance-non-stochastic alignments]{\texttt{Ebi conformance-non-stochastic alignments}} (Section~\ref{command:Ebi conformance-non-stochastic alignments})\\\null\qquad\hyperref[command:Ebi discover alignments]{\texttt{Ebi discover alignments}} (Section~\ref{command:Ebi discover alignments})\\\null\qquad\hyperref[command:Ebi discover directly-follows-graph]{\texttt{Ebi discover directly-follows-graph}} (Section~\ref{command:Ebi discover directly-follows-graph})\\\null\qquad\hyperref[command:Ebi discover occurrence labelled-petri-net]{\texttt{Ebi discover occurrence labelled-petri-net}} (Section~\ref{command:Ebi discover occurrence labelled-petri-net})\\\null\qquad\hyperref[command:Ebi discover occurrence process-tree]{\texttt{Ebi discover occurrence process-tree}} (Section~\ref{command:Ebi discover occurrence process-tree})\\\null\qquad\hyperref[command:Ebi sample]{\texttt{Ebi sample}} (Section~\ref{command:Ebi sample})\\\null\qquad\hyperref[command:Ebi test bootstrap-test]{\texttt{Ebi test bootstrap-test}} (Section~\ref{command:Ebi test bootstrap-test})
\item Graphable.
\\Can be visualised as a graph.
\\File types that can be imported as a graphable: \\\null\qquad\hyperref[filehandler:LoLa Petri net]{LoLa Petri net} (.lola -- Section~\ref{filehandler:LoLa Petri net}), \\\null\qquad\hyperref[filehandler:Petri net markup language]{Petri net markup language} (.pnml -- Section~\ref{filehandler:Petri net markup language}), \\\null\qquad\hyperref[filehandler:deterministic finite automaton]{deterministic finite automaton} (.dfa -- Section~\ref{filehandler:deterministic finite automaton}), \\\null\qquad\hyperref[filehandler:directly follows graph]{directly follows graph} (.dfg -- Section~\ref{filehandler:directly follows graph}), \\\null\qquad\hyperref[filehandler:directly follows model]{directly follows model} (.dfm -- Section~\ref{filehandler:directly follows model}), \\\null\qquad\hyperref[filehandler:labelled Petri net]{labelled Petri net} (.lpn -- Section~\ref{filehandler:labelled Petri net}), \\\null\qquad\hyperref[filehandler:process tree]{process tree} (.ptree -- Section~\ref{filehandler:process tree}), \\\null\qquad\hyperref[filehandler:process tree markup language]{process tree markup language} (.ptml -- Section~\ref{filehandler:process tree markup language}), \\\null\qquad\hyperref[filehandler:stochastic deterministic finite automaton]{stochastic deterministic finite automaton} (.sdfa -- Section~\ref{filehandler:stochastic deterministic finite automaton}), \\\null\qquad\hyperref[filehandler:stochastic directly follows model]{stochastic directly follows model} (.sdfm -- Section~\ref{filehandler:stochastic directly follows model}), \\\null\qquad\hyperref[filehandler:stochastic labelled Petri net]{stochastic labelled Petri net} (.slpn -- Section~\ref{filehandler:stochastic labelled Petri net}), \\\null\qquad\hyperref[filehandler:stochastic process tree]{stochastic process tree} (.sptree -- Section~\ref{filehandler:stochastic process tree}).
\\Commands that accept a graphable as input: \\\null\qquad\hyperref[command:Ebi visualise graph]{\texttt{Ebi visualise graph}} (Section~\ref{command:Ebi visualise graph})
\item Iterable language.
\\Can walk over the traces. May iterate over infinitely many traces.
\\File types that can be imported as an iterable language: \\\null\qquad\hyperref[filehandler:compressed event log]{compressed event log} (.xes.gz -- Section~\ref{filehandler:compressed event log}), \\\null\qquad\hyperref[filehandler:event log]{event log} (.xes -- Section~\ref{filehandler:event log}), \\\null\qquad\hyperref[filehandler:finite language]{finite language} (.lang -- Section~\ref{filehandler:finite language}), \\\null\qquad\hyperref[filehandler:finite stochastic language]{finite stochastic language} (.slang -- Section~\ref{filehandler:finite stochastic language}).
\\Commands that accept an iterable language as input: none
\item Iterable stochastic language.
\\Can walk over the traces and their probabilities. May iterate over infinitely many traces.
\\File types that can be imported as an iterable stochastic language: \\\null\qquad\hyperref[filehandler:compressed event log]{compressed event log} (.xes.gz -- Section~\ref{filehandler:compressed event log}), \\\null\qquad\hyperref[filehandler:event log]{event log} (.xes -- Section~\ref{filehandler:event log}), \\\null\qquad\hyperref[filehandler:finite stochastic language]{finite stochastic language} (.slang -- Section~\ref{filehandler:finite stochastic language}).
\\Commands that accept an iterable stochastic language as input: none
\item Queriable stochastic language.
\\Can query by giving a trace, which will return its probability.
\\File types that can be imported as a queriable stochastic language: \\\null\qquad\hyperref[filehandler:compressed event log]{compressed event log} (.xes.gz -- Section~\ref{filehandler:compressed event log}), \\\null\qquad\hyperref[filehandler:event log]{event log} (.xes -- Section~\ref{filehandler:event log}), \\\null\qquad\hyperref[filehandler:finite stochastic language]{finite stochastic language} (.slang -- Section~\ref{filehandler:finite stochastic language}), \\\null\qquad\hyperref[filehandler:stochastic deterministic finite automaton]{stochastic deterministic finite automaton} (.sdfa -- Section~\ref{filehandler:stochastic deterministic finite automaton}), \\\null\qquad\hyperref[filehandler:stochastic labelled Petri net]{stochastic labelled Petri net} (.slpn -- Section~\ref{filehandler:stochastic labelled Petri net}), \\\null\qquad\hyperref[filehandler:stochastic process tree]{stochastic process tree} (.sptree -- Section~\ref{filehandler:stochastic process tree}).
\\Commands that accept a queriable stochastic language as input: \\\null\qquad\hyperref[command:Ebi conformance entropic-relevance]{\texttt{Ebi conformance entropic-relevance}} (Section~\ref{command:Ebi conformance entropic-relevance})\\\null\qquad\hyperref[command:Ebi conformance jensen-shannon]{\texttt{Ebi conformance jensen-shannon}} (Section~\ref{command:Ebi conformance jensen-shannon})\\\null\qquad\hyperref[command:Ebi conformance unit-earth-movers-stochastic-conformance]{\texttt{Ebi conformance unit-earth-movers-stochastic-conformance}} (Section~\ref{command:Ebi conformance unit-earth-movers-stochastic-conformance})\\\null\qquad\hyperref[command:Ebi probability log]{\texttt{Ebi probability log}} (Section~\ref{command:Ebi probability log})\\\null\qquad\hyperref[command:Ebi probability trace]{\texttt{Ebi probability trace}} (Section~\ref{command:Ebi probability trace})
\item Semantics.
\\An object in which the state space can be traversed. Each deadlock is a final state, and each final state is a deadlock. Does not need to terminate, and may end up in livelocks.
\\File types that can be imported as a semantics: \\\null\qquad\hyperref[filehandler:LoLa Petri net]{LoLa Petri net} (.lola -- Section~\ref{filehandler:LoLa Petri net}), \\\null\qquad\hyperref[filehandler:Petri net markup language]{Petri net markup language} (.pnml -- Section~\ref{filehandler:Petri net markup language}), \\\null\qquad\hyperref[filehandler:compressed event log]{compressed event log} (.xes.gz -- Section~\ref{filehandler:compressed event log}), \\\null\qquad\hyperref[filehandler:deterministic finite automaton]{deterministic finite automaton} (.dfa -- Section~\ref{filehandler:deterministic finite automaton}), \\\null\qquad\hyperref[filehandler:directly follows graph]{directly follows graph} (.dfg -- Section~\ref{filehandler:directly follows graph}), \\\null\qquad\hyperref[filehandler:directly follows model]{directly follows model} (.dfm -- Section~\ref{filehandler:directly follows model}), \\\null\qquad\hyperref[filehandler:event log]{event log} (.xes -- Section~\ref{filehandler:event log}), \\\null\qquad\hyperref[filehandler:finite language]{finite language} (.lang -- Section~\ref{filehandler:finite language}), \\\null\qquad\hyperref[filehandler:finite stochastic language]{finite stochastic language} (.slang -- Section~\ref{filehandler:finite stochastic language}), \\\null\qquad\hyperref[filehandler:labelled Petri net]{labelled Petri net} (.lpn -- Section~\ref{filehandler:labelled Petri net}), \\\null\qquad\hyperref[filehandler:process tree]{process tree} (.ptree -- Section~\ref{filehandler:process tree}), \\\null\qquad\hyperref[filehandler:process tree markup language]{process tree markup language} (.ptml -- Section~\ref{filehandler:process tree markup language}), \\\null\qquad\hyperref[filehandler:stochastic deterministic finite automaton]{stochastic deterministic finite automaton} (.sdfa -- Section~\ref{filehandler:stochastic deterministic finite automaton}), \\\null\qquad\hyperref[filehandler:stochastic directly follows model]{stochastic directly follows model} (.sdfm -- Section~\ref{filehandler:stochastic directly follows model}), \\\null\qquad\hyperref[filehandler:stochastic labelled Petri net]{stochastic labelled Petri net} (.slpn -- Section~\ref{filehandler:stochastic labelled Petri net}), \\\null\qquad\hyperref[filehandler:stochastic process tree]{stochastic process tree} (.sptree -- Section~\ref{filehandler:stochastic process tree}).
\\Commands that accept a semantics as input: \\\null\qquad\hyperref[command:Ebi analyse-non-stochastic executions]{\texttt{Ebi analyse-non-stochastic executions}} (Section~\ref{command:Ebi analyse-non-stochastic executions})\\\null\qquad\hyperref[command:Ebi conformance-non-stochastic alignments]{\texttt{Ebi conformance-non-stochastic alignments}} (Section~\ref{command:Ebi conformance-non-stochastic alignments})\\\null\qquad\hyperref[command:Ebi conformance-non-stochastic escaping-edges-precision]{\texttt{Ebi conformance-non-stochastic escaping-edges-precision}} (Section~\ref{command:Ebi conformance-non-stochastic escaping-edges-precision})\\\null\qquad\hyperref[command:Ebi conformance-non-stochastic set-alignments]{\texttt{Ebi conformance-non-stochastic set-alignments}} (Section~\ref{command:Ebi conformance-non-stochastic set-alignments})
\item Stochastic deterministic semantics.
\\An object in which the state space can be traversed deterministically, that is, in each state every activity appears at most once and silent steps are not present. Each deadlock is a final state, and each final state is a deadlock. Does not need to terminate, and may end up in livelocks.
\\File types that can be imported as a stochastic deterministic semantics: \\\null\qquad\hyperref[filehandler:compressed event log]{compressed event log} (.xes.gz -- Section~\ref{filehandler:compressed event log}), \\\null\qquad\hyperref[filehandler:directly follows graph]{directly follows graph} (.dfg -- Section~\ref{filehandler:directly follows graph}), \\\null\qquad\hyperref[filehandler:event log]{event log} (.xes -- Section~\ref{filehandler:event log}), \\\null\qquad\hyperref[filehandler:finite stochastic language]{finite stochastic language} (.slang -- Section~\ref{filehandler:finite stochastic language}), \\\null\qquad\hyperref[filehandler:stochastic deterministic finite automaton]{stochastic deterministic finite automaton} (.sdfa -- Section~\ref{filehandler:stochastic deterministic finite automaton}), \\\null\qquad\hyperref[filehandler:stochastic directly follows model]{stochastic directly follows model} (.sdfm -- Section~\ref{filehandler:stochastic directly follows model}), \\\null\qquad\hyperref[filehandler:stochastic labelled Petri net]{stochastic labelled Petri net} (.slpn -- Section~\ref{filehandler:stochastic labelled Petri net}), \\\null\qquad\hyperref[filehandler:stochastic process tree]{stochastic process tree} (.sptree -- Section~\ref{filehandler:stochastic process tree}).
\\Commands that accept a stochastic deterministic semantics as input: \\\null\qquad\hyperref[command:Ebi analyse all-traces]{\texttt{Ebi analyse all-traces}} (Section~\ref{command:Ebi analyse all-traces})\\\null\qquad\hyperref[command:Ebi analyse coverage]{\texttt{Ebi analyse coverage}} (Section~\ref{command:Ebi analyse coverage})\\\null\qquad\hyperref[command:Ebi analyse minimum-probability-traces]{\texttt{Ebi analyse minimum-probability-traces}} (Section~\ref{command:Ebi analyse minimum-probability-traces})\\\null\qquad\hyperref[command:Ebi analyse mode]{\texttt{Ebi analyse mode}} (Section~\ref{command:Ebi analyse mode})\\\null\qquad\hyperref[command:Ebi analyse most-likely-traces]{\texttt{Ebi analyse most-likely-traces}} (Section~\ref{command:Ebi analyse most-likely-traces})
\item Stochastic semantics.
\\An object in which the state space can be traversed, with probabilities. Each deadlock is a final state, and each final state is a deadlock. Does not need to terminate, and may end up in livelocks.
\\File types that can be imported as a stochastic semantics: \\\null\qquad\hyperref[filehandler:compressed event log]{compressed event log} (.xes.gz -- Section~\ref{filehandler:compressed event log}), \\\null\qquad\hyperref[filehandler:directly follows graph]{directly follows graph} (.dfg -- Section~\ref{filehandler:directly follows graph}), \\\null\qquad\hyperref[filehandler:event log]{event log} (.xes -- Section~\ref{filehandler:event log}), \\\null\qquad\hyperref[filehandler:finite stochastic language]{finite stochastic language} (.slang -- Section~\ref{filehandler:finite stochastic language}), \\\null\qquad\hyperref[filehandler:stochastic deterministic finite automaton]{stochastic deterministic finite automaton} (.sdfa -- Section~\ref{filehandler:stochastic deterministic finite automaton}), \\\null\qquad\hyperref[filehandler:stochastic directly follows model]{stochastic directly follows model} (.sdfm -- Section~\ref{filehandler:stochastic directly follows model}), \\\null\qquad\hyperref[filehandler:stochastic labelled Petri net]{stochastic labelled Petri net} (.slpn -- Section~\ref{filehandler:stochastic labelled Petri net}), \\\null\qquad\hyperref[filehandler:stochastic process tree]{stochastic process tree} (.sptree -- Section~\ref{filehandler:stochastic process tree}).
\\Commands that accept a stochastic semantics as input: \\\null\qquad\hyperref[command:Ebi conformance earth-movers-stochastic-conformance-sample]{\texttt{Ebi conformance earth-movers-stochastic-conformance-sample}} (Section~\ref{command:Ebi conformance earth-movers-stochastic-conformance-sample})\\\null\qquad\hyperref[command:Ebi conformance jensen-shannon-sample]{\texttt{Ebi conformance jensen-shannon-sample}} (Section~\ref{command:Ebi conformance jensen-shannon-sample})\\\null\qquad\hyperref[command:Ebi probability explain-trace]{\texttt{Ebi probability explain-trace}} (Section~\ref{command:Ebi probability explain-trace})\\\null\qquad\hyperref[command:Ebi sample]{\texttt{Ebi sample}} (Section~\ref{command:Ebi sample})
\end{itemize}}
\def\ebiobjecttypelist{\begin{itemize}
\item alignments
\item stochastic language of alignments
\item stochastic deterministic finite automaton
\item deterministic finite automaton
\item directly follows model
\item stochastic directly follows model
\item event log
\item finite language
\item finite stochastic language
\item labelled Petri net
\item stochastic labelled Petri net
\item process tree
\item stochastic process tree
\item executions
\item directly follows graph
\item scalable vector graphics
\end{itemize}}
\def\promcommands{\\\null\qquad\hyperref[command:Ebi analyse completeness]{\texttt{Ebi analyse completeness}} (Section~\ref{command:Ebi analyse completeness})\\\null\qquad\hyperref[command:Ebi analyse variety]{\texttt{Ebi analyse variety}} (Section~\ref{command:Ebi analyse variety})\\\null\qquad\hyperref[command:Ebi analyse-non-stochastic any-traces]{\texttt{Ebi analyse-non-stochastic any-traces}} (Section~\ref{command:Ebi analyse-non-stochastic any-traces})\\\null\qquad\hyperref[command:Ebi analyse-non-stochastic bounded]{\texttt{Ebi analyse-non-stochastic bounded}} (Section~\ref{command:Ebi analyse-non-stochastic bounded})\\\null\qquad\hyperref[command:Ebi analyse-non-stochastic infinitely-many-traces]{\texttt{Ebi analyse-non-stochastic infinitely-many-traces}} (Section~\ref{command:Ebi analyse-non-stochastic infinitely-many-traces})\\\null\qquad\hyperref[command:Ebi association all-trace-attributes]{\texttt{Ebi association all-trace-attributes}} (Section~\ref{command:Ebi association all-trace-attributes})\\\null\qquad\hyperref[command:Ebi association trace-attribute]{\texttt{Ebi association trace-attribute}} (Section~\ref{command:Ebi association trace-attribute})\\\null\qquad\hyperref[command:Ebi conformance earth-movers-stochastic-conformance]{\texttt{Ebi conformance earth-movers-stochastic-conformance}} (Section~\ref{command:Ebi conformance earth-movers-stochastic-conformance})\\\null\qquad\hyperref[command:Ebi conformance earth-movers-stochastic-conformance-sample]{\texttt{Ebi conformance earth-movers-stochastic-conformance-sample}} (Section~\ref{command:Ebi conformance earth-movers-stochastic-conformance-sample})\\\null\qquad\hyperref[command:Ebi conformance entropic-relevance]{\texttt{Ebi conformance entropic-relevance}} (Section~\ref{command:Ebi conformance entropic-relevance})\\\null\qquad\hyperref[command:Ebi conformance jensen-shannon]{\texttt{Ebi conformance jensen-shannon}} (Section~\ref{command:Ebi conformance jensen-shannon})\\\null\qquad\hyperref[command:Ebi conformance jensen-shannon-sample]{\texttt{Ebi conformance jensen-shannon-sample}} (Section~\ref{command:Ebi conformance jensen-shannon-sample})\\\null\qquad\hyperref[command:Ebi conformance unit-earth-movers-stochastic-conformance]{\texttt{Ebi conformance unit-earth-movers-stochastic-conformance}} (Section~\ref{command:Ebi conformance unit-earth-movers-stochastic-conformance})\\\null\qquad\hyperref[command:Ebi convert labelled-petri-net]{\texttt{Ebi convert labelled-petri-net}} (Section~\ref{command:Ebi convert labelled-petri-net})\\\null\qquad\hyperref[command:Ebi convert stochastic-finite-deterministic-automaton]{\texttt{Ebi convert stochastic-finite-deterministic-automaton}} (Section~\ref{command:Ebi convert stochastic-finite-deterministic-automaton})\\\null\qquad\hyperref[command:Ebi discover alignments]{\texttt{Ebi discover alignments}} (Section~\ref{command:Ebi discover alignments})\\\null\qquad\hyperref[command:Ebi discover directly-follows-graph]{\texttt{Ebi discover directly-follows-graph}} (Section~\ref{command:Ebi discover directly-follows-graph})\\\null\qquad\hyperref[command:Ebi discover occurrence labelled-petri-net]{\texttt{Ebi discover occurrence labelled-petri-net}} (Section~\ref{command:Ebi discover occurrence labelled-petri-net})\\\null\qquad\hyperref[command:Ebi discover uniform labelled-petri-net]{\texttt{Ebi discover uniform labelled-petri-net}} (Section~\ref{command:Ebi discover uniform labelled-petri-net})\\\null\qquad\hyperref[command:Ebi discover-non-stochastic flower deterministic-finite-automaton]{\texttt{Ebi discover-non-stochastic flower deterministic-finite-automaton}} (Section~\ref{command:Ebi discover-non-stochastic flower deterministic-finite-automaton})\\\null\qquad\hyperref[command:Ebi discover-non-stochastic flower process-tree]{\texttt{Ebi discover-non-stochastic flower process-tree}} (Section~\ref{command:Ebi discover-non-stochastic flower process-tree})\\\null\qquad\hyperref[command:Ebi discover-non-stochastic prefix-tree deterministic-finite-automaton]{\texttt{Ebi discover-non-stochastic prefix-tree deterministic-finite-automaton}} (Section~\ref{command:Ebi discover-non-stochastic prefix-tree deterministic-finite-automaton})\\\null\qquad\hyperref[command:Ebi discover-non-stochastic prefix-tree process-tree]{\texttt{Ebi discover-non-stochastic prefix-tree process-tree}} (Section~\ref{command:Ebi discover-non-stochastic prefix-tree process-tree})\\\null\qquad\hyperref[command:Ebi information]{\texttt{Ebi information}} (Section~\ref{command:Ebi information})\\\null\qquad\hyperref[command:Ebi itself html]{\texttt{Ebi itself html}} (Section~\ref{command:Ebi itself html})\\\null\qquad\hyperref[command:Ebi itself java]{\texttt{Ebi itself java}} (Section~\ref{command:Ebi itself java})\\\null\qquad\hyperref[command:Ebi itself logo]{\texttt{Ebi itself logo}} (Section~\ref{command:Ebi itself logo})\\\null\qquad\hyperref[command:Ebi itself manual]{\texttt{Ebi itself manual}} (Section~\ref{command:Ebi itself manual})\\\null\qquad\hyperref[command:Ebi probability log]{\texttt{Ebi probability log}} (Section~\ref{command:Ebi probability log})\\\null\qquad\hyperref[command:Ebi test bootstrap-test]{\texttt{Ebi test bootstrap-test}} (Section~\ref{command:Ebi test bootstrap-test})\\\null\qquad\hyperref[command:Ebi test log-categorical-attribute]{\texttt{Ebi test log-categorical-attribute}} (Section~\ref{command:Ebi test log-categorical-attribute})\\\null\qquad\hyperref[command:Ebi visualise text]{\texttt{Ebi visualise text}} (Section~\ref{command:Ebi visualise text})}
\def\ebiprominput{\begin{itemize}
\item Activities (trait).\\Java class: org.processmining.stochasticlabelledpetrinets.StochasticLabelledPetriNetSimpleWeights, org.processmining.plugins.InductiveMiner.efficienttree.EfficientTree, org.deckfour.xes.model.XLog, org.processmining.acceptingpetrinet.models.AcceptingPetriNet.
\item Event log (trait).\\Java class: org.deckfour.xes.model.XLog.
\item Finite language (trait).\\Java class: org.deckfour.xes.model.XLog.
\item Finite stochastic language (trait).\\Java class: org.deckfour.xes.model.XLog.
\item Graphable (trait).\\Java class: org.processmining.plugins.InductiveMiner.efficienttree.EfficientTree, org.processmining.acceptingpetrinet.models.AcceptingPetriNet, org.processmining.stochasticlabelledpetrinets.StochasticLabelledPetriNetSimpleWeights.
\item Iterable language (trait).\\Java class: org.deckfour.xes.model.XLog.
\item Iterable stochastic language (trait).\\Java class: org.deckfour.xes.model.XLog.
\item Queriable stochastic language (trait).\\Java class: org.processmining.stochasticlabelledpetrinets.StochasticLabelledPetriNetSimpleWeights, org.deckfour.xes.model.XLog.
\item Semantics (trait).\\Java class: org.deckfour.xes.model.XLog, org.processmining.acceptingpetrinet.models.AcceptingPetriNet, org.processmining.plugins.InductiveMiner.efficienttree.EfficientTree, org.processmining.stochasticlabelledpetrinets.StochasticLabelledPetriNetSimpleWeights.
\item Stochastic deterministic semantics (trait).\\Java class: org.processmining.stochasticlabelledpetrinets.StochasticLabelledPetriNetSimpleWeights, org.deckfour.xes.model.XLog.
\item Stochastic semantics (trait).\\Java class: org.processmining.stochasticlabelledpetrinets.StochasticLabelledPetriNetSimpleWeights, org.deckfour.xes.model.XLog.
\item Stochastic deterministic finite automaton.\\Java class: org.deckfour.xes.model.XLog.
\item Event log.\\Java class: org.deckfour.xes.model.XLog.
\item Finite stochastic language.\\Java class: org.deckfour.xes.model.XLog.
\item Labelled petri net.\\Java class: org.processmining.plugins.InductiveMiner.efficienttree.EfficientTree, org.processmining.stochasticlabelledpetrinets.StochasticLabelledPetriNetSimpleWeights, org.processmining.acceptingpetrinet.models.AcceptingPetriNet.
\item Stochastic labelled petri net.\\Java class: org.processmining.stochasticlabelledpetrinets.StochasticLabelledPetriNetSimpleWeights.
\item Process tree.\\Java class: org.processmining.plugins.InductiveMiner.efficienttree.EfficientTree.
\item Event log.\\Java class: org.deckfour.xes.model.XLog.
\item Object.\\Java class: org.processmining.plugins.InductiveMiner.efficienttree.EfficientTree, org.processmining.stochasticlabelledpetrinets.StochasticLabelledPetriNetSimpleWeights, org.processmining.acceptingpetrinet.models.AcceptingPetriNet, org.deckfour.xes.model.XLog.
\item Text.\\Java class: String.
\item Integer.\\Java class: Integer.
\item Fraction.\\Java class: org.apache.commons.math3.fraction.BigFraction.
\end{itemize}}
\def\ebipromoutput{\begin{itemize}
\item Stochastic deterministic finite automaton.\\Java class: org.processmining.models.graphbased.directed.petrinet.Petrinet.
\item Deterministic finite automaton.\\Java class: org.processmining.models.graphbased.directed.petrinet.Petrinet.
\item Directly follows model.\\Java class: org.processmining.models.graphbased.directed.petrinet.Petrinet.
\item Stochastic directly follows model.\\Java class: org.processmining.stochasticlabelledpetrinets.StochasticLabelledPetriNetSimpleWeights, org.processmining.models.graphbased.directed.petrinet.Petrinet.
\item Event log.\\Java class: org.deckfour.xes.model.XLog.
\item Labelled petri net.\\Java class: org.processmining.models.graphbased.directed.petrinet.Petrinet.
\item Stochastic labelled petri net.\\Java class: org.processmining.stochasticlabelledpetrinets.StochasticLabelledPetriNetSimpleWeights, org.processmining.models.graphbased.directed.petrinet.Petrinet.
\item Process tree.\\Java class: org.processmining.models.graphbased.directed.petrinet.Petrinet, org.processmining.plugins.InductiveMiner.efficienttree.EfficientTree.
\item Directly follows graph.\\Java class: org.processmining.models.graphbased.directed.petrinet.Petrinet, org.processmining.stochasticlabelledpetrinets.StochasticLabelledPetriNetSimpleWeights.
\item Text.\\Java class: String.
\item Integer.\\Java class: Integer.
\item Fraction.\\Java class: org.apache.commons.math3.fraction.BigFraction.
\item Logarithm.\\Java class: org.processmining.ebi.objects.EbiLogDiv.
\item Root.\\Java class: org.processmining.framework.util.HTMLToString.
\item Rootlog.\\Java class: org.processmining.framework.util.HTMLToString.
\item Bool.\\Java class: java.lang.Boolean.
\end{itemize}}

